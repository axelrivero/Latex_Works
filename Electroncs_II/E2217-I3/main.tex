\documentclass[a4paper,12pt,spanish]{article}

% Configuración de codificación y lenguaje
\usepackage[utf8]{inputenc}
\usepackage[spanish]{babel}

% Configuración de fuentes y estilo de página
\usepackage{newtxtext,newtxmath}
\usepackage{titlesec}
\usepackage{fancyhdr}
\usepackage[left=2.5cm, right=2.5cm, top=2.5cm, bottom=2.5cm, headheight=20pt]{geometry}

% Configuración de gráficos y figuras
\usepackage{graphicx}
\usepackage{wrapfig}
\usepackage{caption}
\usepackage{float}
\usepackage{pgfplots}
\usepackage{pdfpages}
\usepackage{standalone}

% Configuración de tablas y listas
\usepackage{array}
\usepackage{booktabs}
\usepackage{multirow}
\usepackage{enumitem}
\usepackage{csvsimple}
\usepackage{tabularx}

% Configuración de enlaces y referencias
\usepackage{url}
\usepackage{hyperref}

% Configuración de matemáticas y unidades
\usepackage{amsmath, bm}
\usepackage{siunitx}

% Configuración de código fuente y comentarios
\usepackage{listings}
\usepackage{comment}

% Configuración de circuitos eléctricos
\usepackage{tikz}
\usepackage{circuitikz}
\usepackage{adjustbox}

% Configuración para ocultar recuadros de enlaces
\hypersetup{
    colorlinks=false, % No colorear los enlaces
    hidelinks          % Ocultar recuadros de enlaces
}


% Configuración personalizada
\pgfplotsset{compat=1.18}
\graphicspath{{Circuitos/}}
\newcommand{\volt}{\si{\volt}}
\newcommand{\amp}{\si{\amp}}
\newcommand{\ohm}{\si{\ohm}}
\newcommand{\watt}{\si{\watt}}

% Configuración de títulos de secciones y espacio entre párrafos
\titleformat{\section}{\normalfont\Large\bfseries}{\thesection}{1em}{}[\titlerule]
\captionsetup[table]{labelfont={bf}}
\captionsetup[figure]{labelfont={bf}}
\captionsetup[figure]{justification=centering, singlelinecheck=false}
\captionsetup[table]{justification=centering, singlelinecheck=false}

% Configuración de circuitos eléctricos con circuitikz
\ctikzset{
  monopoles/vcc/arrow={Triangle[width=0.8*\scaledwidth, length=\scaledwidth]},
  monopoles/vee/arrow={Triangle[width=6pt, length=8pt]},
  resistors/thickness=3,
}

% Configuración de encabezados y pies de página
\pagestyle{fancy}
\fancyhf{}
\fancyhead[L]{\leftmark}
\fancyhead[R]{\rightmark}
\fancyfoot[C]{\thepage}
\fancyfoot[L]{\textit{Electrónica II}}
\fancyfoot[R]{\thepage/\pageref{LastPage}}
\renewcommand{\headrulewidth}{0.4pt}
\renewcommand{\footrulewidth}{0.4pt}

% Configuración adicional
\setlength{\parskip}{5pt}


\begin{document}
\renewcommand{\tablename}{Tabla} %establece la palabra "Tabla" como el texto que precede al número.

% Portada

\begin{titlepage}
    \begin{center}
        \textsc{\large Universidad Central de Venezuela}\\
        \textsc{\large Facultad de Ingeniería}\\
        \textsc{\large Escuela de Ingeniería Eléctrica}\\
        \textsc{\large Departamento de Electrónica, Computación y Control}\\[7cm]
        
        {\huge \bfseries INFORME N°1}\\[0.2cm]
        {\Large AMPLIFICADORES DISCRETOS}\\[10cm]
        
        \begin{minipage}{0.4\textwidth}
            \begin{flushleft}
                \emph{Aux. Docente:}\\
                Escobar, Caleb 
            \end{flushleft}
        \end{minipage}
        \begin{minipage}{0.4\textwidth}
            \begin{flushright}
                \emph{Autor:}\\
                Br. Axel Rivero
            \end{flushright}
        \end{minipage}
        
        \vfill
        
        {\large \today}
        
    \end{center}
\end{titlepage}


%Tabla de contenidos

\tableofcontents

\newpage

%Introducción

\section{Introducción}
    
Los amplificadores operacionales, debido a su versatilidad, encuentran aplicaciones diversas. En este análisis, nos enfocaremos en evaluar la eficacia de estos componentes al ser retro alimentados positivamente con una ganancia de lazo mayor a uno, aprovechando su inestabilidad como una ventaja para la creación de funciones especializadas, como en el caso del generador de funciones.

En el desarrollo de esta práctica, inicialmente identificaremos diversos comportamientos asociados con topologías específicas. Este proceso nos permitirá evidenciar que la mencionada inestabilidad puede ser controlada y gestionada en diferentes etapas, posibilitando la obtención de señales de salida ajustadas a las necesidades particulares de la aplicación en cuestión.

Por otro lado, llevaremos a cabo el diseño detallado de cada una de estas topologías. Durante este proceso, se considerarán minuciosamente las características requeridas para optimizar la salida, asegurando así un resultado preciso y estable acorde con los objetivos planteados.

\newpage

%Resumen

\section{Resumen}

Esta practica n° 3 explora a fondo las aplicaciones no lineales de amplificadores operacionales en tres áreas fundamentales: Osciladores, Multivibradores y Generadores de Funciones. En la primera sección, dedicada a los osciladores, se destacan los principios detrás de la generación de oscilaciones sinusoidales y la importancia de la retroalimentación positiva para estabilizar y controlar la amplitud de las oscilaciones. La sección concluye resaltando la necesidad de un control de amplitud para mantener la oscilación sinusoidal.

La segunda sección se centra en los multivibradores, explorando los efectos de la realimentación positiva cuando la ganancia del lazo excede la unidad. Se analizan circuitos no lineales mediante el concepto de comparador, y se establecen las diferencias entre un dispositivo comparador y un amplificador operacional utilizado con esa finalidad. La aplicación práctica de los multivibradores en la generación de señales pulsantes se destaca como un aspecto crucial.

La tercera sección aborda los generadores de funciones, examinando los efectos de la realimentación positiva cuando la ganancia del lazo es mayor que la unidad. Se profundiza en el funcionamiento de generadores basados en amplificadores operacionales y se resalta la importancia de la selección adecuada de componentes y la configuración del circuito para lograr resultados precisos y estables.
\newpage

%Marco Teórico
\section{Marco Teórico}

    \subsection{Osciladores}

        Un oscilador en electrónica es un circuito que genera una señal eléctrica de forma periódica y autónoma. Algunas características clave de los osciladores son:

        \begin{itemize}
            \item Generan una señal alterna, normalmente una onda senoidal o rectangular. La frecuencia de oscilación depende de los componentes del circuito.
            \item No requieren una señal de entrada para funcionar, sino que generan la señal de forma autónoma mediante realimentación positiva.
            \item Constan de un amplificador y un circuito resonante (LC) en una configuración de realimentación positiva. El amplificador compensa las pérdidas del circuito resonante.
            \item Se utilizan en muchas aplicaciones como generadores de señales, relojes de circuitos digitales, sistemas de comunicaciones, etc.
        \end{itemize}

        \subsubsection{Osciladores sinusoidales}

            Son osciladores que generan una señal senoidal o casi senoidal. Estos osciladores emplean conceptos de la teoría de sistemas para crear un par de polos conjugados justo sobre el eje imaginario del plano complejo para mantener la oscilación sinusoidal sostenida. A pesar del nombre de osciladores lineales, se tiene que emplear alguna forma de no linealidad para obtener el control de la amplitud de la onda sinusoidal de salida. De hecho, todos los osciladores son esencialmente circuitos no lineales. Esto complica el trabajo de análisis y diseño de osciladores; ya que no es posible aplicar directamente métodos de transformación (plano s), pero se han perfeccionado técnicas por medio de las cuales se pueden obtener osciladores sinusoidales en dos pasos. El primero de estos es lineal, y fácilmente se pueden emplear métodos del dominio de la frecuencia para el análisis de circuitos de realimentación.

    \subsection{Lazo de realimentación de un oscilador}

        
        \begin{figure}[H]
            \centering
            \includegraphics[width=8cm]{Imagenes/1.png}
            \caption{Estructura básica de un oscilador con entrada igual a cero}
            \label{fig:1}
        \end{figure}
        
        Es la conexión entre la salida y la entrada de un oscilador que permite mantener y regenerar la señal oscilante. Proporciona la realimentación positiva necesaria para que el oscilador genere señales de forma autónoma y continua. La estructura básica de un oscilador sinodal consta de un amplificador en un lazo de realimentación positiva, como se muestra en el diagrama de bloques de la figura \ref{fig:1}. Es importante observar que, a diferencia del lazo de realimentación negativa, aquí la señal de realimentación \(x_f\) se suma con un signo positivo. Entonces, la ganancia de realimentación está dada por


        \begin{equation}
            A(s) = \frac{A(s)}{1 - A(s)\beta(s)} \label{eqn:7}
        \end{equation}

        Donde se observa el signo negativo en el denominador
        La ganancia de lazo del circuito de la figura es \( -A(s)\beta(s) \), pero para nuestro propósito aquí, es más conveniente cancelar el signo menos y definir la ganancia de lazo \( L(s) \) como

        \begin{equation}
            L(s) = A(s)\beta(s)
        \end{equation}


        La ecuación característica se convierte entonces en

        \begin{equation}
            -L(s) + 1 = 0
        \end{equation}
        
    \subsection{Criterio de Barkhausen}

        Si a una frecuencia específica \(f_0\) la ganancia de lazo \(A\beta\) es igual a la unidad, se deduce de la ecuación (7) que \(Af\) será infinita. Esto es, a esta frecuencia, el circuito tendrá una salida finita para la señal de entrada cero. Este circuito es, por definición, un oscilador. Entonces, la condición para que el lazo de realimentación de la ilustración 2 produzca oscilaciones sinodales de frecuencia \(w_0\) es que
        
        \begin{equation}
            A(s)\beta(s) = 1
        \end{equation}

        En otras palabras, el criterio de Barkhausen establece las condiciones necesarias para que un oscilador genere señales de forma sostenida:

    \subsection{Oscilador básico de puente de Wien}
    

        \begin{figure}[H]
            \centering
            \includegraphics[width=10cm]{Imagenes/3.png}
            \caption{Puente de Wien con su Diagrama de Bode en amplitud y fase}
            \label{fig:3}
        \end{figure}
        
        Es un oscilador que utiliza un amplificador operacional realimentado con un puente resistivo-capacitivo para generar oscilaciones sinusoidales. La frecuencia de oscilación depende de los valores de \(R\) y \(C\) del puente.

    \subsection{Control de Amplitud}
        Son técnicas utilizadas en los osciladores para estabilizar y limitar la amplitud de oscilación. Por ejemplo, mediante diodos en el circuito resonante o control automático de ganancia en el amplificador.

        \begin{figure}[H]
            \centering
            \includegraphics[width=8cm]{Imagenes/4.png}
            \caption{Oscildor con control de amplitud}
            \label{fig:4}
        \end{figure}

        El circuito de la figura \ref{fig:4} utiliza un circuito simple diodo-resistor para controlar el valor efectivo de R2. En niveles de señal bajos, los diodos están apagados, por lo tanto, la resistencia de \(100k\Omega\) no tiene ningún efecto. Entonces, se tiene que \(\frac{R2}{R1} = \frac{22.1}{10} = 2.21\), o bien \(T(jf_0) = \frac{(1 + 2.21)}{3} = 1.07 > 1\), lo que indica el surgimiento de la oscilación. Conforme la oscilación crece, los diodos son llevados de forma gradual a la conducción en medios ciclos alternados. En el límite de la conducción fuerte del diodo, efectivamente, \(R2\) cambiaría a \(18.1k\Omega\), donde se obtiene \(T(jf_0) = 0.937 < 1\). Sin embargo, antes de que alcance esta condición límite, la amplitud se estabilizará automáticamente en algún nivel intermedio de la conducción del diodo donde \(\frac{R2}{R1} = 2\) exactamente, o \(T(jf_0) = 1\).

        \subsection{Multivibradores}

            Son osciladores digitales que generan señales cuadradas o Rectangulares. Están formados por amplificadores realimentados con circuitos RC para controlar los tiempos de conmutación. Se clasifican en: Astable y monoestable.

            \subsubsection*{Multivibrador Astable}

                Multivibrador que no tiene estado estable, sino que conmuta continuamente entre
                dos estados. Genera una onda cuadrada periódica cuya frecuencia depende de los valores
                de \(R\) y \(C\).
                
                \subsubsection*{Multivibrador Monoestable}
                
                Multivibrador que tiene un estado estable y uno inestable. Al recibir un pulso de
                disparo cambia al estado inestable durante un tiempo determinado para luego volver al
                estado estable. Genera un único pulso rectangular por cada pulso de disparo.

            \subsection*{Generador de Funciones}

                Circuito electrónico capaz de generar diferentes formas de onda programables
                (senoidal, triangular, cuadrada, rampa, etc). Suele basarse en un oscilador controlado por
                voltaje (VCO).
                
                \subsection*{Histéresis}
                
                Es el fenómeno por el cual el estado de salida de un sistema depende de su estado
                anterior. Se utiliza en comparadores para evitar conmutaciones erráticas debido al ruido.
                
                \subsection*{Comparador}
                
                Circuito que compara dos señales de entrada y produce una salida determinada
                por la relación entre ellas. Tiene dos niveles de salida posibles, alto y bajo. Se usa en
                sistemas de instrumentación, convertidores A/D, etc.
                
                \subsection*{Comparador por Histéresis}
                
                Comparador diseñado intencionalmente para exhibir histéresis, los comparadores
                también pueden ser realimentados, solo que en su caso resulta más beneficiosa la
                realimentación positiva en la zona lineal, que la negativa comúnmente usada en los
                amplificadores operacionales. Al practicar la realimentación positiva en un comparador
                se obtiene fundamentalmente un nuevo comportamiento conocido como histéresis, en el
                cual los niveles de conmutación cambian con el estado (nivel de tensión) que se encuentre
                en dicho circuito.
                
                \subsection*{Generador de Onda Triangular}
                
                Circuito que genera una señal triangular, alternando rampas de subida y bajada. Se
                implementa cargando/descargando un capacitor entre dos niveles de voltaje con una
                constante de tiempo fija.




               

\newpage


%Objetivos generales y Específicos

\section{Objetivo General y Específico}

    \subsection*{Parte 1. Osciladores}
    
        \subsubsection*{Objetivo General:}
        Comprender los fundamentos físicos que posibilitan la generación de oscilaciones sinusoidales mediante el uso de amplificadores operacionales.
        
        \subsubsection*{Objetivos Específicos:}
        \begin{enumerate}
            \item Reconocer y entender las ventajas inherentes al empleo de amplificadores operacionales en el diseño y desarrollo de sistemas analógicos.
            \item Identificar la necesidad y la utilidad de implementar un control de amplitud como mecanismo esencial para el mantenimiento y sostenimiento de oscilaciones sinusoidales en un circuito.
            \item  Entender la importancia de la retroalimentación positiva en la generación de oscilaciones sinusoidales.
            \item Seleccionar adecuadamente componentes para lograr estabilidad y control de amplitud efectivo.
        \end{enumerate}
    
    \subsection*{Parte 2. Multivibradores}
    
        \subsubsection*{Objetivo General:}
        Analizar los efectos de la realimentación positiva cuando la ganancia del lazo es mayor que la unidad, centrándose en el estudio de multivibradores.
        
        \subsubsection*{Objetivos Específicos:}
        \begin{enumerate}
            \item Analizar circuitos no lineales mediante la aplicación del concepto de comparador.
            \item Diferenciar las características y funciones de un dispositivo comparador y de un amplificador operacional utilizado como comparador.
            \item Enfatizar en cómo la realimentación positiva influye en la estabilidad y comportamiento de los multivibradores.
        \end{enumerate}
    
    \subsection*{Parte 3. Generador de funciones}
    
        \subsubsection*{Objetivo General:}
        Examinar los efectos de la realimentación positiva en situaciones donde la ganancia del lazo es mayor que la unidad, centrándose en el análisis de generadores de funciones.
        
        \subsubsection*{Objetivos Específicos:}
        \begin{enumerate}
            \item Analizar el funcionamiento detallado de generadores de funciones, específicamente aquellos basados en amplificadores operacionales y retroalimentación positiva.
            \item Explorar circuitos no lineales empleando el concepto de comparador en el contexto de generadores de funciones.
            \item Observar cómo la realimentación positiva contribuye a la forma de onda generada.
        \end{enumerate}

    

\newpage

%Metodología
\section{Metodología}\label{sec:metodologia}
\subsection{Parte 1. Aplicaciones De Las Topologías Clásicas}

\begin{figure}[H]
  \centering
  \includegraphics[width=8cm]{Imagenes/topologias_basicas.png}
  \caption{Topologías básicas}
  \label{fig:topologia_basicas}
\end{figure}

\begin{enumerate}[label=\textbf{\arabic*.}, font=\bfseries]
    
    \item Determinar la conexiones necesarias ($JP_j$: On/Off y señales de entrada $V_i$) para obtener un:

    \begin{itemize}
        \item \textbf{Amplificador Inversor}
            \begin{figure}[H]
              \centering              
              \includestandalone{Circuitos/inversor}
              \caption{Configuración del Amplificador Inversor}
              \label{fig:inversor}
            \end{figure}

        Si se observa la figura \ref{fig:topologia_basicas} de las topologías básicas, se obtiene un amplificador inversor con la siguiente configuración:

            \begin{itemize}
                \item  Entrada: $V_1$
                \item  $JP_1:$ On
                \item  $JP_2,JP_3,JP_4:$ Off
            \end{itemize}
        
        De esa manera, se obtiene la figura \ref{fig:inversor}
\newpage

        \item \textbf{Amplificador No Inversor}

            \begin{figure}[H]
              \centering              \includestandalone{Circuitos/no_inversor}
              \caption{Configuración del Amplificador No Inversor}
              \label{fig:no_inversor}
            \end{figure}
            
            Si se observa la figura \ref{fig:topologia_basicas} de las topologías básicas, se obtiene un amplificador no inversor con la siguiente configuración:

            \begin{itemize}
                \item  Entrada: $V_3$
                \item  Tierra: $V_1$
                \item  $JP_1, JP_2,JP_3,JP_4:$ Off
            \end{itemize}
        
        De esa manera, se obtiene la figura \ref{fig:no_inversor}
            

        \item \textbf{Amplificador Restador}

            \begin{figure}[H]
              \centering              \includestandalone{Circuitos/restador}
              \caption{Configuración del Amplificador Restador}
              \label{fig:restador}
            \end{figure}
            
            Si se observa la figura \ref{fig:topologia_basicas} de las topologías básicas, se obtiene un amplificador restador con la siguiente configuración:

            \begin{itemize}
                \item  Entrada: $V_1$ y $V_3$
                \item  $JP_1:$ On
                \item  $JP_2,JP_3,JP_4:$ Off
            \end{itemize}
        
        De esa manera, se obtiene la figura \ref{fig:restador}


        \textbf{Nota:} $V_1$ y $V_3$ realizan un corto para poder utilizar ambos nodos para la entrada del oscilador, obteniendo un $V_{osc}$

        \item \textbf{Convertidor de Tensión a Corriente (Fuente de Corriente)}

            \begin{figure}[H]
              \centering              \includestandalone{Circuitos/fuente_de_corriente}
              \caption{Configuración del Convertidor de Tensión a Corriente (Fuente de Corriente)}
              \label{fig:fuente_de_corriente}
            \end{figure}
            
            Si se observa la figura \ref{fig:topologia_basicas} de las topologías básicas, se obtiene un convertidor de fuente de tensión a corriente con la siguiente configuración:

            \begin{itemize}
                \item  Entrada: $V_4$
                \item  Tierra: $V_1$
                \item  $JP_1,JP_3:$ On
                \item  $JP_2,JP_4:$ Off
                \item  Carga: $R_5$
            \end{itemize}
        
        De esa manera, se obtiene la figura \ref{fig:fuente_de_corriente}

        \textbf{Nota:} Se conectará una fuente de voltaje de Corriente continua para observar como a través de su tensión podemos modificar la corriente.

\newpage
        \item \textbf{Integrador No Inversor (Integrador de Boo)}

            \begin{figure}[H]
              \centering              \includestandalone{Circuitos/integrador_no_inversor}
              \caption{Configuración del Integrador No Inversor}
              \label{fig:integrador_no_inversor}
            \end{figure}
            
            Si se observa la figura \ref{fig:topologia_basicas} de las topologías básicas, se obtiene un integrador no inversor con la siguiente configuración:

            \begin{itemize}
                \item  Entrada (Onda cuadrada): $V_4$ y @1KHz
                \item  Tierra: $V_1$
                \item  $JP_1,JP_3:$ On
                \item  $JP_2,JP_4:$ Off
                \item  Carga: $R_5$
            \end{itemize}
        
        De esa manera, se obtiene la figura \ref{fig:integrador_no_inversor}
        
    \end{itemize}
\subsubsection{Diseño y Simulación}

    \item Escoja los valores de las resistencias para obtener un Restador de ganancia 2, un inversor de ganancia -2, un amplificador No Inversor. Para el integrador utilice un condensador de poliéster de 10nF. Realice la simulación del circuito para verificar el resultado obtenido en sus cálculos. Explique cualquier diferencia respecto a sus cálculos, si la hay. Para probar la fuente de corriente, utilice diferentes valores de la resistencia de carga.
   
  % {\Large \textbf{Diseño}}
    \begin{itemize}
        \item \textbf{Inversor}

            \begin{equation}
                A_v=\dfrac{V_o}{V_i}=-\dfrac{R_6}{R_1}=-\dfrac{2K\ohm}{1K\ohm} = -2
                \label{eqn:A_inversor}
            \end{equation}

            \begin{itemize}
\newpage            
                \item Simulación

                    \begin{figure}[H]
                      \centering
                      \renewcommand{\figurename}{Gráfica}
                        \setcounter{figure}{0}
                      \includegraphics[width=\textwidth]{Imagenes/inversor.png}
                      \caption{Forma de ondas de la señal de salida y entrada de la figura \ref{fig:inversor}}
                      \label{fig:sim_inversor}
                    \end{figure}
            \end{itemize}

            Como Se puede observar en la gráfica \ref{fig:sim_inversor}, los cálculos de la ecuación \ref{eqn:A_inversor} concuerdan con la simulación.

        \item \textbf{No Inversor}

            \begin{align}
                A_v&=2=\dfrac{V_o}{V_i}=1+\dfrac{R_6}{R_1} \nonumber \\[0.2cm]
                2&=1+\dfrac{R_6}{R_1} \Longrightarrow 2-1=\dfrac{R_6}{R_1} \nonumber \\[0.8cm]
                R_1&=R_6=2K\ohm                     
                \label{eqn:A_no_inversor}
            \end{align}

\newpage                
            \begin{itemize}
                \item Simulación
                    \begin{figure}[H]
                      \centering
                      \renewcommand{\figurename}{Gráfica}
                      \includegraphics[width=\textwidth]{Imagenes/no_inversor.png}
                      \caption{Forma de ondas de la señal de salida y entrada de la figura \ref{fig:no_inversor}}
                      \label{fig:sim_no_inversor}
                    \end{figure}
            \end{itemize}

            Como Se puede observar en la gráfica \ref{fig:sim_no_inversor}, el diseño que se realizo fue el adecuado debido a que  concuerdan con la simulación.
            
            \textbf{Nota:} El máximo voltaje de entrada es de $4V$ debido a que su ganancia es 2, su $V_o=8V$ despues de $8V$ se satura su salida.

        \item \textbf{Restador}

            Se toma en consideración que si se observa el nodo no inversor de la figura \ref{fig:restador}, siendo este un divisor de tensión se tiene un no inversor y de esta manera se puede hallar su ganancia, debido a esto se realizará una superposición para hallar la ganancia de ambas, de esa manera se tiene un inversor y un no inversor al "apagar" las fuentes de tensión aplicando lo anteriormente mencionado.

            \begin{gather}
                V_o=-\dfrac{R_6}{R_1}V_1+\dfrac{R_5}{R_5+R_3}V_3\left(1+\dfrac{R_6}{R_1}\right)\label{eqn:vo}\\[0.2cm]
                \dfrac{R_6}{R_1}=1 \Longrightarrow R_6=R_1 \label{eqn:1} \\[0.2cm] 
                \dfrac{R_5}{R_5+R_3}V_3 \left(1+\dfrac{R_6}{R_1}\right)=3
                \label{eqn:2}   
            \end{gather}

            Ahora se toma la ecuación \ref{eqn:1} en la ecuación \ref{eqn:2}, teniendo a continuación lo siguiente:

            \begin{gather}
                \dfrac{R_5}{R_5+R_3}V_3 \left(1+\dfrac{R_6}{R_1}\right)=3 \nonumber\\[0.5cm]
                \dfrac{R_5}{R_5+R_3}V_3 \left(1+\dfrac{R_1}{R_1}\right)=3 \Longrightarrow 2R_5=3R_3+3R_5 \nonumber\\[0.5cm]
                -3R_3=3R_5-2R_5 \nonumber\\[0.5cm]
                |-3R_3=R_5| \Longrightarrow R_5=3R_3 \label{eqn:3}           
            \end{gather}

            Por lo tanto, se tiene que el voltaje de salida $V_o$ de la ecuación \ref{eqn:vo} se le van a sustituir las ecuaciones \ref{eqn:1} y \ref{eqn:3}, obteniendo lo siguiente:

            \begin{gather}
                V_o=-V_1+\dfrac{-3R_3}{R_3-3R_3}(1+1)V_3=-V_1+3V_3 \label{eqn:vo_1}
            \end{gather}

            Se tiene la ecuación \ref{eqn:vo_1}, y sabiendo que $V_i=V_1=V_3$ se sustituye en la ecuación antes nombrada dando como resultado,

            \begin{gather}
                V_o=-V_i+3V_i \nonumber \\[0.5cm]
                \dfrac{V_o}{V_i}=2 \nonumber
            \end{gather}

            \begin{itemize}
                \item Simulación
                
                    Se unen las entradas debido a que no tenemos otra salida de voltaje de entrada en el laboratorio por esa razón lo colocamos como si fuese un común.

                    \begin{figure}[H]
                      \centering
                      \renewcommand{\figurename}{Gráfica}
                      \includegraphics[width=\textwidth]{Imagenes/restador.png}
                      \caption{Forma de ondas de la señal de salida y entrada de la figura \ref{fig:restador}}
                      \label{fig:sim_restador}
                    \end{figure}
            \end{itemize}
        \item \textbf{Convertidor de Tensión a Corriente (Fuente de Corriente)}

            \begin{figure}[H]
              \centering
              \renewcommand{\figurename}{Imagen}
              \setcounter{figure}{0}\includegraphics[width=\textwidth]{Imagenes/fuente.png}
              \caption*{}
              \label{fig:fuente}
            \end{figure}
            \begin{figure}[H]
              \centering
              \renewcommand{\figurename}{Imagen}
              \includegraphics[width=10cm]{Imagenes/fuente_2.png}
              \caption{Imagen de simplificación del circuito aplicando la Teorema de Blackman}
              \label{fig:fuente2}
            \end{figure}

            Por ser una fuente de corriente se tiene que la impedancia $Z_a\to \infty$, por esa razón se tiene lo siguiente:

            \begin{gather}
                Z_{aa'}=R_7//R_4 \label{eqn:zaa}\\[0.5cm]
                X_{31CC}=0 \label{eqn:x31cc}\\[0.5cm]
                X_{31CA}=R4 \label{eqn:x31ca}
            \end{gather}

            Haciendo uso de las ecuaciones \ref{eqn:zaa}, \ref{eqn:x31cc} y \ref{eqn:x31ca}, usando el teorema de Blackman se obtiene,

            \begin{gather}
                z_a=Z_{aa}\dfrac{1-AX_{31CC}}{1-AX_{31CA}} \nonumber\\[0.5cm]
                z_a=\dfrac{R_7R_4}{R_7+R_4}\dfrac{1-\left(1+\dfrac{R_6}{R_1}\right)(0)}{1-\left(1-\dfrac{R_6}{R_1}\right)R_4} \label{eqn:za}\\[0.5cm]
                R_7+R_4=R_4\left(1+\dfrac{R_6}{R_1}\right)=R_4\left(\dfrac{R_1+R_6}{R_1}\right) \nonumber\\[0.5cm]
                R_1(R_7+R_4)=R_4(R_1+R_6) \nonumber\\[0.5cm]
                R_1=R_4 \label{eqn:r1}\\[0.5cm]
                (R_7+R_4)=(R_1+R_6) \label{eqn:r7}\\[0.5cm]
                \text{Haciendo uso de la ecuación \ref{eqn:r1} la sustituimos en \ref{eqn:r7}}\nonumber\\[0.5cm]
                R_7=R_6\\[0.5cm]
                R_1=R_4=10k\ohm \nonumber\\[0.5cm]
                R_7=R_6=4k\ohm \nonumber
            \end{gather}

            \begin{figure}[H]
              \centering
              \renewcommand{\figurename}{Imagen}
              \includegraphics[width=10cm]{Imagenes/fuente_eq.png}
              \caption{Circuito equivalente de la fuente de corriente}
              \label{fig:fuenteeq}
            \end{figure}

            Por ley de ohm se tiene que $V_1=I_1R_4$ por lo tanto si se despeja $I=1mA$

            Por lo tanto, $V_o=R_LI_1=4K(1mA)=4V$
                
            \begin{itemize}
                \item Simulación

                    \begin{figure}[H]
                      \centering
                      \renewcommand{\figurename}{Gráfica}
                      \setcounter{figure}{3}
                      \includegraphics[width=\textwidth]{Imagenes/sim_fuente.png}
                      \caption{Forma de ondas de la señal de salida y entrada de la figura \ref{fig:fuente_de_corriente} con $R_L=1k$; Vo=0.5V; Io=0.5mA.}
                      \label{fig:sim_fuente}
                    \end{figure}
                    \begin{figure}[H]
                      \centering
                      \renewcommand{\figurename}{Gráfica}
                      \includegraphics[width=\textwidth]{Imagenes/sim_fuente2.png}
                      \caption{Forma de ondas de la señal de salida y entrada de la figura \ref{fig:fuente_de_corriente} con $R_L=3k$; Vo=1.5V; Io=0.5mA.}
                      \label{fig:sim_fuente2}
                    \end{figure}

                    Se mantuvo la corriente y vario el voltaje de salida, como se observa en los cálculos realizados.

                    \begin{figure}[H]
                      \centering
                      \renewcommand{\figurename}{Gráfica}
                      \includegraphics[width=\textwidth]{Imagenes/sim_fuente3.png}
                      \caption{Forma de ondas de la señal de salida y entrada de la figura \ref{fig:fuente_de_corriente} con $R_L=4k$; Vo=2V; Io=0.5mA.}
                      \label{fig:sim_fuente3}
                    \end{figure}

                    Solo Vario $V_o$

                    \begin{figure}[H]
                      \centering
                      \renewcommand{\figurename}{Gráfica}
                      \includegraphics[width=\textwidth]{Imagenes/sim_fuente4.png}
                      \caption{Forma de ondas de la señal de salida y entrada de la figura \ref{fig:fuente_de_corriente} con $R_L=10k$; Vo=5.004V; Io=0.5mA.}
                      \label{fig:sim_fuente4}
                    \end{figure}

                    $V_o$ aumento pero su corriente no sobre pasa el $1mA$, solo disminuye un poco.
            \end{itemize}
        \item \textbf{Integrador No Inversor (Integrador de Boo)}
            En este caso, en el voltaje de entrada va a ser una onda cuadrada para poder observar el voltaje de salida necesario, haciendo uso de la siguiente ecuación:

            \begin{gather}
                V_{o'}=\left(1+\dfrac{R_6}{R_1}\right)V_o \nonumber\\[0.5cm]
                V_{o'}=\left(\dfrac{R_1+R_6}{R_1}\right)\dfrac{1}{R_4}\dfrac{1}{2CS}V_i \nonumber\\[0.5cm]
                \dfrac{V_{o'}}{V_i}=H(s)=\dfrac{R_1+R_6}{R_1R_42CS}\quad \therefore \nonumber\\[0.5cm]
                \dfrac{V_o}{V_i}=\dfrac{1}{R_42CS} \label{eqn:H}\\[0.5cm]
                Z_{load}=\dfrac{1}{2CS}
            \end{gather}

            \begin{itemize}
                \item Simulación
                    \begin{figure}[H]
                      \centering
                      \renewcommand{\figurename}{Gráfica}
                      \includegraphics[width=\textwidth]{Imagenes/sim_intnoinver.png}
                      \caption{Forma de ondas de la señal de salida y entrada de la figura \ref{fig:integrador_no_inversor}}
                      \label{fig:sim_intnoinv}
                    \end{figure}

                    Se mantiene la configuración de las resistencias del circuito anterior, diferencia una señal cuadrada o un pulso de 10V
            \end{itemize}
    \end{itemize}
\end{enumerate}

\subsection{Parte 2. Amplificador Operacional Real}


\begin{table}[H]
    \centering
    \begin{tabular}{|c|c|}
        \hline
        \textbf{Componente} & \textbf{Valor} \\\hline
        $\mathbf{R_1}$ &  $100\si{\ohm}$ \\\hline
        $\mathbf{R_2}$ & $100 \si{\ohm}$  \\\hline
        $\mathbf{R_3}$ & $22 M \si{\ohm}$  \\\hline
        $\mathbf{R_4}$ & $22 M \si{\ohm}$   \\\hline
        $\mathbf{R_5}$ & $100k\si{\ohm}$  \\\hline
        $\mathbf{R_6}$  & $10k\si{\ohm}$ \\\hline
        $\mathbf{R_7}$  & $1k\si{\ohm}$ \\\hline
        $\mathbf{R_8}$  & $910\si{\ohm}$ \\\hline
        $\mathbf{R_9}$  & $10k\si{\ohm}$ \\\hline
        $\mathbf{R_{10}}$  & $91k\si{\ohm}$ \\\hline
        $\mathbf{R_v}$  & $1\approx 10 \si{\ohm}$ \\\hline
        $\mathbf{C}$  & $100 nF$ \\\hline
    \end{tabular}
    \caption{Lista de Componentes usados en el laboratorio n°2 de la Práctica n°2}
    \label{tab:componentes_1}
\end{table}

\begin{enumerate}[label=\textbf{\arabic*.}, font=\bfseries]
    \begin{figure}[H]
      \centering
      \setcounter{figure}{22}
      \includestandalone{Circuitos/AmplificadorReal}
      \caption{Medición de tensiones de Offset y corriente de polarización}
      \label{fig:amp_op_real}
    \end{figure}
    
    \item Haciendo uso del montaje indicado en el diagrama esquemático de la Figura \ref{fig:amp_op_real} explicar como medir la tensión de Offset y como medir la corriente de polarización de cada entrada.

    \begin{enumerate}
        \item Tensión Offset
        
            Para hallar la \textbf{tensión offset}, denotada como $V_{os}$, se va a cerrar los \textbf{Jumper(JP) 1 y 2}, de esa manera se obtiene la siguiente expresión:
            \begin{align*}
                V_{o} = \frac{R_{5}}{R_{2}}V_{os}    
            \end{align*}
        
            Se medirá la tensión de salida $V_o$, por esa razón, se despeja $V_{os}$, obteniendo de manera indirecta la \textbf{tensión offset}.
            \begin{equation}
                V_{os}=\dfrac{V_o}{1+\dfrac{R_5}{R_2}}
                \label{eqn:vos}
            \end{equation}

        \item Corriente de polarización 1
            Para hallar la \textbf{Corriente de polarización 1}, denotada como $I_{B_1}$, se cierra \textbf{JP 2} y se abre \textbf{JP 1}.
            \textbf{Nota:} Importante acotación para facilitar los cálculos es que la resistencia $R_1$ no se tomará en cuenta su caída de tensión, debido a que la corriente que pasa por allí es muy pequeña, en consecuencia se desprecia esa tensión. Por lo tanto, se obtiene lo siguiente:

            \begin{align*}
                V_{o} = (V_{os}-I_{B_1}R_3)\left(1+\dfrac{R_5}{R_2}\right)      
            \end{align*}

            Se medirá la tensión de salida $V_o$. Teniendo todos los demás datos exceptuando $I_{B_1}$, es la que se despejará, resultando la siguiente ecuación:

            \begin{equation}
                I_{B_1}=\dfrac{V_{os}\left(1+\dfrac{R_5}{R_2}\right)-V_o}{R_3\left(1+\dfrac{R_5}{R_2}\right)}
                \label{eqn:ib1}
            \end{equation}

            Se halla así la corriente de polarización 1, en la medición indirecta de la ecuación \ref{eqn:ib1}.

        \item Corriente de polarización 2

            Para hallar la \textbf{Corriente de polarización 2}, denotada como $I_{B_2}$, se cierra \textbf{JP 1} y se abre \textbf{JP 2}. Se toma en cuenta la nota anterior, se obtiene:

             \begin{align*}
                V_{o} = (V_{os}+I_{B_2}R_3)\left(1+\dfrac{R_5}{R_2}\right)      
            \end{align*}

            Se medirá la tensión de salida $V_o$. Teniendo todos los demás datos exceptuando $I_{B_2}$, es la que se despejará, resultando la siguiente ecuación:

            \begin{equation}
                I_{B_2}=\dfrac{V_o-V_{os}\left(1+\dfrac{R_5}{R_2}\right)}{R_4\left(1+\dfrac{R_5}{R_2}\right)}
                \label{eqn:ib2}
            \end{equation}

            Se halla así la corriente de polarización 2, en la medición indirecta de la ecuación \ref{eqn:ib2}.

        \item Corriente Offset

            Al hallar las corrientes de polarización de cada entrada, se puede hacer uso de la siguiente ecuación para conocer la \textbf{Corriente offset}

            \begin{equation}
                I_{os}=\left|I_{B_1}-I_{B_2}\right|
                \label{eqn:ios}
            \end{equation}
            
    \end{enumerate}

    \begin{figure}[H]
      \centering
      \includestandalone{Circuitos/GBWP}
      \caption{Medición del GBWP}
      \label{fig:GBWP}
    \end{figure}

    \item Mediante el montaje de la Figura \ref{fig:GBWP} explique como comprobar que el Producto del Ancho de Banda por la Ganancia (GBWP) se mantiene.

        En este caso, se verificará que con distintas configuraciones de la figura \ref{fig:GBWP}, se mantiene el GBWP, midiendo de manera experimental su frecuencia de corte en las distintas topologías (variando su frecuencia y observar su atenuación) y poder aproximar su respuesta en frecuencia.

        \textbf{Nota:} El producto de ancho de banda por ganancia, también conocido como producto ganancia-ancho de banda (GBWP), es un parámetro importante en el diseño de amplificadores operacionales. Se define como el producto de la ganancia de lazo cerrado por la banda de frecuencias. Este producto es una constante, lo que significa que si la ganancia disminuye, el ancho de banda aumenta, y viceversa.
        En el contexto de los amplificadores operacionales, el GBWP es crucial, ya que indica la relación inversa entre la ganancia y el ancho de banda. Un amplificador operacional sin realimentar tiene una ganancia considerable y un ancho de banda muy reducido. A partir de la frecuencia de corte, hay una caída de ganancia con una pendiente específica.

        \begin{itemize}
            \item \textbf{JP3 y JP4 abiertos}

                \begin{equation}
                    A_{2} = 1+ \frac{R_{10}+R_{9}}{R_{7}} = 102
                    \label{eqn:A2}
                \end{equation}

                Recordar que la primera ganancia es la que se obtiene en la frecuencia mas baja.

            \item \textbf{JP4 cerrado y JP3 abierto}

                \begin{equation}
                    A_{3} = 1+ \frac{R_{9}}{R_{7}} = 11
                    \label{eqn:A3}
                \end{equation}

            \item \textbf{JP4 y JP3 cerrados (Buffer)}

                \begin{equation}
                    A_{4} = 1 
                    \label{eqn:A4}
                \end{equation}

        \end{itemize}

    

    \begin{figure}[H]
      \centering
      \includestandalone{Circuitos/buffer}
      \caption{Medición de S.R., excursión máxima y corriente de corto circuito}
      \label{fig:buffer}
    \end{figure}

    \item Mediante el montaje de seguidor de tensión de la Figura \ref{fig:buffer}, indique como medir el Slew Rate (S.R. o tasa de variación en español), los limites máximos de excursión y la corriente de corto circuito.

        \begin{itemize}
            \item Slew Rate

                Antes de realizar el experimento colocar una frecuencia de 1KHz para luego poder realizar las variaciones. Se realizará con las siguientes instrucciones:
    
                Para medir el slew rate utilizando un osciloscopio, se debe conectar el osciloscopio a la salida del amplificador y configurarlo para mostrar la forma de onda de la señal de salida. Luego, se debe aplicar una señal de entrada triangular al amplificador y ajustar la frecuencia de la señal para que esté dentro del rango de operación del amplificador. A continuación, se debe medir el tiempo que tarda la señal de salida en cambiar desde el 10\% al 90\% de su valor máximo, y utilizar esta información para calcular el slew rate utilizando la siguiente fórmula:
                
                \begin{gather}
                    SR= \dfrac{\vartriangle V}{\vartriangle t} \label{eqn:sr}
                \end{gather}
     
                Donde SR es el slew rate, $\vartriangle V$ es el cambio en la tensión de salida y $\vartriangle t$ es el tiempo que tarda la señal de salida en cambiar desde el 10\% al 90\% de su valor máximo. Es importante tener en cuenta que el slew rate puede variar dependiendo de la frecuencia de la señal de entrada, por lo que se deben realizar mediciones en diferentes frecuencias para obtener una medida precisa del slew rate.
    
                \begin{figure}[H]
                  \centering
                  \renewcommand{\figurename}{Imagen}
                  \setcounter{figure}{2}
                  \includegraphics[width=8cm]{Imagenes/sr2.png}
                  \caption{Señal de entrada (negro) y señal de salida (Rojo) esta ultima con un tiempo de retardo por el S.R de la variación de frecuencia.}
                  \label{fig:sr2}
                \end{figure}
                \begin{figure}[H]
                  \centering
                  \renewcommand{\figurename}{Imagen}
                  \includegraphics[width=10cm]{Imagenes/sr.png}
                  \caption{Señal de salida y como observar las medidas en el osciloscopio.}
                  \label{fig:sr}
                \end{figure}
    
            \item Limites máximo de excursión

                Se sube solo el voltaje para observar la señal de salida cuando esta se distorsione, recordar que se debe colocar nuevamente la frecuencia en 1KHz

            \item Corriente de corto circuito
                \begin{figure}[H]
                  \centering
                  \setcounter{figure}{25}
                  \includestandalone{Circuitos/bufferv}
                  \caption{Medición de corriente de corto circuito}
                  \label{fig:bufferv}
                \end{figure}

                Para la corriente de cortocircuito, se puede usar la técnica de "\textbf{Resistencia de Carga virtual}",esto es colocar una resistencia virtual en serie con la carga real del circuito, lo que permite medir la caída de tensión a través de la carga virtual.

                La resistencia debe ser lo mas pequeña posible entre $1\si{\ohm}$ y $10\si{\ohm}$, mido la tensión sobre esta resistencia  y por ley de Ohm se puede hallar la corriente de cortocircuito.
        \end{itemize}  
\end{enumerate}

\newpage
\subsection{Parte 3. Filtros Activos}
    \subsubsection{Diseño}
    Para cada uno de los filtros que se muestran en la figura \ref{fig:var_estado}, \ref{fig:sallen_key} y \ref{fig:retro_multiples}.

    \begin{enumerate}
        \item Obtener su modelo circuital de entrada a cada una de sus salidas, observe la importancia de la función de transferencia.

             \begin{figure}[H]
                  \centering
                  \includegraphics[width=12cm]{Imagenes/var_estado.png}
                  \caption{Filtro de Variables de Estado}
                  \label{fig:var_estado}
            \end{figure}

            Se aplica superposición se halla la salida de cada una de las topologías del circuito de la figura \ref{fig:var_estado}.

            \begin{itemize}
                \item Salida $V_{hp}$
                    \begin{gather}
                        V_{hp}= -\dfrac{R_4}{R_3}V_{in} + \dfrac{R_6}{R_6+R_7}\left(1+\dfrac{R_4}{R_3||R_5}\right)V_{bp}-\dfrac{R_4}{R_5}V_{lp}
                        \label{eqn:vhp}
                    \end{gather}

                \item Salida $V_{bp}$
                    \begin{gather}
                        V_{bp}=-\dfrac{\dfrac{1}{SC_1}}{R_1}V_{hp}=-\dfrac{1}{SC_1R_1}V_{hp}
                        \label{eqn:vbp}
                    \end{gather}

                \item Salida $V_{lp}$
                    \begin{gather}
                        V_{lp}=-\dfrac{\dfrac{1}{SC_2}}{R_2}V_{bp}=-\dfrac{1}{SC_2R_2}V_{bp}
                        \label{eqn:vlp}
                    \end{gather}
            \end{itemize}

            Se hallará la función de transferencia de un filtro pasa bajo, haciendo uso del siguiente sistema de ecuación.
            \begin{equation*}
                \begin{cases}
                    V_{hp}= -\dfrac{R_4}{R_3}V_{in} + \dfrac{R_6}{R_6+R_7}\left(1+\dfrac{R_4}{R_3||R_5}\right)V_{bp}-\dfrac{R_4}{R_5}V_{lp}\\[0.5cm]
                    V_{bp}=-\dfrac{1}{SC_1R_1}V_{hp}\\[0.5cm]
                    V_{lp}=-\dfrac{1}{SC_2R_2}V_{bp}
                \end{cases}
            \end{equation*}

            De la ecuación \ref{eqn:vbp} se despeja $V_{hp}$ y se sustituye en la ecuación \ref{eqn:vhp}, en la ecuación \ref{eqn:vlp} despejamos $V_{bp}$ y se sustituye en \ref{eqn:vhp}, de esta manera se halla $\dfrac{V_{lp}}{V_{in}}$.

            \begin{gather*}
                \begin{cases}
                    V_{bp}=-SC_2R_2V_{lp}\\[0.5cm]
                    V_{hp}=-SC_1R_1V_{bp}
                \end{cases}
            \end{gather*}

            Se sustituye la ecuación $V_{bp}$ en $V_{hp}$.

            \begin{gather*}
                \begin{cases}
                    V_{bp}=-SC_2R_2V_{lp}\\[0.5cm]
                    V_{hp}=-SC_1R_1(-SC_2R_2V_{lp})=S^2C_1C_2R_1R_2V_{lp} 
                \end{cases}
            \end{gather*}

            Se sustituye la ecuación $V_{bp}$ y $V_{hp}$ en la ecuación \ref{eqn:vhp}.
            
            \begin{gather}
                S^2C_1C_2R_1R_2V_{lp}=-\dfrac{R_4}{R_3}V_{in}+ \dfrac{R_6}{R_6+R_7}\left(1+\dfrac{R_4}{R_3||R_5}\right)(-SC_2R_2V_{lp})-\dfrac{R_4}{R_5}V_{lp} \nonumber\\[0.5cm]
                \left[S^2C_1C_2R_1R_2V_{lp}+ \dfrac{R_6}{R_6+R_7}\left(1+\dfrac{R_4}{R_3||R_5}\right)(SC_2R_2V_{lp})+\dfrac{R_4}{R_5}V_{lp}=-\dfrac{R_4}{R_3}V_{in}\right]\dfrac{1}{C_1C_2R_1R_2}\nonumber\\[0.5cm]
                V_{lp}\left[S^2+\dfrac{\dfrac{SR_6}{R_6+R_7}\left(1+\dfrac{R_4}{R_3||R_5}\right)}{C_1R_1}+\dfrac{R_4}{R_5}\dfrac{1}{C_1C_2R_1R_2}\right]=-\dfrac{R_4}{R_3}V_{in}\dfrac{1}{C_1C_2R_1R_2}\nonumber\\[0.5cm]
                \dfrac{V{lp}}{V_{in}}=\dfrac{-\dfrac{R_4}{R_3}\dfrac{1}{C_1C_2R_1R_2}}{S^2+\dfrac{\dfrac{SR_6}{R_6+R_7}\left(1+\dfrac{R_4}{R_3||R_5}\right)}{C_1R_1}+\dfrac{R_4}{R_5}*\dfrac{1}{C_1C_2R_1R_2}} \label{eqn:4}
            \end{gather}
\newpage
            \begin{figure}[H]
                  \centering
                  \includegraphics[width=8cm]{Imagenes/sallen_key.png}
                  \caption{Filtro Pasa Bajos con Topologías Sallen-Key}
                  \label{fig:sallen_key}
            \end{figure}

            Se aplica el Método del Amplificador Desvanecido (MAD), aparte se facilitaran los cálculos tomando cada resistencia y capacitancia como admitancias.

            Se le aplicará MAD al Amplificador que tiene configuración no inversora, recordando su formula:

            \begin{gather*}
                A_f=X_{i0}+\dfrac{X_{i1}AX_{30}}{1-X_{31}A}\\[0.5cm]
                A=1+\dfrac{R_4}{R_3}\\[0.5cm]
                X_{i0}=\dfrac{V_0}{V_i}\bigg|_{A=0}=\dfrac{V_iA}{V_i}=0\\[0.5cm]
                X_{i1}=\dfrac{e_1}{V_i}\bigg|_{A=0}=\dfrac{y_1}{y_1+y_2+\dfrac{y_3y_4}{y_3+y_4}}\dfrac{y_3}{y_3+y_4}\\[0.5cm]
                X_{30}=\dfrac{V_0}{e_3}\bigg|_{V_i=0}=\dfrac{V_0}{V_0}=1\\[0.5cm]
                X_{31}=\dfrac{e_1}{e_3}\bigg|_{V_i=0}=\dfrac{y_2}{y_1+y_2+\dfrac{y_3y_4}{y_3+y_4}}\dfrac{y_3}{y_3+y_4}\\[0.5cm]
            \end{gather*}

            Se sustituye cada uno de los valores hallados en la fórmula de MAD, por lo tanto se obtiene lo siguiente:

            \begin{gather}
                A_f=0+\dfrac{\dfrac{y_1}{y_1+y_2+\dfrac{y_3y_4}{y_3+y_4}}\dfrac{y_3}{y_3+y_4}A(1)}{1-A\dfrac{y_2}{y_1+y_2+\dfrac{y_3y_4}{y_3+y_4}}\dfrac{y_3}{y_3+y_4}}=\dfrac{\dfrac{y_1}{\dfrac{(y_1+y_2)(y_3+y_4)+y_3y_4}{y_3+y_4}}\dfrac{y_3}{y_3+y_4}A}{1-A\dfrac{y_2}{\dfrac{(y_1+y_2)(y_3+y_4)+y_3y_4}{y_3+y_4}}\dfrac{y_3}{y_3+y_4}} \nonumber\\[0.5cm]
                A_f=\dfrac{\dfrac{y_1y_3A}{(y_1+y_2)(y_3+y_4)+y_3y_4}}{\dfrac{(y_1+y_2)(y_3+y_4)+y_3y_4-Ay_2y_3}{(y_1+y_2)(y_3+y_4)+y_3y_4}}=\dfrac{y_1y_3A}{(y_1+y_2)(y_3+y_4)+y_3y_4-Ay_2y_3}\nonumber\\[0.5cm]
                A_f=\dfrac{y_1y_3A}{(y_1+y_2+y_4-Ay_2)y_3+y_4(y_1+y_2)}=\dfrac{y_1y_3A}{(y_1+(1-A)y_2+y_4)y_3+y_4(y_1+y_2)}\nonumber\\[0.5cm]
                y_1=\dfrac{1}{R_1}; \quad y_2=SC_1; \quad y_3= \dfrac{1}{R_2}; \quad y_4=SC_2 \nonumber
            \end{gather}

            Se sustituyen cada uno de los valores de $y_1$, $y_2$, $y_3$ y $y_4$, en la ecuación de $A_f$, quedando lo siguiente:

            \begin{gather}
                A_f=\dfrac{\dfrac{1}{R_1}\dfrac{1}{R_2}A}{\left(\dfrac{1}{R_1}+(1+A)SC_1+SC_2\right)\dfrac{1}{R_2}+\left(\dfrac{1}{R_1}+SC_1\right)SC_2}\nonumber\\[0.5cm]
                A_f=\dfrac{\dfrac{A}{R_1R_2}}{\dfrac{1}{R_1R_2}+\dfrac{(1-A)SC_1}{R_2}+\dfrac{SC_2}{R_2}+\dfrac{SC_2}{R_1}+S^2C_1C_2}\nonumber\\[0.5cm]
                A_f=\dfrac{\dfrac{A}{R_1R_2C_1C_2}}{S^2+S\left(\dfrac{(1-A)}{C_2R_2}+\dfrac{1}{C_1R_2}+\dfrac{1}{C_1R_1}\right)+\dfrac{1}{R_1R_2C_1C_2}}\label{eqn:5}
            \end{gather}

\newpage
            \begin{figure}[H]
                  \centering
                  \includegraphics[width=8cm]{Imagenes/retro_multiples.png}
                  \caption{Filtro Pasa Bajos con Topologías de Retroalimentaciones Múltiples}
                  \label{fig:retro_multiples}
            \end{figure}

            Se aplica MAD y para simplificar los cálculos se trabajara en admitancias, al igual que la topología anterior.

            \begin{gather*}
                A=-\dfrac{1}{SC_5R_3}\\[0.5cm] 
                X_{i0}=0; \quad X_{i1}=\dfrac{y_1}{y_1+y_2+y_3+y_4}; \quad X_{30}=1; \quad X_{31}=\dfrac{y_4}{y_1+y_2+y_3+y_4}\\[0.5cm]
                A_f=\dfrac{\dfrac{y_1}{y_1+y_2+y_3+y_4}\left(-\dfrac{1}{SC_5R_3}\right)}{1-\dfrac{y_4}{y_1+y_2+y_3+y_4}\left(-\dfrac{1}{SC_5R_3}\right)}\\[0.5cm]
                A_f=-\dfrac{y_1}{(y_1+y_2+y_3+y_4)(SC_5R_3)+y_4}\\[0.5cm]
                y_1=\dfrac{1}{R_1}; \quad y_2=SC_2; \quad y_3=\dfrac{1}{R_3}; \quad y_4=\dfrac{1}{R_4}
            \end{gather*}

            Se sustituyen cada uno de los valores de $y_1$, $y_2$, $y_3$ y $y_4$, en la ecuación de $A_f$, quedando lo siguiente:

            \begin{gather}
                A_f=\dfrac{-\dfrac{1}{R_1}}{\left(\dfrac{1}{R_1}+SC_2+\dfrac{1}{R_3}+\dfrac{1}{R_4}\right)(SC_5R_3)+\dfrac{1}{R_4}}\nonumber\\[0.5cm]
                A_f=\dfrac{-\dfrac{1}{R_1}}{S^2C_5C_2R_3+SC_5R_3\left(\dfrac{1}{R_1}+\dfrac{1}{R_3}+\dfrac{1}{R_4}\right)+\dfrac{1}{R_4}}\nonumber\\[0.5cm]
                A_f=\dfrac{-\dfrac{1}{R_1R_3C_2C_5}}{S^2+\dfrac{S}{C_2}\left(\dfrac{1}{R_1}+\dfrac{1}{R_3}+\dfrac{1}{R_4}\right)+\dfrac{1}{R_4R_3C_2C_5}} \label{eqn:6}
            \end{gather}

            Recordando que la formula General de filtro pasa bajos es el siguiente:

            \begin{gather}
                H(S)=\dfrac{H_0W_0^2}{S^2+2\zeta w_0S+w_o^2} \label{eqn:HS}
            \end{gather}

        \item Especifique los componentes necesarios, en cada filtro, para obtener frecuencias de corte ($f_0$) de $2.7 KHz$ con factor de amortiguamiento ($\zeta$) de $0.707$, con ganancia de $2$ en la salida pasa bajos.

            \begin{itemize}
                \item \textbf{Filtro de Variables de Estado}
            
            
                    Se hace uso de la ecuación \ref{eqn:HS}, para hallar lo que se pide, empezando con la función de transferencia del filtro de Variables de Estado de la figura \ref{fig:var_estado}, observando la ecuación \ref{eqn:4}, se tiene:
        
                    \begin{gather*}
                        f_0=2.7 KHz ; \quad \zeta=0.707 ; \quad A=2 \\[0.5cm]
                        w_0=2\pi f_o=2\pi (2.7 K)=16.965 K\dfrac{rad}{s}
                    \end{gather*}
        
                    Debido que tenemos tres valores designados como lo son $f_o$, $\zeta$ y $w_0$ y 6 constantes, la cual se deben hallar sus valores para el diseño que se pide, se asumirán distintas constantes permitiendo facilitar el calculo del siguiente sistema de ecuaciones.
        
                    \begin{gather*}
                        \begin{cases}
                            w_0^2=\dfrac{R_4}{R_5C_1C_2R_1R_2}\\[0.5cm]
                            2\zeta w_0=\dfrac{ \dfrac{R_6}{R_6+R_7}\left(1+\dfrac{R_4}{R_3||R_5}\right)}{C_1R_1}\\[0.5cm]
                            H_0w_0^2=-\dfrac{R_4}{R_3}\dfrac{1}{C_1C_2R_1R_2}
                        \end{cases}
                    \end{gather*}
        
                    Se sustituye $w_0^2$ en $H_0w_0^2$ y de esta última despejamos $H_0$.
        
                    \begin{gather*}
                        H_0=-\dfrac{1}{w_0^2}\dfrac{R_4}{R_3}\dfrac{1}{C_1C_2R_1R_2}=-\dfrac{1}{\dfrac{R_4}{R_5C_1C_2R_1R_2}}\dfrac{R_4}{R_3}\dfrac{1}{C_1C_2R_1R_2}\\[0.5cm]
                        H_0=-\dfrac{R_5}{R_3}=-2 \quad \therefore \quad R_5=2R_3
                    \end{gather*}
        
                    Se asume que $R_5=10.2 K \ohm \quad \therefore \quad R_3=5.1K\ohm$ 
                    
                    Asumiendo valores a $C_1=C_2=10nF$ y tomando $R_5$ en la ecuación $w_0^2$, se tiene:
        
                    \begin{gather*}
                        w_0^2=\dfrac{R_4}{C_1C_2R_1R_2R_5}\\[0.5cm]
                        C_1C_2w_0^2=\dfrac{R_4}{R_1R_2R_5}
                    \end{gather*}
        
                    Asumiendo $R_1=3.3k\ohm$ y $R_2=2.2K\ohm$
        
                    \begin{gather*}
                        R_4=C_1C_2w_0^2R_1R_2R_5\\[0.5cm]
                        R_4=(10n)^2(16.965K)^2(3.3K)(2.2K)(10.2K)=2131.3 \approx 2.2K\ohm
                    \end{gather*}
        
                    Ahora se hace uso de los valores anteriormente hallados, para encontrar $R_6$ y $R_7$ en la ecuación $2\zeta w_0$ 
        
                    \begin{gather*}
                        2\zeta w_0=\dfrac{\dfrac{R_6}{R_6+R_7}\left(1+\dfrac{R_4}{R_3||R_5}\right)}{C_1R_1}\\[0.5cm]
                        R_3||R_5=\dfrac{R_3R_5}{R_3+R_5}= \dfrac{5.1K(10.2K)}{5.1K+10.2K}=3.4k\ohm\\[0.5cm]
                        2(0.707)(16.965K)=\dfrac{1}{10n(3.3K}\dfrac{R_6}{R_6R_7}\left(1+\dfrac{2.2K}{3.4K}\right)\\[0.5cm]
                        0.79162= \dfrac{R_6}{R_6R_7}(1.64705)\\[0.5cm]
                        \dfrac{0.79162}{1.64705}= \dfrac{R_6}{R_6R_7} \Longrightarrow 0.4806(R_6+R_7)=R_6 \Longrightarrow 0.4806R_7=0.51937R_6\\[0.5cm]
                        R_7=1.08R_6 \approx R_6
                    \end{gather*}
        
                    Teniendo en cuenta que $R_7\approx R_6$ se asume, $$R_7=R_6=10K\ohm$$
        
                    Asimismo se realizará para los filtros de la figura \ref{fig:sallen_key} y \ref{fig:retro_multiples}, continuamos con la Topología Sallen-Key.

                \item \textbf{Filtro Pasa Bajos con Topología Sallen-Key}
                
                    Como se indica en el proceso anterior, se hará uso de la ecuación \ref{eqn:HS} y la función de transferencia del filtro pasa bajos de la topología de Sallen-Key que es la ecuación \ref{eqn:5}. Obteniendo el siguiente sistema de ecuaciones:
        
                    \begin{gather*}
                        \begin{cases}
                            w_0^2=\dfrac{1}{R_1R_2C_1C_2}\\[0.5cm]
                            2\zeta w_0=\dfrac{(1-A)}{C_2R_2}+\dfrac{1}{C_1R_2}+\dfrac{1}{C_1R_1}=\dfrac{R_4}{C_2R_3R_2}+\dfrac{1}{C_1R_2}+\dfrac{1}{C_1R_1}\\[0.5cm]
                            H_0w_0^2=\dfrac{1}{R_1R_2C_1C_2}\dfrac{R_3+R_4}{R_3}
                        \end{cases}
                    \end{gather*}
        
                    Se sustituye $w_0^2$ en $H_0w_0^2$, obteniendo lo siguiente:
        
                    \begin{gather*}
                        H_0=\dfrac{1}{w_0^2}\dfrac{1}{R_1R_2C_1C_2}\dfrac{R_3+R_4}{R_3}\\[0.5cm]
                        H_0=\dfrac{1}{\dfrac{1}{R_1R_2C_1C_2}}\dfrac{1}{R_1R_2C_1C_2}\dfrac{R_3+R_4}{R_3}\\[0.5cm]
                        H_0=\dfrac{R_3+R_4}{R_3}=2 \Longrightarrow 2R_3=R_3+R_4 \Longrightarrow 2R_3-R_3=R_4\\[0.5cm]
                        R_3=R_4
                    \end{gather*}
        
                    Por esa razón, se asume que $$R_3=R_4=1K\ohm$$
        
                    Con los datos hallados los sustituimos en $2\zeta w_0$, de esa manera se halla $R_1$ y $R_2$. Se asume que $C_1=C_2=\SI{10}{\nano\farad}$

                    \begin{gather*}
                        2\zeta w_0= \dfrac{R_4}{C_2R_3R_2}+\dfrac{1}{C_1R_2}+\dfrac{1}{C_1R_1}\\[0.5cm]
                        2\zeta w_0=\dfrac{1}{R_2}\left(\dfrac{R_4}{C_1R_4}+\dfrac{1}{C_1}\right)+\dfrac{1}{C_1R_1}=\dfrac{2}{R_2C_1}+\dfrac{1}{C_1R_1}\\[0.5cm]
                        2\zeta w_0=\dfrac{2C_1R_1+R_2C_1}{C_1^2R_2R_1}=\dfrac{2R_1+R_2}{C_1R_2R_1}
                    \end{gather*}

                    Se asume que $R_1=3.9K\ohm$ y $R_2=8.2K\ohm$

                \item \textbf{Filtro Pasa Bajos con topología de retroalimentaciones múltiples}

                    Se toma en cuenta nuevamente la ecuación \ref{eqn:HS} y \ref{eqn:6}. Se forma el siguiente sistema de ecuaciones:

                    \begin{gather*}
                        \begin{cases}
                            w_0^2=\dfrac{1}{R_4R_3C_2C_5}\\[0.5cm]
                            H_0w_0^2=-\dfrac{1}{R_1R_3C_2C_5}\\[0.5cm]
                            2\zeta w_0=\dfrac{1}{C_2}\left(\dfrac{1}{R_1}+\dfrac{1}{R_3}+\dfrac{1}{R_4}\right)
                        \end{cases}
                    \end{gather*}

                    Se sustituye la ecuación $w_0^2$ en $H_0w_0^2$, se obtiene:

                    \begin{gather*}
                        H_0=-\dfrac{1}{w_0^2}\dfrac{1}{R_1R_3C_2C_5}=-\dfrac{1}{\dfrac{1}{R_4R_3C_2C_5}}\dfrac{1}{R_1R_3C_2C_5}=-\dfrac{R_4}{R_1}\\[0.5cm]
                        H_0=-2=-\dfrac{R_4}{R_1} \Longrightarrow R_4=2R_1
                    \end{gather*}

                    Por consiguiente, se asume que $R_1=1K\ohm '\quad \therefore \quad R_4=2K \approx 2.2k\ohm$

                    Ahora se usará la ecuación $2\zeta w_0$, para hallar $R_3$, asumiendo que $C_2=80nF$ y $C_5=10nF$

                    \begin{gather*}
                        2\zeta w_0=\dfrac{1}{C_2}\left(\dfrac{1}{R_1}+\dfrac{1}{R_3}+\dfrac{1}{R_4}\right)=\dfrac{1}{C_2}\left(\dfrac{2}{R_4}+\dfrac{1}{R_3}+\dfrac{1}{R_4}\right)\\[0.5cm]
                        2\zeta w_0C_2=\dfrac{3}{R_4}+\dfrac{1}{R_3}\\[0.5cm]
                        2(0.707)(2 \pi 2.7K)(80n)-\dfrac{3}{2.2K}=\dfrac{1}{R_3}=555.4\mu\\[0.5cm]
                        R_3=\dfrac{1}{555.4\mu}=1.8K\ohm
                    \end{gather*}

                    Se busca que $R_3$ y $R_4$ sean iguales, por lo tanto, $$R_3=R_4=1.8K\approx 2.2K$$
            \end{itemize}
        \subsubsection{Simulación}
        \item Verifique sus diseños, mediante simulación, comparando la Respuesta en frecuencia obtenida, con el diagrama asintótico de Bode de cada filtro. Determine la ganancia de cada filtro a las frecuencias en las que planea medir la Respuesta en frecuencia.

        \begin{itemize}
            \item \textbf{Filtro de Variables de Estado}

                Se muestra la siguiente tabla que indica los valores de las resistencias usadas en cada simulación, previamente calculadas tras su diseño.

                \begin{table}[H]
                    \centering
                    \begin{tabular}{|c|c|}
                        \hline
                        \textbf{Componente} & \textbf{Valor} \\\hline
                        $\mathbf{R_1}$ &  $3.3k\si{\ohm}$ \\\hline
                        $\mathbf{R_2}$ & $2.2k \si{\ohm}$  \\\hline
                        $\mathbf{R_3}$ & $3.1k \si{\ohm}$  \\\hline
                        $\mathbf{R_4}$ & $2.2k \si{\ohm}$   \\\hline
                        $\mathbf{R_5}$ & $10k\si{\ohm}$  \\\hline
                        $\mathbf{R_6}$  & $10k\si{\ohm}$ \\\hline
                        $\mathbf{R_7}$  & $10k\si{\ohm}$ \\\hline
                        $\mathbf{C_1}$  & $\SI{10}{\nano\farad}$ \\\hline
                        $\mathbf{C_2}$  & $\SI{10}{\nano\farad}$ \\\hline
                    \end{tabular}
                    \caption{Valores de los componentes que diseña un filtro pasa bajo de variables de estado}
                    \label{tab:diseño_var_estado}
                \end{table}

                \begin{figure}[H]
                      \centering
                      \includegraphics[width=15cm]{Imagenes/sim_var_estado_circuito.png}
                      \caption{Diseño del Filtro de Variables de Estado pasa bajos}
                      \label{fig:sim_var_estado_circuito}
                \end{figure}

                \begin{figure}[H]
                      \centering
                      \renewcommand{\figurename}{Gráfica}
                      \setcounter{figure}{8}
                      \includegraphics[width=15cm]{Imagenes/sim_var_estado_vo.png}
                      \caption{Señal de Salida y Entrada del Diseño del Filtro de Variables de Estado pasa bajos}
                      \label{fig:sim_var_estado_vo}
                \end{figure}

                Como se puede observar en la gráfica \ref{fig:sim_var_estado_vo} se obtiene una ganancia de -2, como la que se realizo en el diseño. Por consiguiente, falta añadir el diagrama de Bode para verificar si fue un  diseño adecuado a la respuesta en frecuencia deseada.

                \begin{figure}[H]
                      \centering
                      \renewcommand{\figurename}{Gráfica}
                      \includegraphics[width=15cm]{Imagenes/sim_var_estado_bode.png}
                      \caption{Diagrama de Bode (Magnitud y Fase) del Diseño del Filtro de Variables de Estado pasa bajos. Frecuencia de corte señalada en negro}
                      \label{fig:sim_var_estado_bode}
                \end{figure}

                Efectivamente los cálculos realizados en el diseño del  filtro son los correctos, para las condiciones deseadas en el laboratorio.

            \item \textbf{Filtro Pasa Bajos con Topología Sallen-Key}

                \begin{table}[H]
                    \centering
                    \begin{tabular}{|c|c|}
                        \hline
                        \textbf{Componente} & \textbf{Valor} \\\hline
                        $\mathbf{R_1}$ &  $3.9k\si{\ohm}$ \\\hline
                        $\mathbf{R_2}$ & $8.2k \si{\ohm}$  \\\hline
                        $\mathbf{R_3}$ & $1k \si{\ohm}$  \\\hline
                        $\mathbf{R_4}$ & $1k \si{\ohm}$   \\\hline
                        $\mathbf{C_1}$  & $\SI{10}{\nano\farad}$ \\\hline
                        $\mathbf{C_2}$  & $\SI{10}{\nano\farad}$ \\\hline
                    \end{tabular}
                    \caption{Valores de los componentes que diseña un filtro pasa bajo con topología Sallen-Key}
                    \label{tab:diseño_sallen_key}
                \end{table}

                \begin{figure}[H]
                      \centering
                      \setcounter{figure}{30}
                      \includegraphics[width=15cm]{Imagenes/sim_sallen_key_circuito.png}
                      \caption{Diseño del Filtro pasa bajos con topología Sallen-Key}
                      \label{fig:sim_sallen_key_circuito}
                \end{figure}

                \begin{figure}[H]
                      \centering
                      \renewcommand{\figurename}{Gráfica}
                      \setcounter{figure}{10}
                      \includegraphics[width=15cm]{Imagenes/sim_sallen_key_vo.png}
                      \caption{Señal de Salida y Entrada del Diseño del Filtro pasa bajos con topología Sallen-Key}
                      \label{fig:sim_sallen_key_vo}
                \end{figure}

                \begin{figure}[H]
                      \centering
                      \renewcommand{\figurename}{Gráfica}
                      \includegraphics[width=15cm]{Imagenes/sim_sallen_key_bode.png}
                      \caption{Diagrama de Bode (Magnitud y Fase) del Diseño del Filtro pasa bajos con topología Sallen-Key. Frecuencia de corte señalada en negro}
                      \label{fig:sim_sallen_key_bode}
                \end{figure}

                Como se visualiza en las gráficas   \ref{fig:sim_sallen_key_vo} y \ref{fig:sim_sallen_key_bode}, Se obtuvo una ganancia de 2, una frecuencia de corte de 2.7 KHz, con su error relativo apreciable, sin embargo, no deja de ser preciso.

            \item \textbf{Filtro Pasa Bajos con topología de retroalimentaciones múltiples}   

                \begin{table}[H]
                    \centering
                    \begin{tabular}{|c|c|}
                        \hline
                        \textbf{Componente} & \textbf{Valor} \\\hline
                        $\mathbf{R_1}$ &  $1k\si{\ohm}$ \\\hline
                        $\mathbf{R_3}$ & $2.2k \si{\ohm}$  \\\hline
                        $\mathbf{R_4}$ & $2.2k \si{\ohm}$   \\\hline
                        $\mathbf{C_2}$  & $\SI{80}{\nano\farad}$ \\\hline
                        $\mathbf{C_5}$  & $\SI{10}{\nano\farad}$ \\\hline
                    \end{tabular}
                    \caption{Valores de los componentes que diseña un filtro pasa bajo con topología de retroalimentaciones múltiples}
                    \label{tab:diseño_retro}
                \end{table}

                \begin{figure}[H]
                      \centering
                      \setcounter{figure}{31}
                      \includegraphics[width=15cm]{Imagenes/sim_retro_circuito.png}
                      \caption{Diseño del Filtro pasa bajos con topología de retroalimentaciones múltiples}
                      \label{fig:sim_retro_circuito}
                \end{figure}

                \begin{figure}[H]
                      \centering
                      \renewcommand{\figurename}{Gráfica}
                      \setcounter{figure}{12}
                      \includegraphics[width=15cm]{Imagenes/sim_sallen_key_vo.png}
                      \caption{Señal de Salida y Entrada del Diseño del Filtro pasa bajos con topología de retroalimentaciones múltiples}
                      \label{fig:sim_retro_vo}
                \end{figure}

                \begin{figure}[H]
                      \centering
                      \renewcommand{\figurename}{Gráfica}
                      \includegraphics[width=15cm]{Imagenes/sim_retro_bode.png}
                      \caption{Diagrama de Bode (Magnitud y Fase) del Diseño del Filtro pasa bajos con topología de retroalimentaciones múltiples. Frecuencia de corte señalada en negro}
                      \label{fig:sim_retro_bode}
                \end{figure}

                Acá si se obtiene una salida distinta como se ve en la gráfica \ref{fig:sim_retro_vo}, donde su salida es un poco mayor que 2, sin embargo, entra dentro de lo tolerable, también se observa una frecuencia de corte menor. Por consiguiente, se logrará realizar en el laboratorio cada una de las topologías que se diseño, para saber si se ajusta a las condiciones que se desean, reflejando los cálculos teóricos y las simulaciones en  la práctica.                
        \end{itemize}

        \item Por simulación obtenga las formas en cada salida al inyectar señales cuadradas con frecuencia tal, que su tercera armónica, coincida con la frecuencia de corte indicadas. Explique a que se deben las formas de onda obtenidas.

            Para hallar esa frecuencia que se le dará a la onda cuadrada, en especifico que se su tecera armonica, coincida con la frecuencia de corte indicada anteriormente, se debe realizar lo siguiente:

            \begin{gather*}
                T=\dfrac{1}{f_o} r\\[0.5cm]
                \text{Lo que se hará a continuación es hallar la frecuencia fundamental, asumiendo que la} \\[0.5cm]
                \text{frecuencia de corte es su tercer armónico, logrando la condición necesaria}\\[0.5cm]
                f_1=\dfrac{f_o}{3}=\dfrac{2.7KHz}{3}=900Hz\\[0.5cm]
                \text{siendo esta la frecuencia fundamental que se le colocará a la onda cuadrada}\\[0.5cm]
                T_1=\dfrac{1}{f_1}=\dfrac{1}{900}=1.111ms
            \end{gather*}

            Debido a las simulaciones anteriores, ya se sabe que cada frecuencia de corte dependiendo de su topología, y el diseño que se realizo, poseen pequeñas variaciones, sin embargo se ajustará de igual manera a 900 Hz y se observará su frecuencia de corte.

            Las ondas resultantes son consecuencia de las variaciones inducidas por la señal de entrada, que en este caso es una onda cuadrada. Al emplear esta forma de onda como entrada, se generan armónicos debido a la presencia de cambios abruptos en la señal. Además, en cada topología de filtro, se incorporan capacitores como elementos del circuito. Estos capacitores introducen más armónicos debido a sus propiedades con la corriente alterna, contribuyendo así a la complejidad de la respuesta de frecuencia del sistema.

            Es crucial destacar que la presencia de armónicos puede afectar el rendimiento del circuito. Sin embargo, la implementación de filtros activos, como los estudiados en este apartado, juega un papel crucial en mitigar los efectos negativos de estos armónicos. Al diseñar estos filtros específicos, se logra atenuar selectivamente ciertos componentes armónicos, preservando así la integridad y estabilidad del circuito frente a las variaciones no deseadas generadas por la onda cuadrada de entrada.

            \begin{itemize}
                \item \textbf{Filtro de Variables de Estado}

                    \begin{figure}[H]
                          \centering
                          \renewcommand{\figurename}{Gráfica}
                          \includegraphics[width=15cm]{Imagenes/sim_var_estado_armonico_bode.png}
                          \caption{Diagrama de Bode de Magnitud del Diseño del Filtro de variables de estados pasa bajos. Frecuencia de corte señalada en negro}
                          \label{fig:sim_var_estado_armonico_bode}
                    \end{figure}

                    \begin{figure}[H]
                      \centering
                      \renewcommand{\figurename}{Gráfica}
                      \includegraphics[width=15cm]{Imagenes/sim_var_estado_armonico_vo.png}
                      \caption{Señal de Salida y Entrada del Diseño del Filtro de variables de estados pasa bajos}
                      \label{fig:sim_var_estado_armonico_vo}
                \end{figure}

                
                
                \item \textbf{Filtro Pasa Bajos con Topología Sallen-Key}

                    \begin{figure}[H]
                          \centering
                          \renewcommand{\figurename}{Gráfica}
                          \includegraphics[width=15cm]{Imagenes/sim_sallen_key_armonico_bode.png}
                          \caption{Diagrama de Bode de Magnitud del Diseño del Filtro Pasa Bajos con Topología Sallen-Key. Frecuencia de corte señalada en negro}
                          \label{fig:sim_sallen_key_armonico_bode}
                    \end{figure}

                    \begin{figure}[H]
                      \centering
                      \renewcommand{\figurename}{Gráfica}
                      \includegraphics[width=15cm]{Imagenes/sim_sallen_key_armonico_vo.png}
                      \caption{Señal de Salida y Entrada del Diseño del Filtro Pasa Bajos con Topología Sallen-Key}
                      \label{fig:sim_sallen_key_armonico_vo}
                    \end{figure}    
                    
                \item \textbf{Filtro Pasa Bajos con topología de retroalimentaciones múltiples} 

                    \begin{figure}[H]
                          \centering
                          \renewcommand{\figurename}{Gráfica}
                          \includegraphics[width=15cm]{Imagenes/sim_retro_armonico_bode.png}
                          \caption{Diagrama de Bode de Magnitud del Diseño del Filtro Pasa Bajos con topología de retroalimentaciones múltiples. Frecuencia de corte señalada en negro}
                          \label{fig:sim_retro_armonico_bode}
                    \end{figure}

                    \begin{figure}[H]
                      \centering
                      \renewcommand{\figurename}{Gráfica}
                      \includegraphics[width=15cm]{Imagenes/sim_retro_armonico_vo.png}
                      \caption{Señal de Salida y Entrada del Diseño del Filtro Pasa Bajos con topología de retroalimentaciones múltiples}
                      \label{fig:sim_retro_armonico_vo}
                    \end{figure}
            \end{itemize}

            Al observar las gráficas \ref{fig:sim_var_estado_armonico_vo}, \ref{fig:sim_sallen_key_armonico_vo} y \ref{fig:sim_retro_armonico_vo} de la señal de salida, se aprecia una curva suave en el tiempo de retardo, resultado de la presencia del tercer armónico. Esta característica contribuye a que la salida se asemeje considerablemente a la forma de onda de entrada. No obstante, es importante destacar que este fenómeno tiene un propósito específico: transformar el circuito en un filtro eficaz.

            La estrategia es limitar la propagación de armónicos no deseados, asegurando que la salida actúe como un filtro a partir del momento en que el tercer armónico alcanza la frecuencia de corte predeterminada. Esta transición se evidencia claramente en las gráficas, donde se puede observar cómo el circuito controla selectivamente los armónicos, permitiendo que solo aquellos hasta el tercer armónico afecten la señal de salida antes de aplicar el filtrado necesario para cumplir con los requisitos del diseño.
    \end{enumerate}

\newpage
\subsection{Parte 4. Fuentes Lineales y Reguladores Monolíticos}
    \subsubsection{Diseño}
    \begin{figure}[H]
          \centering
          \setcounter{figure}{32}
          \includegraphics[width=15cm]{Imagenes/regulador_sal_fija.png}
          \caption{Regulador con Tensión de Salida Fija con Center Tap (solo tomando D1 y D2}
          \label{fig:regulador_sal_fija}
    \end{figure}

    \begin{enumerate}
        \item Para la fuente regulada de $5\volt$ fija de la figura \ref{fig:regulador_sal_fija}:
            \begin{enumerate}
                \item Explique la función de los condensadores $C_2$ y $C_3$.

                    Estos capacitores se utilizan para filtrar el ruido y las fluctuaciones en el voltaje de entrada del regulador. Ayuda a estabilizar la tensión debido a $C_1$ que permite que la inda positiva posea un voltaje de rizado, mejorar la respuesta transitoria y reducir la interferencia  electromagnética. Esto último debido a $C_2$, ya que este es el acoplador, permitiendo filtrar corrientes y/o voltajes indebidos.

                \item Explique como conectar el puente de diodos si el transformador no tiene toma central (CT).

                    Se coloca un puente de diodos sin necesidad del center tap.

                    \begin{figure}[H]
                        \centering
                        \includegraphics[width=15cm]{Imagenes/reg_sinct.png}
                        \caption{Regulador con Tensión de Salida Fija sin Center Tap}
                        \label{fig:reg_sinct}
                    \end{figure}

                    De esta manera como se observa en la figura \ref{fig:reg_sinct} se obtiene una regulación con tensión de salida fija, permitiendo hacer una rectificación de onda completa.

                \item Suponiendo una carga de 80mA determine la tensión de rizado pico-pico que se va a presentar en $C_1$.

                    Una carga de 80mA, es de $R_L=62.5 \approx 68\ohm$, siendo este último el valor comercial.
                    \begin{gather}
                        V_r= \dfrac{I_{DC}}{2fC}\label{eqn:vr}\\[0.5cm]
                        V_r=\dfrac{80mA}{2(60)(470\mu)}=1.418\volt\nonumber\\[0.5cm]
                        \text{Se halla su $V_{rpp}$}\nonumber\\[0.5cm]
                        V_{rpp}=1.418(\sqrt{2})\approx 2\volt\nonumber
                    \end{gather}

                \item Determine la tensión mínima del secundario del transformador en función de la corriente de salida, de manera que el regulador pueda mantener la regulación. Soporte sus cálculos con datos obtenidos en las hojas de datos del regulador (y marca) que usted va a usar.

                    Como se puede observar en el apartado de los anexos, capitulo \ref{sec:anexos} se tiene el datasheet del regulador 7805, para indicar cual es el valor de entrada requerido para poder obtener 5 voltios de salida. 

                    Indicando que el voltaje de entrada deberían ser los siguientes: $$7\leq V_{in} \leq 25$$

                    Haciendo uso de la ecuación \ref{eqn:vmin} se tiene:

                    \begin{gather}
                        V_{min} = V_s\sqrt{2} - \dfrac{V_{rp}}{2} - 2V_d \label{eqn:vmin}\\[0.5cm]
                        \text{Despejando $V_s$, recordar que el resultado será en RMS}\nonumber \\[0.5cm]
                        V_s = \dfrac{1}{\sqrt{2}}\left(V_{min} + \dfrac{V_{rp}}{2} + 2V_d\right)\label{eqn:vsrms}\\[0.5cm]
                        \text{Si usamos los valores mínimos y máximos de entrada en la ecuación \ref{eqn:vsrms}, se obtiene lo siguiente:}\nonumber \\[0.5cm]
                        V_{min} = 7\volt \nonumber \\[0.5cm]
                        V_s = \dfrac{1}{\sqrt{2}}\left(7 + \dfrac{2}{2} + 2(0.7)\right) = 6.6468\volt \nonumber\\[0.5cm]
                        V_{max} = 25 \volt \nonumber \\[0.5cm]
                        V_s = \dfrac{1}{\sqrt{2}}\left(25 + \dfrac{2}{2} + 2(0.7)\right) = 19.37\volt \nonumber
                    \end{gather}

                \item Determine la relación que va a obtener al colocar unas cargas de 100mA, recuerde que 
                \begin{gather}
                    reg=\dfrac{V_{occ}-V_{osc}}{V_{osc}}100\%
                    \label{eqn:regulacion}
                \end{gather}
                

                    \begin{gather*}
                        V_{osc}=5V\\[0.5cm]
                        V_{occ}=IR=100m(50)=5V \quad\therefore \quad reg=\dfrac{5-5}{5}=0\%
                    \end{gather*}

                    Indicando el resultado con una carga de 50 $\ohm$ que mantiene su voltaje de salida, siendo una regulación efectiva.
            \end{enumerate}

        \item Para la fuente regulada de la figura \ref{fig:regulador_sal_ajustable}

            \begin{figure}[H]
                \centering
                \includegraphics[width=8cm]{Imagenes/regulador_sal_ajustable.png}
                \caption{Regulador con Tensión de Salida Ajustable}
                \label{fig:regulador_sal_ajustable}
            \end{figure}

            \begin{enumerate}
                \item Determinar el rango de tensiones de salida en función del accionamiento <<x>>

                    \begin{gather}
                        I=\dfrac{5\volt}{R_1} \quad ; \quad V_2=I(xR_{v1}) \nonumber\\[0.5cm]
                        \text{Sustituyendo $I$ y $V_2$, en la siguiente ecuación}\nonumber\\[0.5cm]
                        V_o=5\volt+V_2 \Longrightarrow V_o=5\volt+I(xR_{v1})=5V+\dfrac{5\volt}{R_1}(xR_{v1})\nonumber\\[0.5cm]
                        V_o=5\left(1+\dfrac{xR_{v1}}{R_1}\right)\label{1}\\[0.5cm]
                        \text{los valores de  $0\leq x \leq 1$}   \nonumber
                    \end{gather}

                \item Asigne el valor de $R_1$ con el fin de que la fuente suministre tensiones hasta al menos 15V.

                    Tomando en cuenta que el valor de $x=1$, se tiene lo siguiente de la ecuación \ref{1}:

                    \begin{gather}
                        V_o=15\volt=5\volt\left(1+\dfrac{\SI{10}{\kilo\ohm}}{R_1}\right)\Longrightarrow \dfrac{15\volt}{5\volt}=3=1+\dfrac{\SI{10}{\kilo\ohm}}{R_1}\nonumber\\[0.5cm]
                        R_1=\dfrac{10k}{2}=\SI{5}{\kilo\ohm}
                    \end{gather}

                \item Determinar la corriente de polarización que suministra el amplificador operacional 

                    La corriente de polarización es la corriente que pasa por la resistencia $R_1$, por lo tanto,
                    
                    \begin{gather*}
                        I=\dfrac{5\volt}{R_1}= \dfrac{5\volt}{R_1}=1mA
                    \end{gather*}

                \item Determinar la tensión minima de secundario del transformador en función de la corriente de salida, de manera que el regulador puede mantener la regulación.

                    Se usa la ecuación \ref{eqn:vsrms} y sustituyendo $V_r$ por la ecuación \ref{eqn:vr}, se tiene:

                    \begin{gather}
                        V_s = \dfrac{1}{\sqrt{2}}\left(V_{min} + \dfrac{I_{DC}}{4fC} + 2V_d\right) \nonumber\\[0.5cm]
                        V_s = \dfrac{1}{\sqrt{2}}\left(15 + \dfrac{I_{DC}}{4(60)(470\mu)} + 1.4\right) \nonumber\\[0.5cm]
                        V_s = 11.6+6.268 I_{DC} \label{eqn:vsidc}               
                    \end{gather}

                    Ahora con la ecuación \ref{eqn:vsidc}, dándole valores a $I_{DC}$ se obtienen distintos valores del voltaje del secundario del transformador

                    \begin{itemize}
                        \item $I_{DC}=0$

                            $$V_s=11.6\volt$$
                        \item $I_{DC}=1mA$

                            $$V_s=11.6\volt$$
                        \item $I_{DC}=100mA$

                            $$V_s=12.226\volt$$
                    \end{itemize}
            \end{enumerate}

        \item Para la fuente de corriente ajustable de la figura \ref{fig:fuente_corriente_variable}.

            \begin{figure}[H]
                \centering
                \includegraphics[width=8cm]{Imagenes/fuente_corriente_variable.png}
                \caption{Fuente de corriente variable}
                \label{fig:fuente_corriente_variable}
            \end{figure}
            \begin{enumerate}
                \item Determinar el rango de corrientes de salida en función del accionamiento x.

                    \begin{gather}
                        I_o=\dfrac{V_o}{R_1+xR_{v1}} \label{2}
                    \end{gather}

                    Con la ecuación \ref{2}, se tienen los siguientes casos:

                    \begin{equation}
                        I_o =
                        \begin{cases}
                            20.83 \, \mathrm{mA}, & \quad \text{si } x = 0; \\[0.5cm]
                            \dfrac{5}{240+x(1k)} \, \mathrm{mA}, & \quad \text{si } 0 < x < 1; \\[0.5cm]
                            4.03 \, \mathrm{mA}, & \quad \text{si } x = 1.
                        \end{cases}
                    \end{equation}
            \end{enumerate}

        \item Para cada uno de los montajes anteriores:
            \begin{enumerate}
                \item Determine las potencias en cada elemento que vaya a usar, incluyendo las resistencias de carga.

                    \begin{itemize}
                        \item Figura \ref{fig:regulador_sal_fija}

                        
                            Se sabe que 
                            \begin{gather}
                                P=VI=\dfrac{V^2}{R} \label{p},
                            \end{gather}
                            
                             por lo tanto, se tiene:

                           \begin{gather*}
                                R_{L1} = 240 \, \Omega \quad ; \quad R_{L2} = 60 \, \Omega \\
                                P_{R_{L1}} = \frac{5^2}{240} = 104.16 \, \mathrm{mW} \quad ; \quad P_{R_{L2}} = \frac{5^2}{60} = 36.76 \, \mathrm{mW}
                            \end{gather*}

                        \item Figura \ref{fig:regulador_sal_ajustable}

                            Se usa la ecuación \ref{p}, se tiene,

                            \begin{table}[H]
                                \centering
                                \begin{tabular}{|c|c|}
                                    \hline
                                    \textbf{$P_{R_1}$}                   & \textbf{$x$} \\ \hline
                                    $(1 \, m)^2(5 \, k)=5 \, mW$         & $0$          \\ \hline
                                    $5 \, mW < P < 1.6667 \, mW$         & $0<x<1$      \\ \hline
                                    $\dfrac{5^2}{15 \, k}= 1.6667 \, mW$ & $1$          \\ \hline
                                \end{tabular}
                                \caption{Potencias de los elementos resistivos.}
                            \end{table}

                    \item Figura \ref{fig:fuente_corriente_variable}

                        Se usa la ecuación \ref{p}, se tiene,

                        \begin{table}[h]
                          \centering
                          \begin{tabular}{|c|c|}
                            \hline
                            \textbf{$P_{R_1}$} & \textbf{x} \\
                            \hline
                            $20.83 \, m^2 (240) = 104.13 \, mW$ & 0 \\
                            \hline
                            $20.83 \, m^2 (240 + 1k) = 538.02 \, mW$ & 1 \\
                            \hline
                          \end{tabular}
                          \caption{Potencia de los elementos resistivos}
                        \end{table}
                    \end{itemize}
    \subsubsection{Diseño}
    
                \item Por simulación verifique cada uno de los items anteriores (aquellos que sean susceptibles de hacerlo).
                    \begin{itemize}
                        \item  Se observara la salida de la figura \ref{fig:reg_sinct} con distintas cargas y así determinar si los cálculos realizados son los adecuados para la práctica.
                    
                   


                            \begin{figure}[H]
                                \centering
                                \renewcommand{\figurename}{Gráfica}
                                \setcounter{figure}{20}
                                \includegraphics[width=15cm]{Imagenes/sim_reg_sinct.png}
                                \caption{Voltaje de salida del regulador (V(out)), Voltaje del primario (V(vp)), Voltaje del secundario (V(vs)), Voltaje de salida del puente de diodos junto al rizado (V(v+))}
                                \label{fig:sim_reg_sinct}
                            \end{figure}
        
        
                            \begin{figure}[H]
                                \centering
                                \renewcommand{\figurename}{Gráfica}
                                \includegraphics[width=15cm]{Imagenes/sim_reg_sinct_rl.png}
                                \caption{Voltaje de salida del regulador (V(out)), Voltaje de salida del puente de diodos junto al rizado (V(v+)) con una carga de 68 ohms, dando un voltaje de rizado de 1.73 V}
                                \label{fig:sim_reg_sinct_rl}
                            \end{figure}
        
        
                            \begin{figure}[H]
                                \centering
                                \renewcommand{\figurename}{Gráfica}
                                \includegraphics[width=15cm]{Imagenes/sim_reg_sinct_rl2.png}
                                \caption{Voltaje de salida del regulador (V(out)), Voltaje de salida del puente de diodos junto al rizado (V(v+)) con una carga de 240 ohms, dando un voltaje de rizado de 0.768 V}
                                \label{fig:sim_reg_sinct_rl2}
                            \end{figure}

                        \item Simulación de la figura \ref{fig:regulador_sal_ajustable}

                            \begin{itemize}
                                \item Sin Carga
                                    
                                    \begin{figure}[H]
                                        \centering
                                        \renewcommand{\figurename}{Gráfica}
                                        \includegraphics[width=15cm]{Imagenes/sim_regulador_sal_ajustable_sinrl1.png}
                                        \caption{Voltaje de salida del regulador (V(out)), Caída de tensión del potenciómetro en RV1=10k ohm}
                                        \label{fig:sim_regulador_sal_ajustable_sinrl1}
                                    \end{figure}
        
                                    \begin{figure}[H]
                                        \centering
                                        \renewcommand{\figurename}{Gráfica}
                                        \includegraphics[width=15cm]{Imagenes/sim_regulador_sal_ajustable_sinrl2.png}
                                        \caption{Voltaje de salida del regulador (V(out)), Caída de tensión del potenciómetro en RV1=5k ohm}
                                        \label{fig:sim_regulador_sal_ajustable_sinrl2}
                                    \end{figure}
        
                                    \begin{figure}[H]
                                        \centering
                                        \renewcommand{\figurename}{Gráfica}
                                        \includegraphics[width=15cm]{Imagenes/sim_regulador_sal_ajustable_sinrl3.png}
                                        \caption{Voltaje de salida del regulador (V(out)), Caída de tensión del potenciómetro en RV1=0.1 ohm}
                                        \label{fig:sim_regulador_sal_ajustable_sinrl3}
                                    \end{figure}
                                    \begin{figure}[H]
                                        \centering
                                        \renewcommand{\figurename}{Gráfica}
                                        \includegraphics[width=15cm]{Imagenes/sim_regulador_sal_ajustable_corrientesinrl.png}
                                        \caption{Corriente de polarización que suministra el amplificador operacional por la resistencia R1}
                                        \label{fig:sim_regulador_sal_ajustable_corrientesinrl}
                                    \end{figure}
                                    
                                \item Con carga ($R_L=240\ohm$)

                                    \begin{figure}[H]
                                        \centering
                                        \renewcommand{\figurename}{Gráfica}
                                        \includegraphics[width=15cm]{Imagenes/sim_regulador_sal_ajustable_conrl1.png}
                                        \caption{Voltaje de salida del regulador (V(out)), Caída de tensión del potenciómetro en RV1=10k ohm}
                                        \label{fig:sim_regulador_sal_ajustable_conrl1}
                                    \end{figure}
        
                                    \begin{figure}[H]
                                        \centering
                                        \renewcommand{\figurename}{Gráfica}
                                        \includegraphics[width=15cm]{Imagenes/sim_regulador_sal_ajustable_conrl2.png}
                                        \caption{Voltaje de salida del regulador (V(out)), Caída de tensión del potenciómetro en RV1=5k ohm}
                                        \label{fig:sim_regulador_sal_ajustable_conrl2}
                                    \end{figure}
        
                                    \begin{figure}[H]
                                        \centering
                                        \renewcommand{\figurename}{Gráfica}
                                        \includegraphics[width=15cm]{Imagenes/sim_regulador_sal_ajustable_conrl3.png}
                                        \caption{Voltaje de salida del regulador (V(out)), Caída de tensión del potenciómetro en RV1=0.1 ohm}
                                        \label{fig:sim_regulador_sal_ajustable_conrl3}
                                    \end{figure}
                                        
                                Como se puede observar en estas simulaciones con carga y sin carga, no existe una variación en su salida, por lo tanto mantiene su regulación
                            \end{itemize}

                        \item Figura \ref{fig:fuente_corriente_variable}

                            \begin{figure}[H]
                                \centering
                                \renewcommand{\figurename}{Gráfica}
                                \includegraphics[width=15cm]{Imagenes/sim_fuente_corriente_variable_x01.png}
                                \caption{Caídas de tensión en las distintas cargas y resistencias cuando xRV1=0.1(1k)}
                                \label{fig:sim_fuente_corriente_variable_x01}
                            \end{figure}

                            \begin{figure}[H]
                                \centering
                                \renewcommand{\figurename}{Gráfica}
                                \includegraphics[width=15cm]{Imagenes/sim_fuente_corriente_variable_x1.png}
                                \caption{Caídas de tensión en las distintas cargas y resistencias cuando xRV1=1(1k)}
                                \label{fig:sim_fuente_corriente_variable_x1}
                            \end{figure}

                            \begin{figure}[H]
                                \centering
                                \renewcommand{\figurename}{Gráfica}
                                \includegraphics[width=15cm]{Imagenes/sim_fuente_corriente_variable_x05.png}
                                \caption{Caídas de tensión en las distintas cargas y resistencias cuando xRV1=0.5(1k)}
                                \label{fig:sim_fuente_corriente_variable_x05}
                            \end{figure}

                            \begin{figure}[H]
                                \centering
                                \renewcommand{\figurename}{Gráfica}
                                \includegraphics[width=15cm]{Imagenes/sim_fuente_corriente_variable_corrx1.png}
                                \caption{Corriente de salida cuando  xRV1=1(1k)}
                                \label{fig:sim_fuente_corriente_variable_corrx1}
                            \end{figure}

                            \begin{figure}[H]
                                \centering
                                \renewcommand{\figurename}{Gráfica}
                                \includegraphics[width=15cm]{Imagenes/sim_fuente_corriente_variable_corrx01.png}
                                \caption{Corriente de salida cuando  xRV1=0.1(1k)}
                                \label{fig:sim_fuente_corriente_variable_corrx01}
                            \end{figure}

                            \begin{figure}[H]
                                \centering
                                \renewcommand{\figurename}{Gráfica}
                                \includegraphics[width=15cm]{Imagenes/sim_fuente_corriente_variable_corrx05.png}
                                \caption{Corriente de salida cuando  xRV1=0.5(1k)}
                                \label{fig:sim_fuente_corriente_variable_corrx05}
                            \end{figure}

                            
                    \end{itemize}
            \end{enumerate}
        
    \end{enumerate}
    
    Como se observa, los cálculos son los adecuados para la practica, debido a la verificación realizada por cada una de las simulaciones.
\newpage



%Equipos e instrumentos

\section{Equipos e instrumentos}

\begin{table}[H]
    \centering
    \begin{tabular}{|c|c|c|}
        \hline
        \textbf{Equipo} & \textbf{Marca} & \textbf{Modelo} \\\hline
        Osciloscopio Digital & UNI-T & UTD2102CEX+ \\\hline
        Fuente de alimentación & UNI-T & UTP3305-II \\\hline
        Generador de señales & UNI-T & UTG932E \\\hline
        Multímetro Digital & BAKU & 9205A \\\hline
    \end{tabular}  
    \caption{Relación de Equipos e Instrumentos}
    \label{tab:equipos}
\end{table}


%Componentes y materiales

\section{Componentes y materiales}

\begin{table}[H]
    \centering
    \begin{tabular}{|c|c|c|}
        \hline
        \textbf{Componente} & \textbf{Valor} & \textbf{Cantidad} \\\hline
        \textbf{Protoboard} & - & 1 \\\hline
        \textbf{Puntas de osciloscopio} & - & 3 \\\hline
        \textbf{Diodos} & 1N400X ($1\leq x \leq 7$) / 1N4148 & 5 \\\hline
        \textbf{Circuito Integrado} & LM741 & 5 \\\hline
        \textbf{Resistencia variable} & $10 \, k  \si{\ohm}\pm5 \%$ & 1 \\\hline
        \multirow{9}{5cm}{\centering \textbf{Resistencia}}
        & $240 \, \si{\ohm}\pm5\%$ & 1 \\
        & $1 \, k  \si{\ohm}\pm5\%$ & 3 \\
        & $2.2 \, k  \si{\ohm}\pm5\%$ & 1 \\
        & $3.3 \, k  \si{\ohm}\pm5\%$ & 3 \\
        & $5.1 \, k  \si{\ohm}\pm5\%$ & 2 \\
        & $8.2 \, k  \si{\ohm}\pm5\%$ & 1 \\
        & $10 \, k  \si{\ohm}\pm5\%$ & 2 \\
        & $18 \, k  \si{\ohm}\pm5\%$ & 1 \\
        & $1 \, M  \si{\ohm}\pm5\%$ & 1 \\\hline
        \textbf{Condensador}& $10 \,  nF\pm20\%$ & 4 \\\hline
    \end{tabular}
    \caption{Relación de Componentes y Materiales (Bill of Materials (BOM))}
    \label{tab:componentes}
\end{table}



\newpage

%Resultados

\section{Resultados}\label{sec:resultados}

\subsection{Parte 1. Aplicaciones De Las Topologías Clásicas}\label{subsec:parte1}

    Se tomará en cuenta la incertidumbre de cada uno de los equipos e instrumentos usados en la practica, la documentación de cada equipo se encuentra en los anexos, capitulo \ref{sec:anexos}.

    Las incertidumbres fueron calculadas bajo las ecuaciones que se encuentran en la sección \ref{sec:apendice}, las ecuaciones \ref{eqn:delta_ganancia} y \ref{eqn:delta_corriente}.


    \begin{table}[H]
      \centering
      \begin{tabular}{|c|c|c|cc|c|c|}
        \hline
        \textbf{Topología}                    & \textbf{$V_{cc} [V_{DC}]$}  & \textbf{$V_{EE} [V_{DC}]$}   & \multicolumn{2}{c|}{\textbf{$V_{in}[V_p]$}}                & \textbf{$V_{out} [V_p]$}          & \textbf{$A_v [V/V]$}             \\ \hline
        \textbf{Inversor}                     & $10 \pm 1$                  & $-10 \pm 1$                  & \multicolumn{2}{c|}{$2 \pm 0.2$}                           & $3.8 \pm 0.2$                     & $1.9 \pm 0.39$                   \\ \hline
        \textbf{No inversor}                  & $10 \pm 1$                  & $-10 \pm 1$                  & \multicolumn{2}{c|}{$2 \pm 0.2$}                           & $4\pm 0.2$                        & $2 \pm 0.41$                     \\ \hline
        \multirow{2}{*}{\textbf{Restador}}    & \multirow{2}{*}{$10 \pm 1$} & \multirow{2}{*}{$-10 \pm 1$} & \multicolumn{1}{c|}{$V_1 (V^-)$}        & $V_2 (V^+)$      & \multirow{2}{*}{$1900 \pm 100 m$} & \multirow{2}{*}{$1.63 \pm 0.17$} \\ \cline{4-5}
                                              &                             &                              & \multicolumn{1}{c|}{$ 1800 \pm 100 m $} & $ 640 \pm 40 m $ &                                   &                                  \\ \hline
        \textbf{Integrador Boo} & $10 \pm 1$                  & $-10 \pm 1$                  & \multicolumn{2}{c|}{$ 2.6 \pm 0.2 $}                       & $ 4.8 \pm 0.4 $                   & -                                \\ \hline
      \end{tabular}
      \caption{Mediciones Experimentales de las Primeras Topologías}
      \label{tab:resultados1}
  \end{table}

    \begin{table}[H]
      \centering
      \begin{tabular}{|c|c|c|c|c|}
        \hline
        \textbf{Topología} & $\mathbf{V_{in} [V_{DC}]}$ & $\mathbf{V_{out} [V_{DC}]}$ & $\mathbf{R_L [k \ohm]}$ & $\mathbf{I_o [m A]}$ \\
        \hline
        \multirow{4}{5cm}{\centering \textbf{Fuente de Corriente}} & $4.8 \pm 0.4$ & $320 \pm 20 m$ & $1 \pm 5\%$ & $0.17 \pm 0.026$ \\
        & $4.8 \pm 0.4$ & $520 \pm 40m $ & $3 \pm 5\%$ & $0.32 \pm 0.015$ \\
        & $4.8 \pm 0.4$ & $600 \pm 40m $ & $4 \pm 5\%$ & $0.15 \pm 0.012 $ \\
        & $4.8 \pm 0.4$ & $ 720 \pm 40m $ & $10 \pm 5\%$ & $0.072 \pm 0.005$ \\
        \hline
      \end{tabular}
      \caption{Medición Experimental del Convertidor de Tensión a Corriente, con Distintas Cargas.}
      \label{tab:resultado12}
    \end{table}


    \begin{table}[H]
      \centering
      \begin{tabular}{|c| c| c|}
        \hline
          \textbf{Topología} & $E_{r_{A_v}} [\%]$ & $E_{r_{I}} [\%]$ \\\hline
        Inversor  & $5$ & - \\\hline
        No inversor &  $0$ & - \\\hline
        Restador  &  $18.5$ & - \\\hline
        Fuente de Corriente  &  - & $64.4$ \\\hline
      \end{tabular}
      \caption{Error Porcentual de las Mediciones Experimentales con respecto a las teóricas de las Primeras Topologías}
      \label{tab:desviacion_resultados1}
    \end{table}
    \subsubsection{Inversor}
        \begin{figure}[H]
            \centering
            \renewcommand{\figurename}{Imagen}
            \setcounter{figure}{4}
            \includegraphics[width=15cm]{Imagenes/exp_inversor.png}
            \caption{Señal de Entrada (Azul) y Salida (Amarilla) del Inversor}
            \label{fig:exp_inversor}
        \end{figure}
    
        \begin{table}[H]
            \centering
            \begin{tabular}{|c|c|c|c|}
                \hline
                \textbf{time/div} $[s]$ & \textbf{Channel} & \textbf{voltios/div $[\volt]$} & \textbf{Acoplamiento} \\ \hline
                $200 \, \mu$ & 1 (Azul) &   $1 $ & AC \\ \hline
                $200 \, \mu$ & 2 (Amarillo)  &   $1 $ & AC \\ \hline  
            \end{tabular}
            \caption{Escalas Usada en el Osciloscopio Digital UNI-T UTD2102CEX+}
            \label{tab:escala_inversor}
        \end{table}

    \subsubsection{No inversor}

        \begin{figure}[H]
            \centering
            \renewcommand{\figurename}{Imagen}
            \includegraphics[width=15cm]{Imagenes/exp_noinversor.png}
            \caption{Señal de Entrada (Azul) y Salida (Amarilla) del No Inversor}
            \label{fig:exp_noinversor}
        \end{figure}
    
        \begin{table}[H]
            \centering
            \begin{tabular}{|c|c|c|c|}
                \hline
                \textbf{time/div} $[s]$ & \textbf{Channel} & \textbf{voltios/div $[\volt]$} & \textbf{Acoplamiento} \\ \hline
                $200 \, \mu$ & 1 (Azul) &   $1 $ & AC \\ \hline
                $200 \, \mu$ & 2 (Amarillo)  &   $1 $ & AC \\ \hline  
            \end{tabular}
            \caption{Escalas Usada en el Osciloscopio Digital UNI-T UTD2102CEX+}
            \label{tab:escala_noinversor}
        \end{table}

    \subsubsection{Restador}
        
        
        \begin{figure}[H]
            \centering
            \renewcommand{\figurename}{Imagen}
            \includegraphics[width=15cm]{Imagenes/exp_restador.png}
            \caption{Señal de Entrada (Azul) y Salida (Amarillo) del Restador}
            \label{fig:exp_restador}
        \end{figure}
    
        \begin{table}[H]
            \centering
            \begin{tabular}{|c|c|c|c|}
                \hline
                \textbf{time/div} $[s]$ & \textbf{Channel} & \textbf{voltios/div $[\volt]$} & \textbf{Acoplamiento} \\ \hline
                $200 \, \mu$ & 1 (Azul) &  $500 \, m $ & AC \\ \hline
                $200 \, \mu$ & 2 (Amarillo)  &   $500 \, m $ & AC \\ \hline  
            \end{tabular}
            \caption{Escalas Usada en el Osciloscopio Digital UNI-T UTD2102CEX+}
            \label{tab:escala_restador}
        \end{table}

    \subsubsection{Integrador No Inversor (Integrador de Boo)}

        \begin{figure}[H]
            \centering
            \renewcommand{\figurename}{Imagen}
            \includegraphics[width=15cm]{Imagenes/exp_integrador_boo.png}
            \caption{Señal de Entrada (Azul) y Salida (Amarilla) del Integrador Boo}
            \label{fig:exp_integrador_boo}
        \end{figure}
    
        \begin{table}[H]
            \centering
            \begin{tabular}{|c|c|c|c|}
                \hline
                \textbf{time/div} $[s]$ & \textbf{Channel} & \textbf{voltios/div $[\volt]$} & \textbf{Acoplamiento} \\ \hline
                $500 \, \mu$ & 1 (Azul) &  $1 $ & AC \\ \hline
                $500 \, \mu$ & 2 (Amarillo)  &   $2 $ & AC \\ \hline  
            \end{tabular}
            \caption{Escalas Usada en el Osciloscopio Digital UNI-T UTD2102CEX+}
            \label{tab:escala_exp_integrador_boo}
        \end{table}

        
\subsection{Parte 2. Amplificador Operacional Real}\label{subsec:parte2}

    En este apartado nos enfocaremos en los resultados experimentales, donde se calcularan las incertidumbres bajo las ecuaciones \ref{eqn:delta_tension_offset}, \ref{eqn:delta_corriente_offset}, \ref{eqn:delta_corriente_polarizacion}\ref{eqn:delta_corriente}, \ref{eqn:delta_gbwp} y \ref{eqn:delta_slewrate}, que se hallan en la sección \ref{sec:apendice} y algunas mediciones indirectas calculadas en el pre-laboratorio, que se hallan en la sección \ref{sec:metodologia}, siendo las ecuaciones \ref{eqn:vos}, \ref{eqn:ib1}, \ref{eqn:ib2}, \ref{eqn:ios} y \ref{eqn:sr}.

    Recordando que JP1 es la entrada inversora, por consiguiente, JP2 es la no inversora.

    Se usaron algunas resistencias de distintos valores, debido a no encontrar las pedidas, con una diferencia del 8.3\% a 10\% con una tolerancia de igual manera de 5\% en la siguiente tabla se indican.

    \begin{table}[H]
        \centering
        \begin{tabular}{|c|c|}
        \hline
           \textbf{Resistencias indicadas} $[\ohm]$  & \textbf{Resistencias usadas} $[\ohm]$\\\hline
            $R_3=R_4=22 \,M  $ & $R_3=R_4=24 \,M \pm 5\%$ \\\hline
            $R_8=910$ & $R_8=820 \pm 5\%$ \\\hline
        \end{tabular}
        \caption{Resistencias Reemplazadas}
        \label{tab:resistencias_reemplazadas}
    \end{table}
    
    \subsubsection{Tensión Offset}

        \begin{table}[H]
          \centering
          \begin{tabular}{|c|c|c|c|c|}
            \hline
            \textbf{Configuración} & $\mathbf{V_{CC} [V_{DC}]}$ & $\mathbf{V_{EE} [V_{DC}]}$ & $\mathbf{V_{o} [V_{DC}]}$ & $\mathbf{V_{os} [mV_{DC}]}$ \\
            \hline
            JP1 y JP2 (corto) & $10 \pm 0.1$ & $-10 \pm 0.1$ & $-2 \pm 0.2$ & $-1.998 \pm 0.245$ \\
            \hline
          \end{tabular}
          \caption{Medición experimental de la Tensión offset de la figura \ref{fig:amp_op_real}}
          \label{tab:tension_offset}
        \end{table}

    \subsubsection{Corriente de Polarización}

        \begin{enumerate}
            \item \textbf{Corriente de polarización 1}

                \begin{table}[H]
                  \centering
                  \begin{tabular}{|c|c|c|c|c|}
                    \hline
                    \textbf{Configuración} & $\mathbf{V_{CC} [V_{DC}]}$ & $\mathbf{V_{EE} [V_{DC}]}$ & $\mathbf{V_{o} [V_{DC}]}$ & $\mathbf{I_{B_1} [pA]}$ \\
                    \hline
                    JP1 (abierto) y JP2 (corto) & $10 \pm 1$ & $-10 \pm 1$ & $-2.6 \pm 0.2$ & $27.245 \pm 16.67$ \\
                    \hline
                  \end{tabular}
                  \caption{Medición experimental de la corriente de polarización de la entrada inversora de la figura \ref{fig:amp_op_real}}
                  \label{tab:corriente_polarizacion1}
                \end{table}

            \item \textbf{Corriente de polarización 2}

                \begin{table}[H]
                  \centering
                  \begin{tabular}{|c|c|c|c|c|}
                    \hline
                    \textbf{Configuración} & $\mathbf{V_{CC} [V_{DC}]}$ & $\mathbf{V_{EE} [V_{DC}]}$ & $\mathbf{V_{o} [V_{DC}]}$ & $\mathbf{I_{B_2} [pA]}$ \\
                    \hline
                    JP1 (corto) y JP2 (abierto) & $10 \pm 0.1$ & $-10 \pm 1$ & $10 \pm 1$ & $544.909 \pm 20.81$ \\
                    \hline
                  \end{tabular}
                  \caption{Medición experimental de la corriente de polarización de la entrada no inversora de la figura \ref{fig:amp_op_real}}
                  \label{tab:corriente_polarizacion2}
                \end{table}
        \end{enumerate}

    \subsubsection{Corriente Offset}

        \begin{table}[H]
          \centering
          \begin{tabular}{|c|}
            \hline
            $\mathbf{I_{os} [pA]}$ \\
            \hline
            $517.664 \pm 26.66$ \\
            \hline
          \end{tabular}
          \caption{Medición indirecta de la corriente de offset, mediante los valores de las tablas \ref{tab:corriente_polarizacion1} y \ref{tab:corriente_polarizacion2}.}
          \label{tab:corriente_offset}
        \end{table} 

    \subsubsection{Producto del Ancho de Banda por la Ganancia (GBWP)}   

        El voltaje de Salida es el voltaje medido justo en su frecuencia de corte, debido que se desea observar su GBWP en las distintas configuraciones.

        \begin{table}[H]
          \centering
          \begin{tabular}{|c|c|c|c|c|}
            \hline
            \textbf{Configuración} & $\mathbf{V_{CC} [V_p]}$ & $\mathbf{V_{EE} [V_p]}$ & $\mathbf{V_{in} [V_p]}$ & $\mathbf{V_{out} [V_p]}$ \\
            \hline
            JP3 y JP4 (abiertos) & $10 \pm 0.1$ & $-10 \pm 0.1$ & $1 \pm 0.1$ & $6 \pm 1$  \\
            \hline
            JP3 (abierto); JP4(corto) & $10 \pm 0.1$ & $-10 \pm 0.1$ & $1 \pm 0.1$  & $6 \, \pm 1$ \\
            \hline
            JP3 y JP4 (corto) Buffer & $10 \pm 0.1$ & $-10 \pm 0.1$ & $1 \pm 0.1$  & $1 \pm 0.1 $ \\
            \hline
          \end{tabular}
        \end{table}
        \begin{table}[H]
          \centering
          \begin{tabular}{|c|c|c|}
            \hline
            \textbf{Configuración} & \textbf{G} $\mathbf{[V/V]}$ & \textbf{F} $\mathbf{[Hz]}$ \\
            \hline
            JP3 y JP4 (abiertos) & $6 \pm 1.166$ & $25 \pm 6.25 \, k $ \\
            \hline
            JP3 (abierto); JP4(corto) & $6 \pm 1.116$ & $25 \pm 6.25 \, k$ \\
            \hline
            JP3 y JP4 (corto) Buffer & $1 \pm 0.141$ & Un valor muy grande se distorsiona \\
            \hline
          \end{tabular}
          \caption{Mediciones experimentales del GBWP de la figura \ref{fig:GBWP}}
          \label{tab:gbwp}
        \end{table}

         \begin{table}[H]
          \centering
          \begin{tabular}{|c|c|}
            \hline
            \textbf{Configuración} & \textbf{GBWP} $[(V/V)Hz$ \\
            \hline
            JP3 y JP4 (abiertos) & $150\pm 15.31 \, k$ \\
            \hline
            JP3 (abierto); JP4(corto) & $150 \pm  15.31 \, k$ \\
            \hline
            JP3 y JP4 (corto) Buffer & Muy grande \\
            \hline
          \end{tabular}
          \caption{Calculo del GBWP de la figura \ref{fig:GBWP}}
          \label{tab:calculo_gbwp}
        \end{table}
        Reflejando la ganancia y la frecuencia en un plano cartesiano de Ganancia vs Frecuencia, se puede visualizar mejor los puntos de las mediciones experimentales en la gráfica \ref{fig:puntos_gbwp}.

        \begin{figure}[H]
            \centering
            \renewcommand{\figurename}{Gráfica}
            \setcounter{figure}{36}
            \includegraphics[width=15cm]{Imagenes/puntos_gbwp.png}
            \caption{Puntos reflejados de la Ganancia Vs Frecuencia, cumpliendo el Producto  del Ancho de Banda por la Ganancia (GBWP)}
            \label{fig:puntos_gbwp}
        \end{figure}

    \subsubsection{Slew Rate(SR) o Tasa de Variación}

        Acá haremos uso de la mediciones indirectas de la ecuación \ref{eqn:sr}, bajo las mediciones realizadas con el osciloscopio digital.
        
        \begin{figure}[H]
            \centering
            \renewcommand{\figurename}{Imagen}
            \setcounter{figure}{11}
            \includegraphics[width=15cm]{Imagenes/exp_sr.png}
            \caption{Señal de Salida (Azul) y Entrada (Amarilla) del Buffer, donde se puede observar la tasa de variación en su señal de salida de la figura \ref{fig:buffer}}
            \label{fig:exp_sr}
        \end{figure}

        \begin{table}[H]
            \centering
            \begin{tabular}{|c|c|c|c|}
                \hline
                \textbf{time/div} $[s]$ & \textbf{Channel} & \textbf{voltios/div $[\volt]$} & \textbf{Acoplamiento} \\ \hline
                $1 \, \mu$ & 1 (Azul) &  $500 \, m $ & AC \\ \hline
                $1 \, \mu$ & 2 (Amarillo)  &   $500 \, m $ & AC \\ \hline  
            \end{tabular}
            \caption{Escalas Usada en el Osciloscopio Digital UNI-T UTD2102CEX+}
            \label{tab:escala_exp_sr}
        \end{table}

        \begin{table}[H]
          \centering
          \begin{tabular}{|c|c|c|c|c|c|c|c|c|}
            \hline
            $\mathbf{V_{CC} [V_p]}$ & $\mathbf{V_{EE} [V_p]}$ & $\mathbf{V_{in} [mV_p]}$ & $\mathbf{t_1 [\mu s]}$ & $\mathbf{t_2 [\mu s]}$ & $\mathbf{V_1 [mV]}$ & $\mathbf{V_2 [mV]}$ & $\mathbf{SR [V/\mu s]}$ \\
            \hline
            $10 \pm 0.1$ & $-10 \pm 0.1$ & $1000 \pm 100$ & $0.3 \pm 0.2$ & $2.7 \pm 0.2$ & $-500 \pm 100$ & $700 \pm 100$ & $0.5 \pm 0.05$ \\
            \hline
          \end{tabular}
          \caption{Mediciones del Slew Rate de la figura \ref{fig:buffer}}
          \label{tab:exp_sr}
        \end{table}

    \subsubsection{Corriente de Cortocircuito}

        \begin{table}[H]
          \centering
          \begin{tabular}{|c|c|c|c|c|c|}
            \hline
            $\mathbf{V_{CC} [V_p]}$ & $\mathbf{V_{EE} [V_p]}$ & $\mathbf{V_{in} [V_p]}$ & $\mathbf{V_o [V_p]}$ & $\mathbf{R_v [\ohm]}$ & $\mathbf{I_{CC} [mA]}$ \\
            \hline
            $10 \pm 0.1$ & $-10 \pm 0.1$ & $5 \pm 1$ & $0.1 \pm 0.02$ & $3.9 \pm 5\%$ & $25.64 \pm 5.29$ \\
            \hline
          \end{tabular}
          \caption{Mediciones Experimentales de la Corriente de Cortocircuito de la Figura \ref{fig:bufferv}}
          \label{tab:corriente_cc}
        \end{table}

    \subsubsection{Límites Máximos de Excursión}    

        \begin{figure}[H]
            \centering
            \renewcommand{\figurename}{Imagen}
            \includegraphics[width=15cm]{Imagenes/exp_buffer_limites.png}
            \caption{Señal de Salida (Azul) y Entrada (Amarilla) del Buffer, donde se puede observar los limites máximos de excursión en su señal de salida de la figura \ref{fig:buffer}}
            \label{fig:exp_buffer_limites}
        \end{figure}

        \begin{table}[H]
            \centering
            \begin{tabular}{|c|c|c|c|}
                \hline
                \textbf{time/div} $[s]$ & \textbf{Channel} & \textbf{voltios/div $[\volt]$} & \textbf{Acoplamiento} \\ \hline
                $200 \, \mu$ & 1 (Azul) &  $5 $ & AC \\ \hline
                $200 \, \mu$ & 2 (Amarillo)  &   $5 $ & AC \\ \hline  
            \end{tabular}
            \caption{Escalas Usada en el Osciloscopio Digital UNI-T UTD2102CEX+}
            \label{tab:escala_exp_buffer_limites}
        \end{table}

        \begin{table}[H]
          \centering
          \begin{tabular}{|c|c|}
            \hline
            \textbf{Limites máximos} & $\mathbf{V_o [V_p]}$ \\
            \hline
            Superior & $9 \pm 1$ \\
            \hline
            Inferior & $7 \pm 1$ \\
            \hline
          \end{tabular}
          \caption{Mediciones Experimentales de los Límites Máximos de Excursión de la Figura \ref{fig:buffer}}
          \label{tab:exp_buffer_limites}
        \end{table}

    
\subsection{Parte 3. Filtros Activos}\label{subsec:parte3}

      En este apartado, se calcularan las incertidumbres de ganancia y frecuencia, siendo las ecuaciones \ref{eqn:delta_frecuencia} y \ref{eqn:delta_ganancia}, que se hallan en la sección \ref{sec:apendice}. Las figuras que se le realizaron las mediciones experimentales fueron la \ref{fig:var_estado}, \ref{fig:sallen_key} y \ref{fig:retro_multiples}. 
      
      A continuación se procede a indicar los resultados.

      \subsubsection{Filtro de Variables de Estado}

        \begin{itemize}
          \item \textbf{Pasa Bajos}
        
            En este apartado, se diseño el circuito que se pedía solo para la salida del pasa bajo, sin embargo, fueron simuladas las otras salidas y concuerdan con los resultados obtenidos.

            \begin{table}[H]
              \centering
              \begin{tabular}{|c|c|c|c|c|c|c|}
                \hline
                $\mathbf{V_{CC} [V_p]}$ & $\mathbf{V_{EE} [V_p]}$ & $\mathbf{V_{in} [V_p]}$ & $\mathbf{V_{out} [V_p]}$ & \textbf{Ganancia} $\mathbf{[V/V]}$ & \textbf{Frecuencia} $\mathbf{[Hz]}$ \\
                \hline
                $10 \pm 0.1$ & $-10 \pm 0.1$ & $1 \pm 0.1$ & $1.9 \pm 0.1$ & $1.9 \pm 0.21$ & $1 \pm 0.04 k$ \\
                \hline
                $10 \pm 0.1$ & $-10 \pm 0.1$ & $1 \pm 0.1$ & $1.8 \pm 0.1$ & $1.8 \pm 0.21$ & $1.5 \pm 0.045 k$ \\
                \hline
                $10 \pm 0.1$ & $-10 \pm 0.1$ & $1 \pm 0.1$ & $1.7 \pm 0.1 $ & $1.7 \pm 0.20$ & $2 \pm 0.08 k$ \\
                \hline
                $10 \pm 0.1$ & $-10 \pm 0.1$ & $1 \pm 0.1$ & $1.2 \pm 0.1$ & $1.2 \pm 0.16$ & $3.2 \pm 0.205 k$ \\
                \hline
                $10 \pm 0.1$ & $-10 \pm 0.1$ & $1 \pm 0.1$ & $1 \pm 0.1$ & $1 \pm 0.14$ & $3.5 \pm 0.123 k$ \\
                \hline
                $10 \pm 0.1$ & $-10 \pm 0.1$ & $1 \pm 0.1$ & $0.5 \pm 0.1$ & $0.5 \pm 0.11$ & $5.5 \pm 0.303 k$ \\
                \hline
              \end{tabular}
              \caption{Medidas y Cálculos Experimentales de la Ganancia y Frecuencia de la Figura \ref{fig:var_estado}}
              \label{tab:exp_var_estado}
            \end{table}

            \begin{table}[H]
              \centering
              \begin{tabular}{|c|c|c|c|}
                \hline
                $\mathbf{f_{c_{exp}} [Hz]}$ & $\mathbf{f_{c_{teo}} [Hz]}$ & \textbf{Desviación de frecuencia [\%]} \\
                \hline
                $3.5 \pm 0.123  \, k$ & $2.74 \pm 0.55 \, k$ & $21.71$ \\
                \hline
              \end{tabular}
              \caption{Desviación Estándar de la Frecuencia de Corte de la Figura \ref{fig:var_estado}}
              \label{tab:exp_var_estado_frecorte}
            \end{table}

            Haciendo uso de la tabla \ref{tab:exp_var_estado}, tomando los valores que se hallan en las columnas de Ganancia y frecuencia, se puede obtener el Diagrama  Asintótico de Bode en magnitud (respuesta en frecuencia) como se tiene en la gráfica \ref{fig:resp_frec_var_estado} a continuación.

             \begin{figure}[H]
                \centering
                \renewcommand{\figurename}{Gráfica}
                \setcounter{figure}{37}
                \includegraphics[width=15cm]{Imagenes/resp_frec_var_estado.png}
                \caption{Diagrama  Asintótico de Bode en magnitud (respuesta en frecuencia pasa bajo) de las mediciones experimentales de la tabla \ref{tab:exp_var_estado}}
                \label{fig:resp_frec_var_estado}
            \end{figure}

            El gráfico \ref{fig:resp_frec_var_estado}, fue generado por el lenguaje de programación GNU Octave, el código se encuentra en la sección \ref{sec:anexos}, en la división \ref{subsec:cod_octave}, denominado \textbf{Diagrama asintótico de Bode}, además todas las respuestas en frecuencia se hallarán a través del mismo script.
            \begin{itemize}
                \item \textbf{Armónicos}

                     \begin{figure}[H]
                        \centering
                        \renewcommand{\figurename}{Imagen}
                        \setcounter{figure}{13}
                        \includegraphics[width=15cm]{Imagenes/armonico_var_estado.png}
                        \caption{Señal de Entrada (Azul) y Salida (Amarilla), donde se puede observar los cambios en la señal de salida con su tercer armónico coincidiendo con la frecuencia de corte teórico de la tabla \ref{tab:exp_var_estado_frecorte}}
                        \label{fig:armonico_var_estado}
                    \end{figure}
            
                    \begin{table}[H]
                        \centering
                        \begin{tabular}{|c|c|c|c|}
                            \hline
                            \textbf{time/div} $[s]$ & \textbf{Channel} & \textbf{voltios/div $[\volt]$} & \textbf{Acoplamiento} \\ \hline
                            $200 \, \mu$ & 1 (Azul) &  $500 \, m $ & AC \\ \hline
                            $200 \, \mu$ & 2 (Amarillo)  &   $500 \, m $ & AC \\ \hline  
                        \end{tabular}
                        \caption{Escalas Usada en el Osciloscopio Digital UNI-T UTD2102CEX+}
                        \label{tab:escala_exp_var_estado_armonico}
                    \end{table}
            \end{itemize}

            \item \textbf{Pasa Banda}
            
            \begin{table}[H]
              \centering
              \begin{tabular}{|c|c|c|c|c|c|c|}
                \hline
                $\mathbf{V_{CC} [V_p]}$ & $\mathbf{V_{EE} [V_p]}$ & $\mathbf{V_{in} [V_p]}$ & $\mathbf{V_{out} [mV_p]}$ & \textbf{Ganancia} $\mathbf{[V/V]}$ & \textbf{Frecuencia} $\mathbf{[Hz]}$ \\
                \hline
                $10 \pm 0.1$ & $-10 \pm 0.1$ & $1 \pm 0.1$ & $200 \pm 20$ & $0.2 \pm 0.028$ & $700 \pm 49 $ \\
                \hline
                $10 \pm 0.1$ & $-10 \pm 0.1$ & $1 \pm 0.1$ & $360 \pm 20$ & $0.32 \pm 0.041$ & $1.3 \pm 0.068 k$ \\
                \hline
                $10 \pm 0.1$ & $-10 \pm 0.1$ & $1 \pm 0.1$ & $520 \pm 40 $ & $0.52 \pm 0.066$ & $3.5 \pm 0.25 k$ \\
                \hline
                $10 \pm 0.1$ & $-10 \pm 0.1$ & $1 \pm 0.1$ & $360 \pm 20$ & $0.36 \pm 0.041$ & $6.2 \pm 0.38 k$ \\
                \hline
                $10 \pm 0.1$ & $-10 \pm 0.1$ & $1 \pm 0.1$ & $200 \pm 20$ & $0.2 \pm 0.028$ & $12.4 \pm 0.62 k$ \\
                \hline
              \end{tabular}
              \caption{Medidas y Cálculos Experimentales de la Ganancia y Frecuencia de la Figura \ref{fig:var_estado}}
              \label{tab:exp_var_estado_pb}
            \end{table}

            \begin{table}[H]
              \centering
              \begin{tabular}{|c|c|}
                \hline
                $\mathbf{f_{c_{exp_L}} [Hz]}$ & $\mathbf{f_{c_{exp_H}} [Hz]}$  \\
                \hline
                $1.3 \pm 0.068 k$ & $6.2 \pm 0.38 k$ \\
                \hline
              \end{tabular}
              \caption{Desviación Estándar de la Frecuencia de Corte de la Figura \ref{fig:var_estado}}
              \label{tab:exp_var_estado_frecorte_pb}
            \end{table}

            Haciendo uso de la tabla \ref{tab:exp_var_estado_pb}, tomando los valores que se hallan en las columnas de Ganancia y frecuencia, se puede obtener el Diagrama  Asintótico de Bode en magnitud (respuesta en frecuencia) como se tiene en la gráfica \ref{fig:resp_frec_var_estado_pb} a continuación.

             \begin{figure}[H]
                \centering
                \renewcommand{\figurename}{Gráfica}
                \setcounter{figure}{38}
                \includegraphics[width=15cm]{Imagenes/resp_frec_var_estado_pb.png}
                \caption{Diagrama  Asintótico de Bode en magnitud (respuesta en frecuencia pasa banda) de las mediciones experimentales de la tabla \ref{tab:exp_var_estado_pb}}
                \label{fig:resp_frec_var_estado_pb}
            \end{figure}

            \item \textbf{Pasa Altos}
            
            
            \begin{table}[H]
              \centering
              \begin{tabular}{|c|c|c|c|c|c|c|}
                \hline
                $\mathbf{V_{CC} [V_p]}$ & $\mathbf{V_{EE} [V_p]}$ & $\mathbf{V_{in} [V_p]}$ & $\mathbf{V_{out} [mV_p]}$ & \textbf{Ganancia} $\mathbf{[V/V]}$ & \textbf{Frecuencia} $\mathbf{[Hz]}$ \\
                \hline
                $10 \pm 0.1$ & $-10 \pm 0.1$ & $1 \pm 0.1$ & $180 \pm 10$ & $0.18 \pm 0.020$ & $2 \pm 0.08 k$ \\
                \hline
                $10 \pm 0.1$ & $-10 \pm 0.1$ & $1 \pm 0.1$ & $300 \pm 20$ & $0.3 \pm 0.036$ & $3 \pm 0.18 k$ \\
                \hline
                $10 \pm 0.1$ & $-10 \pm 0.1$ & $1 \pm 0.1$ & $400 \pm 20 $ & $0.40 \pm 0.045$ & $5 \pm 0.25 k$ \\
                \hline
                $10 \pm 0.1$ & $-10 \pm 0.1$ & $1 \pm 0.1$ & $440 \pm 40$ & $0.44 \pm 0.060$ & $10 \pm 0.40 k$ \\
                \hline
              \end{tabular}
              \caption{Medidas y Cálculos Experimentales de la Ganancia y Frecuencia de la Figura \ref{fig:var_estado}}
              \label{tab:exp_var_estado_pa}
            \end{table}

            
            \begin{table}[H]
              \centering
              \begin{tabular}{|c|}
                \hline
                $\mathbf{f_{c_{exp} [Hz]}}$ \\
                \hline
                $3 \pm 0.068 k$  \\
                \hline
              \end{tabular}
              \caption{Desviación Estándar de la Frecuencia de Corte de la Figura \ref{fig:var_estado}}
              \label{tab:exp_var_estado_frecorte_pa}
            \end{table}

            Haciendo uso de la tabla \ref{tab:exp_var_estado_pa}, tomando los valores que se hallan en las columnas de Ganancia y frecuencia, se puede obtener el Diagrama  Asintótico de Bode en magnitud (respuesta en frecuencia) como se tiene en la gráfica \ref{fig:resp_frec_var_estado_pa} a continuación.

             \begin{figure}[H]
                \centering
                \renewcommand{\figurename}{Gráfica}
                \setcounter{figure}{39}
                \includegraphics[width=15cm]{Imagenes/resp_frec_var_estado_pa.png}
                \caption{Diagrama  Asintótico de Bode en magnitud (respuesta en frecuencia pasa altos) de las mediciones experimentales de la tabla \ref{tab:exp_var_estado_pa}}
                \label{fig:resp_frec_var_estado_pa}
            \end{figure}

          \end{itemize}

        \subsubsection{Filtro Pasa Bajos con Topología Sallen-Key}

            \begin{table}[H]
              \centering
              \begin{tabular}{|c|c|c|c|c|c|c|}
                \hline
                $\mathbf{V_{CC} [V_p]}$ & $\mathbf{V_{EE} [V_p]}$ & $\mathbf{V_{in} [V_p]}$ & $\mathbf{V_{out} [V_p]}$ & \textbf{Ganancia} $\mathbf{[V/V]}$ & \textbf{Frecuencia} $\mathbf{[Hz]}$ \\
                \hline
                $10 \pm 0.1$ & $-10 \pm 0.1$ & $1 \pm 0.1$ & $2 \pm 0.1$ & $2 \pm 0.22$ & $100 \pm 4 $ \\
                \hline
                $10 \pm 0.1$ & $-10 \pm 0.1$ & $1 \pm 0.1$ & $2.2 \pm 0.2$ & $2.2 \pm 0.29$ & $1 \pm 0.40 k$ \\
                \hline
                $10 \pm 0.1$ & $-10 \pm 0.1$ & $1 \pm 0.1$ & $1.5 \pm 0.1 $ & $1.5 \pm 0.18$ & $5.4 \pm 0.29 k$ \\
                \hline
                $10 \pm 0.1$ & $-10 \pm 0.1$ & $1 \pm 0.1$ & $1 \pm 0.1$ & $1 \pm 0.14$ & $8.1 \pm 0.66 k$ \\
                \hline
                $10 \pm 0.1$ & $-10 \pm 0.1$ & $1 \pm 0.1$ & $ 0.48\pm 0.04$ & $0.48 \pm 0.062$ & $17 \pm 0.58 k$ \\
                \hline
              \end{tabular}
              \caption{Medidas y Cálculos Experimentales de la Ganancia y Frecuencia de la Figura \ref{fig:sallen_key}}
              \label{tab:exp_sallen_key}
            \end{table}

            \begin{table}[H]
              \centering
              \begin{tabular}{|c|c|c|c|}
                \hline
                $\mathbf{f_{c_{exp}} [Hz]}$ & $\mathbf{f_{c_{teo}} [Hz]}$ & \textbf{Desviación de frecuencia [\%]} \\
                \hline
                $5.4 \pm 0.29 \, k$ & $2.74 \pm 0.55 \, k$ & $97.08$ \\
                \hline
              \end{tabular}
              \caption{Desviación Estándar de la Frecuencia de Corte de la Figura \ref{fig:sallen_key}}
              \label{tab:exp_sallen_key_frecorte}
            \end{table}
    
            Haciendo uso de la tabla \ref{tab:exp_sallen_key}, tomando los valores que se hallan en las columnas de Ganancia y frecuencia, se puede obtener el Diagrama  Asintótico de Bode en magnitud (respuesta en frecuencia) como se tiene en la gráfica \ref{fig:resp_frec_sallen_key} a continuación.

            \begin{figure}[H]
                \centering
                \renewcommand{\figurename}{Gráfica}
                \setcounter{figure}{40}
                \includegraphics[width=15cm]{Imagenes/resp_frec_sallen_key.png}
                \caption{Diagrama  Asintótico de Bode en magnitud (respuesta en frecuencia) de las mediciones experimentales de la tabla \ref{tab:exp_sallen_key}}
                \label{fig:resp_frec_sallen_key}
            \end{figure}

             \begin{itemize}
                \item \textbf{Armónicos}

                     \begin{figure}[H]
                        \centering
                        \renewcommand{\figurename}{Imagen}
                        \setcounter{figure}{14}
                        \includegraphics[width=15cm]{Imagenes/armonico_sallen_key.png}
                        \caption{Señal de Entrada (Azul) y Salida (Amarilla), donde se puede observar los cambios en la señal de salida con su tercer armónico coincidiendo con la frecuencia de corte teórico de la tabla \ref{tab:exp_sallen_key_frecorte}}
                        \label{fig:armonico_sallen_key}
                    \end{figure}
            
                    \begin{table}[H]
                        \centering
                        \begin{tabular}{|c|c|c|c|}
                            \hline
                            \textbf{time/div} $[s]$ & \textbf{Channel} & \textbf{voltios/div $[\volt]$} & \textbf{Acoplamiento} \\ \hline
                            $100 \, \mu$ & 1 (Azul) &  $500 \, m $ & AC \\ \hline
                            $100 \, \mu$ & 2 (Amarillo)  &   $500 \, m $ & AC \\ \hline  
                        \end{tabular}
                        \caption{Escalas Usada en el Osciloscopio Digital UNI-T UTD2102CEX+}
                        \label{tab:escala_exp_sallen_key_armonico}
                    \end{table}
            \end{itemize}

        \subsubsection{Filtro Pasa Bajos con Topología de Retroalimentaciones Múltiples}    

            \begin{table}[H]
              \centering
              \begin{tabular}{|c|c|c|c|c|c|}
                \hline
                $\mathbf{V_{CC} [V_p]}$ & $\mathbf{V_{EE} [V_p]}$ & $\mathbf{V_{in} [V_p]}$ & $\mathbf{V_{out} [V_p]}$ & \textbf{Ganancia} $\mathbf{[V/V]}$ & \textbf{Frecuencia} $\mathbf{[Hz]}$ \\
                \hline
                $10 \pm 0.1$ & $-10 \pm 0.1$ & $1 \pm 0.1$ & $1.9 \pm 0.1$ & $1.9 \pm 0.21$ & $100 \pm 40 $ \\
                \hline
                $10 \pm 0.1$ & $-10 \pm 0.1$ & $1 \pm 0.1$ & $1.9 \pm 0.2$ & $1.9 \pm 0.28$ & $1 \pm 0.40 k$ \\
                \hline
                $10 \pm 0.1$ & $-10 \pm 0.1$ & $1 \pm 0.1$ & $1.4 \pm 0.1 $ & $1.4 \pm 0.17$ & $2.8 \pm 0.16 k$ \\
                \hline
                $10 \pm 0.1$ & $-10 \pm 0.1$ & $1 \pm 0.1$ & $0.8 \pm 0.1$ & $0.8 \pm 0.13$ & $4.1 \pm 0.17 k$ \\
                \hline
                $10 \pm 0.1$ & $-10 \pm 0.1$ & $1 \pm 0.1$ & $ 0.6 \pm 0.04$ & $0.6 \pm 0.072$ & $4.9 \pm 0.24 k$ \\
                \hline
              \end{tabular}
              \caption{Medidas y Cálculos Experimentales de la Ganancia y Frecuencia de la Figura \ref{fig:retro_multiples}}
              \label{tab:exp_retro_multiples}
            \end{table}

             \begin{table}[H]
              \centering
              \begin{tabular}{|c|c|c|c|}
                \hline
                $\mathbf{f_{c_{exp}} [Hz]}$ & $\mathbf{f_{c_{teo}} [Hz]}$ & \textbf{Desviación de frecuencia [\%]} \\
                \hline
                $2.8 \pm 0.16 k$ & $2.42 \pm 0.48 \, k$ & $15.70$ \\
                \hline
              \end{tabular}
              \caption{Desviación Estándar de la Frecuencia de Corte de la Figura \ref{fig:retro_multiples}}
              \label{tab:exp_retro_multiples_frecorte}
            \end{table}

             Haciendo uso de la tabla \ref{tab:exp_retro_multiples}, tomando los valores que se hallan en las columnas de Ganancia y frecuencia, se puede obtener el Diagrama  Asintótico de Bode en magnitud (respuesta en frecuencia) como se tiene en la gráfica \ref{fig:resp_frec_retro_multiples} a continuación.

             \begin{figure}[H]
                \centering
                \renewcommand{\figurename}{Gráfica}
                \setcounter{figure}{41}
                \includegraphics[width=15cm]{Imagenes/resp_frec_retro_multiples.png}
                \caption{Diagrama  Asintótico de Bode en magnitud (respuesta en frecuencia) de las mediciones experimentales de la tabla \ref{tab:exp_retro_multiples}}
                \label{fig:resp_frec_retro_multiples}
            \end{figure}

             \begin{itemize}
                \item \textbf{Armónicos}

                     \begin{figure}[H]
                        \centering
                        \renewcommand{\figurename}{Imagen}
                        \setcounter{figure}{15}
                        \includegraphics[width=15cm]{Imagenes/armonico_retro_multiples.png}
                        \caption{Señal de Entrada (Azul) y Salida (Amarilla), donde se puede observar los cambios en la señal de salida con su tercer armónico coincidiendo con la frecuencia de corte teórico de la tabla \ref{tab:exp_retro_multiples_frecorte}}
                        \label{fig:armonico_retro_multiples}
                    \end{figure}
            
                    \begin{table}[H]
                        \centering
                        \begin{tabular}{|c|c|c|c|}
                            \hline
                            \textbf{time/div} $[s]$ & \textbf{Channel} & \textbf{voltios/div $[\volt]$} & \textbf{Acoplamiento} \\ \hline
                            $200 \, \mu$ & 1 (Azul) &  $500 \, m $ & AC \\ \hline
                            $200 \, \mu$ & 2 (Amarillo)  &   $1  $ & AC \\ \hline  
                        \end{tabular}
                        \caption{Escalas Usada en el Osciloscopio Digital UNI-T UTD2102CEX+}
                        \label{tab:escala_exp_retro_multiples_armonico}
                    \end{table}
            \end{itemize}


\subsection{Parte 4. Fuentes Lineales y Reguladores Monolíticos}\label{subsec:parte4}

    En este apartado, se calcularan las incertidumbres de regulación, corriente y voltaje mínimo, siendo las ecuaciones \ref{eqn:delta_corriente}, \ref{eqn:delta_reg} y \ref{eqn:delta_vmin}, que se hallan en la sección \ref{sec:apendice}. Las figuras que se le realizaron las mediciones experimentales fueron la \ref{fig:reg_sinct}, \ref{fig:regulador_sal_ajustable} y \ref{fig:fuente_corriente_variable}.

    Adicional,las incertidumbres del voltaje primario del transformador se puede hallar en la hoja de especificaciones del apartado de anexos \ref{sec:anexos}, en la imagen \ref{eq:DT830D}, esto es debido a que no se quiso arriesgar el osciloscopio al tener una mala práctica y de esta manera evitar accidentes, usando de manera opcional el multímetro digital DT830D. Al igual, que con las mediciones precisas realizadas con el osciloscopio por aumento de ruido en las mediciones, se toma la incertidumbre del equipo dependiendo de la medición realizada.

    \subsubsection{Regulador con tensión de salida fija}

        \begin{table}[H]
          \centering
          \begin{tabular}{|c|c|c|c|c|c|}
            \hline
            \textbf{Condición} & $\mathbf{V_p [V_{RMS}]}$ & $\mathbf{V_s [V_p]}$ & $\mathbf{V_r [V_p]}$ & $\mathbf{V_o [V_{DC}]}$ & \textbf{R} $\mathbf{[\ohm]}$ \\
            \hline
            SC & $120 \pm 1\%$ & $13.3 \pm 1\%$ & $80 \pm 4 m$ & $5.6 \pm 0.4$ & -  \\
            \hline
            CC & $120 \pm 1\%$ & $13.3 \pm 1\%$ & $340 \pm 10 m$ & $5.6 \pm 0.4$ & $240 \pm 5\%$ \\
            \hline
            CC & $120 \pm 1\%$ & $13.3 \pm 1\%$ & $2 \pm 0.1$ & $3.7 \pm 0.21$ & $10 \pm 5\%$  \\
            \hline
          \end{tabular}
          \caption{Mediciones Experimentales de la Figura \ref{fig:reg_sinct}}
          \label{exp_reg_sinct}
        \end{table}



        \begin{table}[H]
          \centering
          \begin{tabular}{|c|c|c|c|c|}
            \hline
            \textbf{Condición} & \textbf{R} $\mathbf{[\ohm]}$ & $\mathbf{V_{in_{7805}} [DC]}$& $I_{DC} [mA]]$ & \textbf{Reg [\%]} \\
            \hline
            SC  & - & $18 \pm 1$& - & - \\
            \hline
            CC & $240 \pm 5\%$ & $17 \pm 1$ & $23.33 \pm 2.03$& $0 \pm 0.12$ \\
            \hline
            CC & $10 \pm 5\%$ & $5.35 \pm 0.261$ &$370 \pm 40.04$ & $0 \pm 0.12$ \\
            \hline
          \end{tabular}
          \caption{Mediciones Experimentales y Resultados Indirectos de la Figura \ref{fig:reg_sinct}}
          \label{exp_reg_sinct2}
        \end{table}

        \begin{table}[H]
            \centering
            \begin{tabular}{|c|c|c|c|c|}
                \hline
                Condición & R $[\ohm]$ & $V_{r_{experimental}}$ & $V_{r_{\text{teórico}}}$ & Desviación $[\%]$  \\ \hline
                 CC & $240 \pm 5\%$ & $0.340 \pm 0.1$  &  $0.768 \pm 0.037$ & $55.73$ \\ \hline
            \end{tabular}
            \caption{Desviación Estándar del Voltaje de Rizo de la figura \ref{fig:reg_sinct}}
            \label{tab:desv_reg_sinct}
        \end{table}


    \subsubsection{Regulador con Tensión de Salida Ajustable}

        \begin{table}[H]
          \centering
          \begin{tabular}{|c|c|c|c|c|c|}
            \hline
            $\mathbf{R [\ohm]}$ & $\mathbf{XR_{v1}}$ & $\mathbf{V_{R_1} [V_p]}$ & $\mathbf{V_{osc} [V_p]}$ & $\mathbf{V_{occ} [V_p]}$ & $\mathbf{I_{polarizacion} [mA]}$ \\\hline
            \multirow{3}{5cm}{\centering $5.1 \, k \pm 5 \%$} & $0$ & $7.2 \pm 0.4$ & $7.2 \pm 0.4$ & $7.2 \pm 0.4$ & $ 1.41 \pm 0.11$ \\
            & $0.5$ & $5 \pm 0.4$ & $10 \pm 1$ & $10 \pm 1$ & $0.98 \pm 0.092$ \\
            & $1$ & $5 \pm 0.4$ & $15 \pm 1$ & $12 \pm 1$ & $0.98 \pm 0.092$ \\
            \hline
          \end{tabular}
          \caption{Mediciones Experimentales de la Figura \ref{fig:regulador_sal_ajustable}}
          \label{tab:exp_regulador_sal_ajustable}
        \end{table}

    \subsubsection{Fuente de Corriente Variable}

        \begin{table}[H]
          \centering
          \begin{tabular}{|c|c|c|c|c|c|}
            \hline
            $\mathbf{XR_{v1}}$ & $\mathbf{V_o [V_{DC}]}$ & $\mathbf{V_{D1} [V_{DC}]}$ & $\mathbf{V_{R_1} [V_{DC}]}$ & $\mathbf{V_{R_L} [V_{DC}]}$ & $\mathbf{I_{R_L} [mA]}$ \\
            \hline
            $0$ & $2.2 \pm 0.2$ & $1.1 \pm 0.2$ & $1 \pm 0.1$ & $0.4 \pm 0.04$ & $4 \pm 0.45$ \\
            \hline
            $0.5$ & $2.4 \pm 0.2$ & $1.8 \pm 0.2$ & $1.50 \pm 0.1$ & $0.6 \pm 0.1$ & $6 \pm 1.04$ \\
            \hline
            $1$ & $3.6 \pm 0.1$ & $1.9 \pm 0.1$ & $3.8 \pm 0.1$ & $1.6 \pm 0.1$ & $16 \pm 1.28$ \\
            \hline
          \end{tabular}
          \caption{Mediciones Experimentales de la Figura \ref{fig:fuente_corriente_variable}}
          \label{tab:exp_fuente_corriente_variable}
        \end{table}

        \begin{table}[H]
              \centering
              \begin{tabular}{|c|c|c|c|}
                \hline
                $\mathbf{XR_{v1}}$ & $\mathbf{I_{R_{Lexp}} [mA]}$ & $\mathbf{I_{R_{Lteo}} [mA]}$ & \textbf{Desviación de frecuencia [\%]} \\
                \hline
                $0$ & $4 \pm 0.45$ & $4.03 \pm 0.38 $ & $0.74$ \\
                \hline
                $0.5$ &  $6 \pm 1.04$   & $6.76 \pm 0.64 \, $ & $11.24$ \\
                \hline
                $1$ & $16 \pm 1.28$ & $20.83 \pm 4.03 $ & $23.19$ \\
                \hline
              \end{tabular}
              \caption{Desviación Estándar de la Corriente de la Figura \ref{fig:fuente_corriente_variable}}
                \label{tab:desv_fuente_corriente_variable}
        \end{table}


    
\newpage


%Análisis de Resultados

\section{Análisis de Resultados}

    \subsection{Parte 1. Osciladores}
        En el apartado \ref{sec:resultados}, se consiguen los resultados de la primera parte en el apartado \ref{subsec:parte1}.

        Como se tienen en las tablas de los apartados mencionado anteriormente, el más relevante en este espacio son las tablas \ref{tab:desviacion_puente_wien_sc} y \ref{tab:desviacion_puente_wien_control}, allí podemos observar que no hubo un error mayor al 65.36 \% permitiendo evidenciar que el diseño para la configuración del Puente de Wien con control de amplitud, fueron los adecuados para las mediciones experimentales realizadas y lo que se quería demostrar, sin embargo, el error dado de último tuvo que ver un poco con la posición que se tomo del potenciómetro, debido a un defecto del potenciómetro en su estructura afectando el contacto, importante siempre verificar este tipo de inconvenientes para evitar errores tan grandes. Por otro lado, se llevo a cabo a través de esa topología un control de amplitud a consecuencia de un oscilador o generador, debido a que el amplificador operacional no posee un voltaje de entrada, solo se determina su voltaje de saturación por los voltajes de polarización en ambos polos.

        Por otra parte, se evidencia en la tabla \ref{tab:exp_puente_wien_sincontrol} su único valor alcanzado en estabilidad fue con una variación del potenciómetro con un valor de x de 0.63, de allí en adelante o por debajo de ello se satura o corta, indicándonos que en la figura \ref{fig:puente_wien_control} sin control de amplitud, no permite valores extensos en el potenciómetro para obtener una salida estable. A diferencia de cuando se posee un control de amplitud, permitiéndonos controlar la señal de salida como se observa en la tabla \ref{tab:exp_puente_wien_control} donde al tener una valor del potenciómetro del 40 \% de su valor nominal, nos entrega una salida estable con un voltaje mucho menor al de saturación que seria de  $4 \pm 1 V$ indicando que hasta llegar al valor de 10\% del potenciómetro este se saturaría, permitiendo un control de amplitud adecuado, antes de que este se sature. 

        Se utiliza comúnmente como un puente de medición de frecuencia en el cual se puede ajustar la frecuencia de la señal hasta que se equilibra el puente. 

        En fin, los datos proporcionados en el sección \ref{sec:metodologia} de metodología fueron los adecuados permitiendo los objetivos de esta parte de la practica.

    \subsection{Parte 2. Multivibradores}    
        Como se observa en el apartado \ref{sec:resultados} en la segunda sección, nos dieron distintas tablas e imágenes para corroboran las características más importantes del multivibrador astable, este nos indica una oscilación donde no se mantiene estable hasta un cierto diseño para poder variar su frecuencia debido a la carga y descarga del capacitor, como se puede notar la tabla \ref{tab:exp_astable} y \ref{tab:desv_astable}, se tienen pequeños errores, indicando que los componentes usados en el diseño fueron los indicados para obtener una salida de 5KHz con una pequeña desviación de 9 \%. Se tiene la forma de onda de salida y del capacitor en la imagen \ref{fig:exp_astable_vc_vout}, donde se puede apreciar perfectamente el voltaje de salida saturada, generando una onda cuadrada donde esta mantiene la frecuencia a la que fue diseñada cercana a los 5 KHz, y se observa la carga y descarga del voltaje del capacitor delimitada por las subidas y bajadas de su voltaje de salida, recordando que es un circuito realimentado negativa y positivamente sin ninguna señal de entrada, permitiendo ser este circuito inestable sin embargo, se aprovecha esta singularidad para distintas aplicaciones donde puedas usar esa señal de salida cuadrada.

        Por otro lado, tenemos el generador Monoestable, que se puede comportar como un astable, sino fuese por el circuito añadido que se observa en la figura \ref{fig:monoestable}, donde se le inyecta un voltaje negativo para generar una diferencia de potencial que permita el diodo conducir corriente y este pueda tener una salida estable cuando conduce, por lo contrario, se convierte en un circuito inestable con un posible tiempo de retardo que se indicará más adelante con la referencia de las imágenes que se hallan en el sección de resultados \ref{sec:resultados}.

        En la tabla \ref{tab:exp_monoestable}, tenemos los puntos importantes medidos en el circuito \ref{fig:monoestable}, en este apartado tomaremos en cuenta la frecuencia máxima, la del pulso y el voltaje en ambas entradas del amplificador operacional. 

        En la imagen \ref{fig:exp_monoestable_vc_vout}, se observa como el voltaje de salida se sigue manteniendo en un pulso pero en este caso con un duty cycle de 35.7 \%, acá detalla ese lado estable e inestable, siendo el estable el ciclo de encendido y el inestable el de apagado, en la siguiente imagen \ref{fig:exp_monoestable_vin_vout}, se puede evidencias como por cada pulso de entrada negativo, cuando este va a -5V el voltaje de salida cambia allí nos indica un pulso, más adelante en la siguiente imagen \ref{fig:exp_monoestable_vcat_vc} se detalla de mejor manera en la medición del circuito como por cada activación en su estado estable, es cuando el diodo que se encuentra en la entrada no inversora conduce, y cuando este ya no posee, la diferencia de potencial umbral que permite que el diodo conduzca se observa el lado inestable, cuando se convierte es un multivibrador astable.

        Ahora uno de los datos más importante es el siguiente, donde se visualiza el pulso donde permite el cambio de estado visualizando el ciclo de carga y descarga del capacitor, donde apreciamos su estado estable e inestable en el circuito de medición.

        De esta manera, podemos concluir en este análisis que el diseño realizado fue el adecuado para cumplir los objetivos de la practica y detallar el estudio de  circuitos no lineales utilizando el concepto de comparador haciendo uso de un amplificador operacional. 

        

    \subsection{Parte 3. Generador de Funciones}
        Como se observa en el apartado de resultados \ref{sec:resultados}, se tiene distintos valores de las mediciones realizadas en el laboratorio, como se puede ver la tabla \ref{tab:desviacion_gf}, existen distintas desviaciones sin embargo, la que le prestamos mas atención es la de la frecuencia donde se obtiene un 28.6 \% de error, esto lo que ocasiona es que la señal que se tiene en la salida del integrador no seria un señal triangular, sino una senoidal, debido a que este tiene una señal de entrada cuadrada por ser un integrador aquellos pulsos constantes, nos genera una recta, al tener el tiempo menos rápido genera un tiempo de retardo permitiendo tener una señal de salida del integrador mas suave. 

        Por otro lado, el voltaje de salida es mas alto de lo que se visualizo en la simulación del diseño realizado teóricamente, como se observa en la tabla. Lo importante del diseño y que se cumple es que el voltaje de salida del integrador, no supere ese voltaje debido a que este puede saturarse, cumpliendo con las especificaciones del diseño.

        Las mediciones, análisis y diseño fueron los adecuados para esta practica. 

        Otro dato importante es los diodos que se encuentra en la etapa de salida, lo que permite estos son la regulación de voltaje a consecuencia del diodo zener, y el puente de diodos lo que nos permite es que esto funcione tanto para los valores positivos como los negativos.
    


\newpage

%Conclusiones

\section{Conclusiones}

En relación con los objetivos planteados en la práctica, se lograron resultados coherentes con las expectativas teóricas previas. Específicamente en la etapa de potencia, las mediciones realizadas en el laboratorio se asemejaron significativamente a los valores teóricos. No se experimentaron inconvenientes con los materiales ni con los instrumentos de medición, y las formas de onda de salida no presentaron ruido perceptible.

En cuanto a la etapa diferencial, a pesar de que se observaron errores porcentuales notables en comparación con las mediciones teóricas de impedancias en modo común y diferencial, se pudo realizar una comparación detallada y comprender los límites de operación para cada configuración.

En la etapa impulsora, aunque inicialmente todo transcurrió según lo previsto, se enfrentaron desafíos significativos debido a cambios en los transistores utilizados. Estos cambios impactaron considerablemente en las impedancias y la saturación de la señal de salida, especialmente al observar los límites de excursión para esta configuración. A pesar de estos contratiempos, las mediciones fueron aceptables, excepto por la impedancia, que mostró un error porcentual notable.

En la respuesta en frecuencia, se observaron cambios en las mediciones debido a modificaciones en los transistores. Sin embargo, el barrido en frecuencia se realizó de manera eficiente, obteniendo las frecuencias bajas y altas necesarias para construir el diagrama de Bode requerido.

Desafortunadamente, la etapa de realimentación concluyó de manera adversa, ya que no fue posible realizar mediciones. Los transistores utilizados se calentaron en exceso, resultando en la quema de varias resistencias. La causa de este inconveniente no pudo determinarse con certeza, pero se especula que podría atribuirse a diferencias entre los transistores utilizados, lo cual es crítico para la realimentación.

A pesar de los desafíos experimentados al final de la práctica, se adquirieron los conocimientos necesarios para alcanzar los objetivos propuestos.

En el ámbito teórico, los amplificadores de potencia desempeñan un papel crucial en la amplificación de señales eléctricas a niveles suficientes para controlar la potencia de salida de dispositivos como altavoces. Estos amplificadores, que pueden diseñarse como Clase A, B, AB, C, entre otros, ofrecen diferentes características de rendimiento, eficiencia y distorsión.

La etapa diferencial, una configuración común de circuito que utiliza transistores complementarios, resulta esencial para proporcionar alta ganancia y rechazo de ruido común. Además, la etapa diferencial puede configurarse para amplificar señales en modo común, facilitando la eliminación efectiva del ruido común en las entradas.

La etapa impulsora, encargada de amplificar la señal de entrada a niveles suficientes para controlar la etapa de potencia, debe proporcionar la ganancia y corriente adecuadas. La respuesta en frecuencia, esencial para aplicaciones de audio, puede verse afectada por componentes capacitivos e inductivos en el circuito y por la configuración de la realimentación.

La realimentación, un concepto clave en el diseño de amplificadores, puede mejorar el rendimiento reduciendo la distorsión y mejorando la estabilidad. Sin embargo, su implementación debe llevarse a cabo cuidadosamente para evitar problemas de estabilidad y ruido, así como para mantener la eficiencia del amplificador.

En resumen, a pesar de los desafíos encontrados durante la práctica, se lograron los objetivos planteados, proporcionando una comprensión profunda de los amplificadores discretos y sus diversas etapas.

\newpage

%Apéndice

\section{Apéndice}\label{sec:apendice}
\begin{itemize}
    \item Incertidumbre según "Guide of Uncertain Measurements" de la ISO o GUM

        \begin{gather}
            \Delta f=\sqrt{\sum\left( \dfrac{\partial f(\Bar{X} }{\partial X_i} \Delta X_i\right)^2}
        \end{gather}


    \item Incertidumbre de Frecuencia
        \begin{gather}
            \Delta f = \sqrt{\left(-\dfrac{1}{T^2}\Delta T\right)^2}=\dfrac{1}{T^2}\Delta T
            \label{eqn:delta_frecuencia}
        \end{gather}

        
    \item Error o Desviación Estándar
        \begin{gather}
            Desv=\dfrac{|Valor_{teorico}-Valor_{Experimental}|}{Valor_{teorico}} \, 100\%
            \label{eqn:desviación}
        \end{gather}
        
\end{itemize}

\newpage

%Anexos

\section{Anexos}\label{sec:anexos}


\includepdf[pages=- scale=0.9, fitpaper=true]{pdf/P2 Hoja de datos 1 (2).pdf}
\includepdf[pages=- scale=0.9, fitpaper=true]{pdf/P2 Hoja de datos 2.pdf}
\includepdf[pages=- scale=0.9, fitpaper=true]{pdf/Hoja de datos 3 P2.pdf}
\includepdf[pages=- scale=0.9, fitpaper=true]{pdf/Hoja de datos 4 P2.pdf}
\includepdf[pages=- scale=0.9, fitpaper=true]{pdf/LM741.PDF}
\newpage
\includepdf[pages=1-3, scale=0.8, fitpaper=true]{pdf/KIA7805AP.PDF}
\includepdf[pages=14-20, scale=0.8, fitpaper=true]{pdf/KIA7805AP.PDF}
\includepdf[pages=-, scale=0.9,fitpaper=true]{pdf/utd2000_series-001.pdf}
\includepdf[pages=-, scale=0.8, fitpaper=true]{pdf/UTP3305-II.pdf}
\includepdf[pages=-, scale=0.8, fitpaper=true]{pdf/UTG932E.pdf}

\begin{figure}[H]
    \centering
    \includegraphics[width=8cm]{Imagenes/DT830D.png}
    \label{eq:DT830D}
\end{figure}

\subsection{Códigos de Octave}\label{subsec:cod_octave}
    \begin{itemize}
        \item Comparación de puntos
            
                \lstinputlisting[language=Octave,basicstyle=\small, caption={Código Octave para análisis de puntos}, label=lst:comparacion]{Scripts/comparacion_puntos.m}
\newpage                
        \item Diagrama asintótico de Bode
            
                \lstinputlisting[language=Octave,basicstyle=\small, caption={Código Octave para análisis de puntos}, label=lst:resp.frecuencia]{Scripts/resp_frecuencia.m}
    \end{itemize}
\newpage








%Bibliografía

\begin{thebibliography}{9}

       
    \bibitem{Bautista2019}
      Bautista, A. (2019). Estudio de circuitos eléctricos en estado de circuito abierto. Revista de Investigación Académica, 36, 1-7. doi: 10.18359/ria.3636
    
    \bibitem{HernandezTorres2019}
      Hernández, M., \& Torres, J. (2019). Diseño y construcción de fuentes de alimentación lineales. Tecnología en Marcha, 32, 52-57. doi: 10.18845/tm.v32i1.3677
    
    \bibitem{HorowitzHill2015}
      Horowitz, P., \& Hill, W. (2015). \textit{The Art of Electronics} (3rd ed.). Cambridge University Press. (Capítulo 4, sección 4.4.1, Capítulo 4, sección 4.5.2). DOI: 10.1017/CBO9781139643773
        
    \bibitem{GrayHurstLewisMeyer2001}
      Gray, P. R., Hurst, P. J., Lewis, S. H., \& Meyer, R. G. (2001). \textit{Analysis and Design of Analog Integrated Circuits} (4th ed.). John Wiley \& Sons. (Capítulo 2, sección 2.3.1). DOI: 10.1109/EDT.2000.882767
    
    \bibitem{MehdiRazavi1998}
      Mehdi, I., \& Razavi, B. (1998). A Flicker-Noise Measurement Technique Using a Vanishing-Gain Amplifier. \textit{IEEE Journal of Solid-State Circuits}, 33(5), 791-796. DOI: 10.1109/4.668882
    
    \bibitem{LiuLu2013}
      Liu, Z., \& Lu, Y. (2013). Design of a High Precision Operational Amplifier with Multiple Feedback Loops. \textit{Journal of Electrical and Computer Engineering}, 2013, 1-7. DOI: 10.1155/2013/276765
        
    \bibitem{SedraSmith2015}
      Sedra, A. S., \& Smith, K. C. (2015). \textit{Microelectronic Circuits} (7th ed.). Oxford University Press. (Capítulo 5, sección 5.2.3, Capítulo 8, sección 8.1, Capítulo 11, sección 11.1.2, Capítulo 11, sección 11.3). DOI: 10.1093/acprof:oso/9780199339136.003.0016
    
    \bibitem{Smith2008}
      Smith, J. (2008). Amplificadores operacionales: conceptos y aplicaciones. \textit{Revista de Electrónica}, 5(2), 25-30. Disponible en: \url{https://www.revelec.com/articulos/numeros/vol5num2/Articulo5_Vol5Num2.pdf} Consultado: 29 de junio.

\end{thebibliography}



\label{LastPage}
\end{document}
