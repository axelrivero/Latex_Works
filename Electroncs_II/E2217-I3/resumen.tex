%Resumen

\section{Resumen}

Esta practica n° 3 explora a fondo las aplicaciones no lineales de amplificadores operacionales en tres áreas fundamentales: Osciladores, Multivibradores y Generadores de Funciones. En la primera sección, dedicada a los osciladores, se destacan los principios detrás de la generación de oscilaciones sinusoidales y la importancia de la retroalimentación positiva para estabilizar y controlar la amplitud de las oscilaciones. La sección concluye resaltando la necesidad de un control de amplitud para mantener la oscilación sinusoidal.

La segunda sección se centra en los multivibradores, explorando los efectos de la realimentación positiva cuando la ganancia del lazo excede la unidad. Se analizan circuitos no lineales mediante el concepto de comparador, y se establecen las diferencias entre un dispositivo comparador y un amplificador operacional utilizado con esa finalidad. La aplicación práctica de los multivibradores en la generación de señales pulsantes se destaca como un aspecto crucial.

La tercera sección aborda los generadores de funciones, examinando los efectos de la realimentación positiva cuando la ganancia del lazo es mayor que la unidad. Se profundiza en el funcionamiento de generadores basados en amplificadores operacionales y se resalta la importancia de la selección adecuada de componentes y la configuración del circuito para lograr resultados precisos y estables.
\newpage