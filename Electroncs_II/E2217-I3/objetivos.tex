%Objetivos generales y Específicos

\section{Objetivo General y Específico}

    \subsection*{Parte 1. Osciladores}
    
        \subsubsection*{Objetivo General:}
        Comprender los fundamentos físicos que posibilitan la generación de oscilaciones sinusoidales mediante el uso de amplificadores operacionales.
        
        \subsubsection*{Objetivos Específicos:}
        \begin{enumerate}
            \item Reconocer y entender las ventajas inherentes al empleo de amplificadores operacionales en el diseño y desarrollo de sistemas analógicos.
            \item Identificar la necesidad y la utilidad de implementar un control de amplitud como mecanismo esencial para el mantenimiento y sostenimiento de oscilaciones sinusoidales en un circuito.
            \item  Entender la importancia de la retroalimentación positiva en la generación de oscilaciones sinusoidales.
            \item Seleccionar adecuadamente componentes para lograr estabilidad y control de amplitud efectivo.
        \end{enumerate}
    
    \subsection*{Parte 2. Multivibradores}
    
        \subsubsection*{Objetivo General:}
        Analizar los efectos de la realimentación positiva cuando la ganancia del lazo es mayor que la unidad, centrándose en el estudio de multivibradores.
        
        \subsubsection*{Objetivos Específicos:}
        \begin{enumerate}
            \item Analizar circuitos no lineales mediante la aplicación del concepto de comparador.
            \item Diferenciar las características y funciones de un dispositivo comparador y de un amplificador operacional utilizado como comparador.
            \item Enfatizar en cómo la realimentación positiva influye en la estabilidad y comportamiento de los multivibradores.
        \end{enumerate}
    
    \subsection*{Parte 3. Generador de funciones}
    
        \subsubsection*{Objetivo General:}
        Examinar los efectos de la realimentación positiva en situaciones donde la ganancia del lazo es mayor que la unidad, centrándose en el análisis de generadores de funciones.
        
        \subsubsection*{Objetivos Específicos:}
        \begin{enumerate}
            \item Analizar el funcionamiento detallado de generadores de funciones, específicamente aquellos basados en amplificadores operacionales y retroalimentación positiva.
            \item Explorar circuitos no lineales empleando el concepto de comparador en el contexto de generadores de funciones.
            \item Observar cómo la realimentación positiva contribuye a la forma de onda generada.
        \end{enumerate}

    

\newpage