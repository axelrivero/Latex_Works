%Conclusiones

\section{Conclusiones}


La realización de la práctica sobre aplicaciones no lineales de amplificadores operacionales ha proporcionado resultados satisfactorios y ha permitido un análisis detallado de tres importantes componentes: osciladores, multivibradores y generadores de funciones.

En la sección dedicada a los osciladores, se ha observado el comportamiento del Puente de Wien con control de amplitud. Los resultados obtenidos en las mediciones experimentales han demostrado que el diseño para esta configuración fue acertado, logrando un control de amplitud efectivo. Se ha destacado la importancia de la retroalimentación positiva en la estabilidad y control de la señal de salida. Además, se evidenció que la presencia de un control de amplitud permitió mantener una salida estable en un rango más amplio de ajustes, en comparación con la configuración sin control de amplitud.

En la sección dedicada a los multivibradores, se ha explorado tanto el comportamiento del multivibrador astable como del monoestable. Los resultados indican que el multivibrador astable genera una onda cuadrada con una frecuencia controlada por los componentes del circuito, y se observa la carga y descarga del capacitor durante el ciclo. En cuanto al monoestable, se ha resaltado su capacidad para generar pulsos de salida de duración controlada a través de la inyección de un voltaje negativo.

Finalmente, en la sección del generador de funciones, se ha analizado el comportamiento del circuito integrador con la retroalimentación de un amplificador operacional. Aunque se han observado algunas desviaciones en las mediciones, se ha concluido que el diseño permitió cumplir con los objetivos de la práctica. Se destacó la importancia de los diodos en la etapa de salida para la regulación de voltaje.

En conclusión, la práctica ha proporcionado una comprensión más profunda de los principios y aplicaciones de los amplificadores operacionales en circuitos no lineales, demostrando la importancia de la retroalimentación positiva y la elección adecuada de componentes en el diseño de estos circuitos.
\newpage