%Introducción

\section{Introducción}
En el apasionante mundo de la electrónica, los Amplificadores Operacionales (AO) sobresalen como componentes esenciales que desempeñan un papel crucial en el diseño y la implementación de sistemas analógicos. En nuestra práctica N° 2, nos sumergimos en el estudio detallado de las aplicaciones lineales de estos dispositivos versátiles, explorando desde las topologías clásicas hasta los desafíos y oportunidades que presenta el uso de amplificadores operacionales reales.

A lo largo de este informe técnico, nos adentramos en el fascinante universo de los amplificadores operacionales, desde la teoría hasta la práctica, explorando su aplicación en diversas configuraciones y desafiándonos a comprender y superar las limitaciones de los dispositivos reales. En este contexto, resaltamos la asombrosa versatilidad del Circuito Integrado LM741, un amplificador operacional de propósito general ofrece posibilidades intrigantes dependiendo del diseño implementado. Además, exploramos el funcionamiento interno de un cargador de 5V, destacando la complejidad de este sistema compuesto por elementos pasivos y activos, incluyendo diodos y circuitos integrados, que se combinan para formar una fuente regulable. Este análisis teórico nos permite comprender cómo diseñar y ajustar estas configuraciones para satisfacer las especificaciones necesarias en la carga eficiente de dispositivos.

\newpage