%Resumen

\section{Resumen}

En la práctica N° 2 sobre las aplicaciones lineales del Amplificador Operacional, se abordan diversas temáticas divididas en cuatro partes. En la primera sección, se busca reconocer las ventajas del uso de amplificadores operacionales en sistemas analógicos, explorando desviaciones de implementaciones comerciales y comparando su rendimiento con sistemas discretos. El trabajo de preparación incluye la determinación de conexiones para diversas topologías y la propuesta de ensayos para verificar su funcionamiento, seguido de simulaciones para confirmar los resultados teóricos.

La segunda parte se centra en el Amplificador Operacional Real, con el objetivo de reconocer las desviaciones respecto al modelo ideal. El trabajo de preparación implica explicar la medición de la tensión de Offset y la corriente de polarización, así como comprobar que el Producto del Ancho de Banda por la Ganancia se mantenga. Se propone la construcción de hojas de datos para el laboratorio, donde se realizarán ensayos y mediciones.

En la tercera sección, dedicada a los Filtros Activos, el objetivo general es reconocer las ventajas del uso de amplificadores operacionales en el diseño de sistemas analógicos. El trabajo de preparación abarca la obtención de modelos circuital y la especificación de componentes para filtros pasa bajos, pasa banda y pasa altos. Se propone verificar los diseños mediante simulaciones y comparar las respuestas en frecuencia con diagramas de Bode.

Finalmente, en la cuarta parte se exploran las Fuentes Lineales y Reguladores Monolíticos. El objetivo general es analizar el funcionamiento de topologías de fuentes reguladas lineales. El trabajo de preparación incluye explicar funciones y cálculos para fuentes reguladas fijas y ajustables, así como determinar rangos de corriente y tensión. El laboratorio implica la implementación de los circuitos propuestos y la realización de ensayos para validar los resultados. 
\newpage