%Conclusiones

\section{Conclusiones}

En relación con los objetivos planteados en la práctica, se lograron resultados coherentes con las expectativas teóricas previas. Específicamente en la etapa de potencia, las mediciones realizadas en el laboratorio se asemejaron significativamente a los valores teóricos. No se experimentaron inconvenientes con los materiales ni con los instrumentos de medición, y las formas de onda de salida no presentaron ruido perceptible.

En cuanto a la etapa diferencial, a pesar de que se observaron errores porcentuales notables en comparación con las mediciones teóricas de impedancias en modo común y diferencial, se pudo realizar una comparación detallada y comprender los límites de operación para cada configuración.

En la etapa impulsora, aunque inicialmente todo transcurrió según lo previsto, se enfrentaron desafíos significativos debido a cambios en los transistores utilizados. Estos cambios impactaron considerablemente en las impedancias y la saturación de la señal de salida, especialmente al observar los límites de excursión para esta configuración. A pesar de estos contratiempos, las mediciones fueron aceptables, excepto por la impedancia, que mostró un error porcentual notable.

En la respuesta en frecuencia, se observaron cambios en las mediciones debido a modificaciones en los transistores. Sin embargo, el barrido en frecuencia se realizó de manera eficiente, obteniendo las frecuencias bajas y altas necesarias para construir el diagrama de Bode requerido.

Desafortunadamente, la etapa de realimentación concluyó de manera adversa, ya que no fue posible realizar mediciones. Los transistores utilizados se calentaron en exceso, resultando en la quema de varias resistencias. La causa de este inconveniente no pudo determinarse con certeza, pero se especula que podría atribuirse a diferencias entre los transistores utilizados, lo cual es crítico para la realimentación.

A pesar de los desafíos experimentados al final de la práctica, se adquirieron los conocimientos necesarios para alcanzar los objetivos propuestos.

En el ámbito teórico, los amplificadores de potencia desempeñan un papel crucial en la amplificación de señales eléctricas a niveles suficientes para controlar la potencia de salida de dispositivos como altavoces. Estos amplificadores, que pueden diseñarse como Clase A, B, AB, C, entre otros, ofrecen diferentes características de rendimiento, eficiencia y distorsión.

La etapa diferencial, una configuración común de circuito que utiliza transistores complementarios, resulta esencial para proporcionar alta ganancia y rechazo de ruido común. Además, la etapa diferencial puede configurarse para amplificar señales en modo común, facilitando la eliminación efectiva del ruido común en las entradas.

La etapa impulsora, encargada de amplificar la señal de entrada a niveles suficientes para controlar la etapa de potencia, debe proporcionar la ganancia y corriente adecuadas. La respuesta en frecuencia, esencial para aplicaciones de audio, puede verse afectada por componentes capacitivos e inductivos en el circuito y por la configuración de la realimentación.

La realimentación, un concepto clave en el diseño de amplificadores, puede mejorar el rendimiento reduciendo la distorsión y mejorando la estabilidad. Sin embargo, su implementación debe llevarse a cabo cuidadosamente para evitar problemas de estabilidad y ruido, así como para mantener la eficiencia del amplificador.

En resumen, a pesar de los desafíos encontrados durante la práctica, se lograron los objetivos planteados, proporcionando una comprensión profunda de los amplificadores discretos y sus diversas etapas.

\newpage