%Resumen

\section{Resumen}

El presente resumen condensa el análisis exhaustivo realizado sobre amplificadores multietapas, dispositivos esenciales en el ámbito de la electrónica. Estos amplificadores, caracterizados por su capacidad para aumentar señales de manera secuencial, juegan un papel crucial en el diseño y rendimiento de sistemas electrónicos avanzados. Durante este estudio, se exploraron detalladamente las propiedades de un amplificador multietapas. 

Se incorpora un software de simulación (LTspice) especializado para estos circuitos, constituyendo una herramienta integral que respalda y verifica cada aspecto de la caracterización y la implementación experimental llevada a cabo en el laboratorio. Esta aplicación se revela como una guía esencial durante el desarrollo práctico, permitiendo una validación precisa de los resultados teóricos y experimentales, así como una valiosa asistencia en la interpretación y análisis de los datos obtenidos.

 La sección inicial se centra en el análisis dinámico y la polarización de etapas amplificadoras de potencia (EP), incluyendo la caracterización de distorsiones y el diseño previo al laboratorio. La implementación experimental implica el montaje, medición de puntos de operación y la determinación de modelos dinámicos. 

En la sección siguiente, se examina un Amplificador Diferencial (ED), modelando circuitalmente etapas y verificando experimentalmente la polarización. La distinción entre señales modo común y modo diferencial es clave. El laboratorio implica la construcción y análisis detallado del amplificador, incluyendo el diseño de hojas de datos y mediciones experimentales. 

Posteriormente, se aborda un Amplificador Multietapas, destacando la importancia de acoplamientos capacitivos. El objetivo es analizar aspectos como la ganancia, impedancia de entrada/salida y tensión de alimentación, respaldado por la construcción y caracterización experimental. 

La sección sobre Respuesta en Frecuencia busca definir el modelo de banda ancha y analizar los efectos de reactancias, con una preparación detallada y laboratorio que involucra simulaciones y mediciones experimentales. 

Finalmente, se explora el impacto de la realimentación en un amplificador, detallando los objetivos de la sección y su implementación en el laboratorio. 

\newpage