%Objetivos generales y Específicos

\section{Objetivo General y Específico}
\subsection{Objetivo General}
El presente informe tiene como propósito primordial alcanzar los siguientes objetivos:

\begin{enumerate}
    \item Analizar detalladamente el comportamiento de un Amplificador de Potencia basado en un diseño de Amplificador Diferencial.
    \item Examinar e identificar las distintas etapas de un amplificador multietapa, determinando sus características distintivas, especialmente en lo que respecta al acoplamiento capacitivo y su ganancia.
    \item Evaluar la respuesta en frecuencia del amplificador multietapas para lograr una comprensión profunda de sus propiedades y rendimiento en función de la frecuencia.
    \item Investigar el impacto de la realimentación en el comportamiento global del amplificador, explorando sus efectos en la estabilidad, ganancia y linealidad del sistema.
    \item Obtener una visión integral de las características fundamentales del sistema, centrándose en la interacción entre el diseño de Amplificador Diferencial, el acoplamiento capacitivo, y la realimentación.
    \item Proporcionar una base sólida para la comprensión y aplicación de los principios analizados en contextos electrónicos.
\end{enumerate}


\subsection{Objetivos Específicos}
\begin{enumerate}
    \item Calcular y medir la polarización de los Amplificadores de potencia B y AB
    \item Caracterizar, en dinámico, etapas amplificadoras de potencia B y AB.
    \item Determinar las impedancias de entrada y salida de las distintas etapas.
    \item Reconocer las distorsiones generadas por las etapas de simetría complementaria. 
    \item Determinar la polarización de un Amplificador diferencial y verificarla experimentalmente.
    \item Distinguir entre una señal de entrada modo común y modo diferencial.
    \item Construir el amplificador con los transistores existentes comercialmente y con la especificaciones y curvas suministradas por el fabricante.
    \item Analizar un amplificador de múltiples etapas, acoplado capacitivamente de acuerdo a: Ganancia del amplificador, Impedancia de entrada y salida. Tensión de alimentación.
    \item Caracterizar el amplificador multietapas con los valores obtenidos experimental y teóricamente.
    \item Definir el modelo de banda ancha para amplificadores.
    \item Analizar los efectos de las reactancias en la respuesta en frecuencia de un amplificador y discriminar aquellas que actúan en la región de bajas frecuencias de la que actúan en altas frecuencias.
    \item Reconocer que la realimentación negativa reduce la ganancia y a cambio ofrece:
    \begin{itemize}
        \item Mayor independencia de la ganancia del valor de la ganancia del amplificador base.
        \item Disminución de la impedancia de salida.
        \item Aumento de la impedancia de entrada.
        \item Aumento del ancho de banda.
        \item Mejoramiento de la linealidad del amplificador.
    \end{itemize}
\end{enumerate}

\newpage