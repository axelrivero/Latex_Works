\section{Planteamiento del problema}

    Las líneas de transmisión son esenciales en la distribución de energía eléctrica desde los generadores hasta los consumidores. Sin embargo, durante este proceso, se producen diversas pérdidas que afectan la eficiencia del sistema de transmisión. Estas pérdidas no solo representan un costo económico significativo, sino que también impactan la estabilidad y confiabilidad del suministro eléctrico. Es crucial entender y mitigar estas pérdidas para mejorar la eficiencia energética y reducir los costos operativos.
    
    En el ámbito de la transmisión de energía, se identifican varios tipos de pérdidas que ocurren debido a diferentes factores:

    \begin{enumerate}
        \item Pérdida en el conductor.

        \item Pérdida por calentamiento del dieléctrico.

        \item Pérdida por radiación.

        \item Pérdida por acoplamiento.

        \item Efecto corona.
    \end{enumerate}

    \subsection{Objetivo}
    
        El objetivo de este estudio es investigar y analizar los procedimientos para medir cada uno de los tipos de pérdidas mencionados en las líneas de transmisión de energía. Además, se diseñará un montaje práctico en un laboratorio para medir uno de estos tipos de pérdidas, proporcionando así una metodología clara y precisa para su evaluación.

    \subsection{Preguntas de investigación}
    
        \begin{itemize}
            \item ¿Cuáles son los procedimientos más efectivos para medir la pérdida en el conductor en las líneas de transmisión?
            \item ¿Cómo se pueden cuantificar las pérdidas por calentamiento del dieléctrico?    
            \item ¿Qué métodos se utilizan para medir la pérdida por radiación en diferentes condiciones de transmisión?    
            \item ¿Qué técnicas existen para evaluar las pérdidas por acoplamiento entre líneas de transmisión?    
            \item ¿Cómo se puede medir y mitigar el efecto corona en las líneas de transmisión?
        \end{itemize}

    \subsection{Importancia}

        Este estudio es crucial para mejorar la eficiencia de los sistemas de transmisión de energía eléctrica. Al identificar y cuantificar adecuadamente las diversas pérdidas, se pueden desarrollar estrategias más efectivas para mitigarlas.





\newpage