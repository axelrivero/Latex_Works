\section{Conclusiones y posibles aplicaciones}


   En este trabajo se simuló y analizó un caso hipotético en el que se evaluaron las pérdidas resistivas en diferentes conductores, utilizando Python para modelar las propiedades resistivas de cada material. Los conductores analizados fueron cobre, aluminio y oro, considerando distintos valores de resistividad, diámetro y corriente. Esta simulación permitió obtener una visión simplificada de cómo la resistividad y el diámetro del conductor afectan las pérdidas de potencia en una línea de transmisión.

    De acuerdo con los resultados obtenidos, se concluyó que el mejor material para mitigar las pérdidas resistivas en los conductores es el cobre. Si bien el oro también podría ofrecer buenos resultados debido a su baja resistividad, su alto costo y uso limitado en aplicaciones específicas de electrónica lo hacen menos viable para aplicaciones de transmisión a gran escala. Al analizar los datos de la tabla \ref{fig:sim_datos_perdidas}, se observó que las pérdidas con el cobre fueron menores en un $31.15 \%$ en comparación con el aluminio, lo que resulta en una menor resistencia total. Se evidenció que la resistencia es directamente proporcional a las pérdidas resistivas y que estas pueden reducirse aumentando el área transversal del conductor (en este caso, con un cable de longitud fija de 1000 m), ya que un mayor diámetro reduce las pérdidas resistivas.
    
    Por otra parte, tanto el aluminio como el oro pueden descartarse como materiales preferidos para una línea de transmisión, lo que coincide con las prácticas industriales donde el cobre es el material de referencia para la mayoría de los cables conductores, aunque existen excepciones en situaciones específicas.
    
    En cuanto a los métodos de medición de las pérdidas, la elección del procedimiento más adecuado dependerá de las herramientas disponibles, la complejidad del proyecto y la experiencia del personal a cargo. Sin embargo, es importante destacar que las pérdidas en una línea de transmisión pueden medirse de manera simplificada comparando la potencia de entrada con la potencia de salida. Cualquier diferencia entre ambas representaría la cantidad de energía disipada en forma de calor debido a las pérdidas resistivas.
    
    Finalmente, en la sección \ref{sec:dispositivos} se presentaron distintos dispositivos y técnicas para mitigar las pérdidas en sistemas eléctricos, proporcionando alternativas para mejorar la eficiencia. Aunque inicialmente se consideró la posibilidad de simular los circuitos mediante herramientas como Multisim, Proteus o LTSpice, se optó por un enfoque más eficiente y directo, centrado en la simulación de las pérdidas resistivas en los conductores, lo que permitió demostrar con claridad los efectos de la resistividad y el diámetro en la eficiencia del sistema.


\newpage
