\section{Introducción}

    Las líneas de transmisión eléctricas juegan un papel fundamental en la infraestructura energética moderna, permitiendo la transferencia eficiente de energía desde los centros de generación hasta los puntos de consumo. No obstante, el diseño y la operación de estas líneas presentan una serie de desafíos técnicos que deben ser abordados para optimizar su rendimiento y minimizar las pérdidas de energía. Entre los principales problemas se encuentran las diversas pérdidas de energía que ocurren durante la transmisión, lo que puede reducir la cantidad de energía útil que llega al destino.

    Existen varios tipos de pérdidas en las líneas de transmisión, entre las cuales se destacan las pérdidas resistivas, las pérdidas por radiación, las pérdidas por calentamiento del dieléctrico y las pérdidas por efecto corona. Las pérdidas resistivas, ocurren debido a la resistencia inherente de los materiales conductores y están directamente relacionadas con la longitud de la línea y la corriente que fluye a través de ella. Por otro lado, las pérdidas por radiación y calentamiento del dieléctrico son más prominentes a frecuencias más altas y pueden reducirse mediante el uso de materiales adecuados y técnicas de blindaje. El efecto corona, que ocurre principalmente en líneas de alta tensión, genera pérdidas significativas de energía debido a la ionización del aire alrededor de los conductores.
    
    Este trabajo tiene como objetivo analizar las pérdidas resistivas en los conductores más utilizados en líneas de transmisión: cobre, aluminio y oro. Para ello, se ha implementado una simulación en Python, donde se modelan los efectos de la resistividad, el diámetro del conductor y la corriente sobre las pérdidas de energía. Los resultados obtenidos permitirán proponer soluciones viables para mejorar la eficiencia de las líneas de transmisión, enfocándose en la reducción de las pérdidas de potencia.
    
    Además, se evaluarán métodos para medir y mitigar las pérdidas mencionadas, brindando una visión más completa de las diferentes estrategias de optimización disponibles. Finalmente, se presentarán propuestas de diseño que permitan aumentar la eficiencia de las líneas de transmisión, siempre considerando la viabilidad técnica y económica de cada opción.




\newpage