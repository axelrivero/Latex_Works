% Tipo de documento
\documentclass[a4paper,12pt,spanish]{article}

% Configuración de codificación y lenguaje
\usepackage[es-tabla]{babel}
\usepackage[utf8]{inputenc}
\usepackage[T1]{fontenc}    % Buena codificación de fuentes

% Configuración de fuentes y estilo de página
\usepackage{newtxtext,newtxmath}
\usepackage{titlesec}
\usepackage{fancyhdr}
\usepackage{setspace}
\usepackage[margin=1in , headheight=20pt]{geometry}

% Configuración de gráficos, tablas, listas y figuras
\usepackage{caption}
\usepackage{float}
\usepackage{graphicx}
\usepackage{enumitem}
%\usepackage{pdfpages}

% Configuración de matemáticas y unidades
\usepackage{amsmath}
\usepackage{amsfonts}
\usepackage{siunitx}

% Carga del paquete csquotes para manejar citas correctamente
\usepackage{csquotes}
% Carga del paquete biblatex (bibliografía)
\usepackage[backend=biber, style=ieee]{biblatex} 
% Añadir la base de datos de bibliografía
\addbibresource{bibliografia.bib} 

% Configuración de enlaces y referencias
\usepackage[colorlinks=true, urlcolor=blue, linkcolor=black, hidelinks]{hyperref}
\usepackage{lastpage}


%Puedes incluir el código de cualquier LP dentro de un entorno lstlisting utilizando el paquete listings.
\usepackage{listings} 
%Ajusta los colores de tu codigo
\usepackage{xcolor}

% Configuración de interlineado
\setstretch{1.5}

% Configuración de sangría
\parindent=1.27cm

% Configuración de títulos de secciones
\titleformat{\section}{\normalfont\Large\bfseries}{\thesection.}{1em}{}[\titlerule]

% Configuración de Figuras y Tablas
\captionsetup[figure]{labelfont={bf}}
%\captionsetup[table]{labelfont={bf}}

% Configuración de los encabezados y pie de página con fancyhdr
\pagestyle{fancy}
\fancyhf{}
\fancyhead[L]{\textit{\leftmark}}
\fancyhead[R]{\textit{\rightmark}}
\fancyfoot[L]{\textit{Redes Eléctricas III}}
\fancyfoot[C]{\textit{\thepage/\pageref{LastPage}}}
\fancyfoot[R]{\textit{Semestre 2024-1}}
\renewcommand{\footrulewidth}{1pt} % Ancho de la línea divisoria
\renewcommand{\headrulewidth}{1pt} % Establece el ancho de la línea divisoria del encabezado

%Configuración de listings
\lstset{
  language=Python,         % Lenguaje del código
  basicstyle=\ttfamily\small, % Estilo básico de la fuente
  keywordstyle=\color{blue},  % Color de las palabras clave
  commentstyle=\color{gray},  % Color de los comentarios
  stringstyle=\color{red},    % Color de los strings
  breaklines=true,            % Cortar las líneas largas
  frame=single,               % Colocar marco alrededor del código
  captionpos=b                % Posición del título
}
\renewcommand{\lstlistingname}{Código}





\begin{document}
    
    % Portada

\begin{titlepage}
    \begin{center}
        \textsc{\large Universidad Central de Venezuela}\\
        \textsc{\large Facultad de Ingeniería}\\
        \textsc{\large Escuela de Ingeniería Eléctrica}\\
        \textsc{\large Departamento de Electrónica, Computación y Control}\\[7cm]
        
        {\huge \bfseries INFORME N°1}\\[0.2cm]
        {\Large AMPLIFICADORES DISCRETOS}\\[10cm]
        
        \begin{minipage}{0.4\textwidth}
            \begin{flushleft}
                \emph{Aux. Docente:}\\
                Escobar, Caleb 
            \end{flushleft}
        \end{minipage}
        \begin{minipage}{0.4\textwidth}
            \begin{flushright}
                \emph{Autor:}\\
                Br. Axel Rivero
            \end{flushright}
        \end{minipage}
        
        \vfill
        
        {\large \today}
        
    \end{center}
\end{titlepage}

    
    %Tabla de contenidos

\tableofcontents

\newpage
    
    \section{Introducción}

    Las líneas de transmisión eléctricas juegan un papel fundamental en la infraestructura energética moderna, permitiendo la transferencia eficiente de energía desde los centros de generación hasta los puntos de consumo. No obstante, el diseño y la operación de estas líneas presentan una serie de desafíos técnicos que deben ser abordados para optimizar su rendimiento y minimizar las pérdidas de energía. Entre los principales problemas se encuentran las diversas pérdidas de energía que ocurren durante la transmisión, lo que puede reducir la cantidad de energía útil que llega al destino.

    Existen varios tipos de pérdidas en las líneas de transmisión, entre las cuales se destacan las pérdidas resistivas, las pérdidas por radiación, las pérdidas por calentamiento del dieléctrico y las pérdidas por efecto corona. Las pérdidas resistivas, ocurren debido a la resistencia inherente de los materiales conductores y están directamente relacionadas con la longitud de la línea y la corriente que fluye a través de ella. Por otro lado, las pérdidas por radiación y calentamiento del dieléctrico son más prominentes a frecuencias más altas y pueden reducirse mediante el uso de materiales adecuados y técnicas de blindaje. El efecto corona, que ocurre principalmente en líneas de alta tensión, genera pérdidas significativas de energía debido a la ionización del aire alrededor de los conductores.
    
    Este trabajo tiene como objetivo analizar las pérdidas resistivas en los conductores más utilizados en líneas de transmisión: cobre, aluminio y oro. Para ello, se ha implementado una simulación en Python, donde se modelan los efectos de la resistividad, el diámetro del conductor y la corriente sobre las pérdidas de energía. Los resultados obtenidos permitirán proponer soluciones viables para mejorar la eficiencia de las líneas de transmisión, enfocándose en la reducción de las pérdidas de potencia.
    
    Además, se evaluarán métodos para medir y mitigar las pérdidas mencionadas, brindando una visión más completa de las diferentes estrategias de optimización disponibles. Finalmente, se presentarán propuestas de diseño que permitan aumentar la eficiencia de las líneas de transmisión, siempre considerando la viabilidad técnica y económica de cada opción.




\newpage

    \section{Planteamiento del problema}

    Las líneas de transmisión son esenciales en la distribución de energía eléctrica desde los generadores hasta los consumidores. Sin embargo, durante este proceso, se producen diversas pérdidas que afectan la eficiencia del sistema de transmisión. Estas pérdidas no solo representan un costo económico significativo, sino que también impactan la estabilidad y confiabilidad del suministro eléctrico. Es crucial entender y mitigar estas pérdidas para mejorar la eficiencia energética y reducir los costos operativos.
    
    En el ámbito de la transmisión de energía, se identifican varios tipos de pérdidas que ocurren debido a diferentes factores:

    \begin{enumerate}
        \item Pérdida en el conductor.

        \item Pérdida por calentamiento del dieléctrico.

        \item Pérdida por radiación.

        \item Pérdida por acoplamiento.

        \item Efecto corona.
    \end{enumerate}

    \subsection{Objetivo}
    
        El objetivo de este estudio es investigar y analizar los procedimientos para medir cada uno de los tipos de pérdidas mencionados en las líneas de transmisión de energía. Además, se diseñará un montaje práctico en un laboratorio para medir uno de estos tipos de pérdidas, proporcionando así una metodología clara y precisa para su evaluación.

    \subsection{Preguntas de investigación}
    
        \begin{itemize}
            \item ¿Cuáles son los procedimientos más efectivos para medir la pérdida en el conductor en las líneas de transmisión?
            \item ¿Cómo se pueden cuantificar las pérdidas por calentamiento del dieléctrico?    
            \item ¿Qué métodos se utilizan para medir la pérdida por radiación en diferentes condiciones de transmisión?    
            \item ¿Qué técnicas existen para evaluar las pérdidas por acoplamiento entre líneas de transmisión?    
            \item ¿Cómo se puede medir y mitigar el efecto corona en las líneas de transmisión?
        \end{itemize}

    \subsection{Importancia}

        Este estudio es crucial para mejorar la eficiencia de los sistemas de transmisión de energía eléctrica. Al identificar y cuantificar adecuadamente las diversas pérdidas, se pueden desarrollar estrategias más efectivas para mitigarlas.





\newpage
    
    %Marco Teórico
\section{Marco Teórico}

    \subsection{Osciladores}

        Un oscilador en electrónica es un circuito que genera una señal eléctrica de forma periódica y autónoma. Algunas características clave de los osciladores son:

        \begin{itemize}
            \item Generan una señal alterna, normalmente una onda senoidal o rectangular. La frecuencia de oscilación depende de los componentes del circuito.
            \item No requieren una señal de entrada para funcionar, sino que generan la señal de forma autónoma mediante realimentación positiva.
            \item Constan de un amplificador y un circuito resonante (LC) en una configuración de realimentación positiva. El amplificador compensa las pérdidas del circuito resonante.
            \item Se utilizan en muchas aplicaciones como generadores de señales, relojes de circuitos digitales, sistemas de comunicaciones, etc.
        \end{itemize}

        \subsubsection{Osciladores sinusoidales}

            Son osciladores que generan una señal senoidal o casi senoidal. Estos osciladores emplean conceptos de la teoría de sistemas para crear un par de polos conjugados justo sobre el eje imaginario del plano complejo para mantener la oscilación sinusoidal sostenida. A pesar del nombre de osciladores lineales, se tiene que emplear alguna forma de no linealidad para obtener el control de la amplitud de la onda sinusoidal de salida. De hecho, todos los osciladores son esencialmente circuitos no lineales. Esto complica el trabajo de análisis y diseño de osciladores; ya que no es posible aplicar directamente métodos de transformación (plano s), pero se han perfeccionado técnicas por medio de las cuales se pueden obtener osciladores sinusoidales en dos pasos. El primero de estos es lineal, y fácilmente se pueden emplear métodos del dominio de la frecuencia para el análisis de circuitos de realimentación.

    \subsection{Lazo de realimentación de un oscilador}

        
        \begin{figure}[H]
            \centering
            \includegraphics[width=8cm]{Imagenes/1.png}
            \caption{Estructura básica de un oscilador con entrada igual a cero}
            \label{fig:1}
        \end{figure}
        
        Es la conexión entre la salida y la entrada de un oscilador que permite mantener y regenerar la señal oscilante. Proporciona la realimentación positiva necesaria para que el oscilador genere señales de forma autónoma y continua. La estructura básica de un oscilador sinodal consta de un amplificador en un lazo de realimentación positiva, como se muestra en el diagrama de bloques de la figura \ref{fig:1}. Es importante observar que, a diferencia del lazo de realimentación negativa, aquí la señal de realimentación \(x_f\) se suma con un signo positivo. Entonces, la ganancia de realimentación está dada por


        \begin{equation}
            A(s) = \frac{A(s)}{1 - A(s)\beta(s)} \label{eqn:7}
        \end{equation}

        Donde se observa el signo negativo en el denominador
        La ganancia de lazo del circuito de la figura es \( -A(s)\beta(s) \), pero para nuestro propósito aquí, es más conveniente cancelar el signo menos y definir la ganancia de lazo \( L(s) \) como

        \begin{equation}
            L(s) = A(s)\beta(s)
        \end{equation}


        La ecuación característica se convierte entonces en

        \begin{equation}
            -L(s) + 1 = 0
        \end{equation}
        
    \subsection{Criterio de Barkhausen}

        Si a una frecuencia específica \(f_0\) la ganancia de lazo \(A\beta\) es igual a la unidad, se deduce de la ecuación (7) que \(Af\) será infinita. Esto es, a esta frecuencia, el circuito tendrá una salida finita para la señal de entrada cero. Este circuito es, por definición, un oscilador. Entonces, la condición para que el lazo de realimentación de la ilustración 2 produzca oscilaciones sinodales de frecuencia \(w_0\) es que
        
        \begin{equation}
            A(s)\beta(s) = 1
        \end{equation}

        En otras palabras, el criterio de Barkhausen establece las condiciones necesarias para que un oscilador genere señales de forma sostenida:

    \subsection{Oscilador básico de puente de Wien}
    

        \begin{figure}[H]
            \centering
            \includegraphics[width=10cm]{Imagenes/3.png}
            \caption{Puente de Wien con su Diagrama de Bode en amplitud y fase}
            \label{fig:3}
        \end{figure}
        
        Es un oscilador que utiliza un amplificador operacional realimentado con un puente resistivo-capacitivo para generar oscilaciones sinusoidales. La frecuencia de oscilación depende de los valores de \(R\) y \(C\) del puente.

    \subsection{Control de Amplitud}
        Son técnicas utilizadas en los osciladores para estabilizar y limitar la amplitud de oscilación. Por ejemplo, mediante diodos en el circuito resonante o control automático de ganancia en el amplificador.

        \begin{figure}[H]
            \centering
            \includegraphics[width=8cm]{Imagenes/4.png}
            \caption{Oscildor con control de amplitud}
            \label{fig:4}
        \end{figure}

        El circuito de la figura \ref{fig:4} utiliza un circuito simple diodo-resistor para controlar el valor efectivo de R2. En niveles de señal bajos, los diodos están apagados, por lo tanto, la resistencia de \(100k\Omega\) no tiene ningún efecto. Entonces, se tiene que \(\frac{R2}{R1} = \frac{22.1}{10} = 2.21\), o bien \(T(jf_0) = \frac{(1 + 2.21)}{3} = 1.07 > 1\), lo que indica el surgimiento de la oscilación. Conforme la oscilación crece, los diodos son llevados de forma gradual a la conducción en medios ciclos alternados. En el límite de la conducción fuerte del diodo, efectivamente, \(R2\) cambiaría a \(18.1k\Omega\), donde se obtiene \(T(jf_0) = 0.937 < 1\). Sin embargo, antes de que alcance esta condición límite, la amplitud se estabilizará automáticamente en algún nivel intermedio de la conducción del diodo donde \(\frac{R2}{R1} = 2\) exactamente, o \(T(jf_0) = 1\).

        \subsection{Multivibradores}

            Son osciladores digitales que generan señales cuadradas o Rectangulares. Están formados por amplificadores realimentados con circuitos RC para controlar los tiempos de conmutación. Se clasifican en: Astable y monoestable.

            \subsubsection*{Multivibrador Astable}

                Multivibrador que no tiene estado estable, sino que conmuta continuamente entre
                dos estados. Genera una onda cuadrada periódica cuya frecuencia depende de los valores
                de \(R\) y \(C\).
                
                \subsubsection*{Multivibrador Monoestable}
                
                Multivibrador que tiene un estado estable y uno inestable. Al recibir un pulso de
                disparo cambia al estado inestable durante un tiempo determinado para luego volver al
                estado estable. Genera un único pulso rectangular por cada pulso de disparo.

            \subsection*{Generador de Funciones}

                Circuito electrónico capaz de generar diferentes formas de onda programables
                (senoidal, triangular, cuadrada, rampa, etc). Suele basarse en un oscilador controlado por
                voltaje (VCO).
                
                \subsection*{Histéresis}
                
                Es el fenómeno por el cual el estado de salida de un sistema depende de su estado
                anterior. Se utiliza en comparadores para evitar conmutaciones erráticas debido al ruido.
                
                \subsection*{Comparador}
                
                Circuito que compara dos señales de entrada y produce una salida determinada
                por la relación entre ellas. Tiene dos niveles de salida posibles, alto y bajo. Se usa en
                sistemas de instrumentación, convertidores A/D, etc.
                
                \subsection*{Comparador por Histéresis}
                
                Comparador diseñado intencionalmente para exhibir histéresis, los comparadores
                también pueden ser realimentados, solo que en su caso resulta más beneficiosa la
                realimentación positiva en la zona lineal, que la negativa comúnmente usada en los
                amplificadores operacionales. Al practicar la realimentación positiva en un comparador
                se obtiene fundamentalmente un nuevo comportamiento conocido como histéresis, en el
                cual los niveles de conmutación cambian con el estado (nivel de tensión) que se encuentre
                en dicho circuito.
                
                \subsection*{Generador de Onda Triangular}
                
                Circuito que genera una señal triangular, alternando rampas de subida y bajada. Se
                implementa cargando/descargando un capacitor entre dos niveles de voltaje con una
                constante de tiempo fija.




               

\newpage

    
    \section{Dispositivos} \label{sec:dispositivos}

    La transmisión de energía eléctrica a través de líneas de transmisión implica inevitablemente la disipación de una parte de esa energía en forma de calor, radiación y otras pérdidas. Estas pérdidas energéticas, que pueden reducir la eficiencia del sistema y generar problemas operativos, son causadas por diversos factores relacionados con las características de la línea, los materiales utilizados y las condiciones ambientales. Para cuantificar y analizar estas pérdidas de manera precisa, es fundamental emplear dispositivos de medición especializados. En esta sección, exploraremos en detalle los distintos instrumentos utilizados para medir las pérdidas en líneas de transmisión en los diferentes tipos de perdidas antes mencionados, lo que permitirá una evaluación exhaustiva del desempeño de los sistemas de transmisión y la identificación de áreas de mejora.

    \subsection{Medición de pérdidas en el conductor (Efecto Joule)}

        Las pérdidas en los conductores de una línea de transmisión, que son principalmente pérdidas resistivas (efecto Joule), pueden medirse utilizando diversos métodos, tanto directos como indirectos, dependiendo del equipo disponible, el tipo de línea, y la precisión deseada. A continuación te explico algunas de las formas más comunes para medir estas pérdidas:

        \subsubsection{Medición de la caída de voltaje}

            Uno de los métodos más sencillos para medir las pérdidas resistivas en un conductor es medir la caída de voltaje entre dos puntos de la línea de transmisión, generalmente entre la fuente y la carga.

            La pérdida resistiva puede calcularse como:

            \begin{gather}
                P_{perdida} = I \cdot \Delta V \label{eq:potencia_perd}
            \end{gather}

            Donde:
            \begin{align*}
                I &= \text{es la corriente que fluye por el conductor.} \\[0.2cm]
                \Delta V &= \text{es la caída de voltaje entre los dos puntos del conductor (medida con un voltímetro).}
            \end{align*}

            \begin{itemize}
                \item \textbf{Ventajas} 

                    Método directo y sencillo.  

                
                \item \textbf{Desventajas}

                    Es más útil para distancias cortas y situaciones donde el voltaje no es muy elevado.
            \end{itemize}

        \subsubsection{Medición de la resistencia del conductor y cálculo de pérdidas}

            Otra forma común de cuantificar las pérdidas resistivas en un conductor es medir la resistencia del conductor y luego calcular las pérdidas a partir de la corriente que fluye por el conductor.

            Las pérdidas resistivas se calculan como:

            \begin{gather}
                P_{resistivas} = I^2 R \label{eq:pot_resistiva}
            \end{gather}

            Donde:

            \begin{align*}
                I&= \text{es la corriente que circula por el conductor.}
                R & = \text{ es la resistencia del conductor.}
            \end{align*}

            \paragraph{Proceso:}
                \begin{enumerate}
                    \item Mide la resistencia del conductor mediante un ohmímetro o calculando su resistencia a partir de las propiedades conocidad del material.

                    \begin{gather}
                        R = \rho \dfrac{L}{A} \label{eq:rho}
                    \end{gather}

                    \item Mide la corriente que circula por el conductor.

                    \item Aplicar la fórmula \ref{eq:pot_resistiva} para calcular las pérdidas.
                \end{enumerate}

            \begin{itemize}
                \item \textbf{Ventaja}

                    Método preciso para pérdidas resistivas.

                \item \textbf{Desventaja}

                    Requiere conocer o medir la resistencia exacta del conductor, lo que puede ser complicado si las condiciones de temperatura varían, ya que la resistividad de los materiales cambia con la temperatura.
            \end{itemize}

        \subsubsection{Medición directa de la potencia de entra y salida}

            Este método consiste en medir la potencia de entrada en un extremo de la línea de transmisión y la potencia de salida en el otro extremo. La diferencia entre ambas potencias representa las pérdidas totales en la línea, incluidas las pérdidas resistivas en el conductor.

            \begin{gather}
                P_{perdida} = P_{entrada} - P_{salida} \label{eq:pot_perdida_2}
            \end{gather}

            \begin{itemize}
                \item \textbf{Ventajas}

                    Este método permite obtener una visión global de las pérdidas en la línea.

                    
                \item \textbf{Desventajas}

                    Las mediciones incluyen todas las pérdidas, por lo que no es fácil separar las pérdidas resistivas de otras.
                    
            \end{itemize}

        \subsubsection{Medición de la temperatura del conductor}

             Las pérdidas resistivas en los conductores provocan un aumento de la temperatura debido al efecto Joule. Midiendo el incremento de temperatura en el conductor, se puede estimar la potencia disipada.

             \paragraph{Proceso}

                \begin{enumerate}
                    \item Utiliza un termómetro infrarrojo o sensores de temperatura para medir la temperatura del conductor en condiciones de operación.
                    
                    \item Compara la temperatura medida con la temperatura ambiente o la temperatura del conductor cuando no circula corriente.
                    
                    \item Utiliza las propiedades térmicas del conductor para calcular la energía disipada como calor, lo que te da una idea de las pérdidas resistivas.
                    
                \end{enumerate}


            \begin{itemize}
                \item \textbf{Ventaja}
                
                     Es una forma indirecta de medir las pérdidas que no requiere desconectar la línea.

                     
                \item \textbf{Desventaja}

                     Depende de una buena calibración y comprensión del entorno de operación (pérdidas adicionales debidas a la radiación térmica, convección, etc.).
            \end{itemize}

        \subsubsection{Medición mediante técnicas de reflectometría (TDR)}

            La reflectometría en el dominio del tiempo (TDR) es una técnica utilizada para detectar fallos o cambios en la impedancia de una línea de transmisión, lo cual puede ayudar a identificar zonas con alta resistencia o pérdidas significativas.

            \paragraph{Proceso}

                \begin{enumerate}
                    \item Se envía un pulso de corriente o voltaje a lo largo del conductor.
                    
                    \item Se mide el tiempo que tarda el pulso en reflejarse desde los extremos o puntos de discontinuidad en la línea.
                    
                    \item Las reflexiones indican cambios en la impedancia que pueden estar asociados a pérdidas resistivas o defectos en la línea.
                \end{enumerate}

            \begin{itemize}
                \item \textbf{Ventaja}

                    Detecta pérdidas no uniformes a lo largo del conductor, como puntos calientes o conexiones defectuosas.
                    
                \item \textbf{Desventaja}

                    Requiere equipo especializado y es más útil para diagnósticos y mantenimiento que para una medición continua de pérdidas resistivas.
            \end{itemize}

        \subsubsection{Medición de pérdidas mediante análisis armónico}

            Las pérdidas resistivas pueden aumentar en presencia de armónicos, especialmente en sistemas de transmisión de corriente alterna (CA). Midiendo los armónicos presentes en la línea, se puede estimar cómo los armónicos contribuyen a las pérdidas adicionales en los conductores.

            \paragraph{Proceso}

                \begin{enumerate}
                    \item Se usa un analizador de armónicos para medir la distorsión de la forma de onda de la corriente en el conductor.
                    
                    \item Se analiza el contenido armónico y su impacto en las pérdidas resistivas.
                \end{enumerate}

            \begin{itemize}
                \item \textbf{Ventaja}

                    Este método es útil para sistemas donde las pérdidas pueden verse incrementadas por el uso de cargas no lineales.
                    
                \item \textbf{Desventaja}

                     Es un método indirecto y requiere analizar la contribución específica de los armónicos a las pérdidas.
            \end{itemize}

        \subsubsection{Medición con sensores en tiempo real (sensores de corriente y tensión)}

            En sistemas modernos, se pueden utilizar sensores inteligentes para medir corriente y voltaje en diferentes secciones de la línea de transmisión en tiempo real. Los sensores envían los datos a un sistema central que calcula las pérdidas de manera continua.

            \paragraph{Proceso}

                \begin{enumerate}
                    \item Instalar sensores de corriente y voltaje a lo largo de la línea de transmisión.
                    
                    \item Medir en tiempo real las condiciones operativas y calcular la potencia disipada en cada sección de la línea.
                    
                    \item Los sistemas inteligentes pueden detectar aumentos repentinos en las pérdidas y generar alertas.
                    
                \end{enumerate}

            \begin{itemize}
                \item \textbf{Ventaja}

                    Monitoreo en tiempo real de las pérdidas sin interrumpir el servicio.
                    
                \item \textbf{Desventaja}

                      Costo elevado y requerimiento de infraestructura adicional.
            \end{itemize}

            
        \subsubsection{Estrategias de mitigación}

            \begin{itemize}
                \item \textbf{Usar conductores de mayor diámetro:} Al aumentar el área transversal del conductor, se reduce su resistencia. Esto disminuye las pérdidas resistivas, ya que $R = \dfrac{\rho L}{A},$ donde $R$ es la resistencia, $\rho$ es la resistividad del material, $L$ la longitud y $A$ el área de la sección transversal.

                \item \textbf{Utilizar materiales conductores con menor resistividad:} El uso de conductores de materiales con menor resistividad, como el aluminio o el cobre, reduce las pérdidas. El cobre, por ejemplo, tiene una resistividad más baja que el aluminio, lo que lo hace más eficiente en términos de pérdidas resistivas.

                \item \textbf{Aumentar la conductancia:}   Utilizar aleaciones o técnicas de fabricación avanzadas para mejorar la conductancia de los materiales, minimizando así las pérdidas.
                
                \item \textbf{Reducir la longitud de las líneas: }   Si es factible, la reducción de la longitud de las líneas de transmisión disminuye la resistencia global y, por ende, las pérdidas resistivas.
            \end{itemize}




            
    \subsection{Medición de pérdidas por calentamiento del dieléctrico}

        Las líneas de transmisión son elementos fundamentales en los sistemas eléctricos de potencia, encargados de transportar energía eléctrica a grandes distancias. Sin embargo, durante este proceso, una parte de la energía se pierde en forma de calor debido a diversos factores, entre ellos, el calentamiento dieléctrico. Este fenómeno ocurre cuando el material aislante de la línea se somete a un campo eléctrico intenso, lo que puede generar pérdidas de energía y, en casos extremos, fallas en el sistema.

        Para garantizar la eficiencia y la seguridad de las líneas de transmisión, es crucial contar con dispositivos capaces de medir y monitorear las pérdidas por calentamiento dieléctrico. Estos dispositivos permiten detectar de manera temprana cualquier anomalía en el aislamiento, evitando así fallas catastróficas y minimizando los tiempos de inactividad del sistema.

        \subsubsection{Probadores de Pérdida Dieléctrica}

            Los medidores de pérdidas dieléctricas son ampliamente utilizados en laboratorios y en campo para evaluar la condición de materiales aislantes en cables, transformadores, condensadores y otros componentes eléctricos. Al identificar y cuantificar las pérdidas dieléctricas, estos instrumentos permiten detectar de manera temprana el deterioro del aislamiento, prevenir fallas catastróficas y garantizar la fiabilidad de los sistemas eléctricos. Además, los datos obtenidos a partir de estas mediciones son fundamentales para optimizar el diseño de nuevos materiales aislantes y mejorar la eficiencia energética de los equipos eléctricos

        \subsubsection*{Funcionamiento de los probadores de Pérdida Dieléctrica}

            \begin{enumerate}
                \item \textbf{Aplicación de Corriente Alterna:} El generador de corriente alterna aplica una señal de prueba al material dieléctrico.
                \item \textbf{Medición de Parámetros Eléctricos:} El medidor de voltaje y corriente registra los valores de voltaje y corriente aplicados al material.
                \item \textbf{Cálculo de Pérdida Dieléctrica:} La calculadora de pérdida analiza los datos registrados para determinar la cantidad de energía disipada en forma de calor.
                \item \textbf{Visualización de Resultados:} Los resultados se muestran en la pantalla del dispositivo, proporcionando una medida cuantitativa de la pérdida dieléctrica.
            \end{enumerate}

        \subsubsection*{Componentes de los probadores de Pérdida Dieléctrica}

            \begin{itemize}
              \item Generador de Corriente Alterna
              \item Medidor de Voltaje y Corriente
              \item Calculadora de Pérdida
            \end{itemize}

        \subsubsection{Cámaras Termográficas}

            Mediante el uso de cámaras termográficas, se capturan imágenes térmicas que revelan las variaciones de temperatura en la superficie de los equipos eléctricos. Posteriormente, mediante el empleo de software especializado, se analizan estas imágenes para identificar con precisión las áreas con anomalías térmicas, cuantificar la magnitud de dichas anomalías y determinar su potencial impacto en la operación del sistema. 

        \subsubsection*{Funcionamiento de las Cámaras Termográficas}

            \begin{enumerate}
                \item \textbf{Captura de Radiación Infrarroja:} La lente de la cámara termográfica captura la radiación infrarroja emitida por la línea de transmisión y la enfoca en el sensor.
                \item \textbf{Conversión a Señal Eléctrica:} El sensor de imagen infrarroja convierte la radiación capturada en una señal eléctrica proporcional a la temperatura de la superficie.
                \item \textbf{Procesamiento de Señal:} La electrónica de procesamiento convierte la señal eléctrica en una imagen térmica, que se muestra en la pantalla de la cámara.
                \item \textbf{Análisis de Imágenes Térmicas:} El software de análisis permite identificar puntos calientes en la línea de transmisión, que pueden indicar pérdidas de energía o problemas de aislamiento.
            \end{enumerate}

        \subsubsection*{Componentes de las Cámaras Termográficas}

            \begin{itemize}
              \item Lente
              \item Sensor de Imagen Infrarroja (Detector)
              \item Electrónica de Procesamiento
              \item Software de Análisis
            \end{itemize}

        \subsubsection{Medidores de Factor de Potencia}

            Los medidores de factor de potencia son instrumentos diseñados para evaluar la eficiencia con la que se utiliza la energía eléctrica en un sistema. Un factor de potencia bajo indica una elevada cantidad de potencia reactiva, lo cual puede generar pérdidas energéticas significativas en la transmisión y distribución. Mediante la medición y ajuste del factor de potencia, es posible optimizar la eficiencia del sistema eléctrico y minimizar las pérdidas. Estos dispositivos resultan fundamentales para identificar de manera temprana problemas que puedan comprometer la fiabilidad del sistema, permitiendo realizar acciones correctivas antes de que se produzcan fallas mayores.

        \subsubsection*{Funcionamiento de los medidores de Factor de Potencia}

            \begin{enumerate}
                \item \textbf{Medición de Corriente y Voltaje}: Los transformadores de corriente y voltaje miden las señales de corriente y voltaje en la línea de transmisión.
                \item \textbf{Cálculo del Factor de Potencia}: El circuito de medición utiliza las señales de corriente y voltaje para calcular el factor de potencia, que es la relación entre la potencia activa (real) y la potencia aparente.
                \item \textbf{Visualización de Resultados}: Los resultados se muestran en la pantalla del medidor, permitiendo al usuario evaluar la eficiencia del sistema.
            \end{enumerate}

        \subsubsection*{Componentes de los medidores de Factor de Potencia}

            \begin{itemize}
              \item Transformador de Corriente (CT)
              \item Transformador de Voltaje (VT)
              \item Circuito de Medición
              \item Pantalla
              \item Unidad de Procesamiento
            \end{itemize}

        \subsubsection{Medidores de Impedancia}

            Los medidores de impedancia son instrumentos diseñados para determinar la impedancia característica de las líneas de transmisión. Esta medición es fundamental para identificar desajustes que pueden provocar pérdidas de energía por calentamiento del dieléctrico y reflexiones de señal. Al garantizar que la impedancia se encuentre dentro de los parámetros establecidos, se optimiza la eficiencia de la transmisión eléctrica y se minimizan las pérdidas. Estos dispositivos son esenciales en las etapas de instalación y mantenimiento de líneas de transmisión, asegurando un funcionamiento óptimo del sistema.
            
        \subsubsection*{Funcionamiento de los Medidores de Impedancia}

            \begin{enumerate}
                \item \textbf{Aplicación de Señal de Prueba}: La fuente de señal genera una corriente alterna que se aplica al circuito bajo prueba.
                \item \textbf{Medición de Corriente y Voltaje}: Los transformadores de corriente y voltaje miden las señales de corriente y voltaje en el circuito.
                \item \textbf{Cálculo de Impedancia}: El circuito de medición utiliza las señales de corriente y voltaje para calcular la impedancia, que es la combinación de resistencia (R) y reactancia (X) en el circuito.
                \item \textbf{Visualización de Resultados}: Los resultados se muestran en la pantalla del medidor, permitiendo al usuario evaluar la impedancia del circuito.
            \end{enumerate}

        \subsubsection*{Componentes de los Medidores de Impedancia}

            \begin{itemize}
              \item Fuente de Señal
              \item Transformador de Corriente (CT)
              \item Transformador de Voltaje (VT)
              \item Circuito de Medición
              \item Pantalla
              \item Unidad de Procesamiento
            \end{itemize}

        \subsubsection{Estrategias de mitigación}

           \begin{itemize}
                \item \textbf{Utilizar dieléctricos de baja pérdida:} Seleccionar materiales aislantes con una tangente de pérdidas (tan $\delta$) baja. Materiales como polietileno reticulado (XLPE) o polipropileno tienen mejores propiedades dieléctricas y menor factor de pérdida que otros materiales más antiguos.

                \item \textbf{Mejorar la calidad del aislamiento:}  Un aislamiento más grueso y de mejor calidad reduce las pérdidas dieléctricas. Esto implica el uso de mejores recubrimientos y técnicas de aislamiento en los cables.

                \item \textbf{Control de la humedad:}  El agua en el dieléctrico aumenta las pérdidas. El uso de barreras impermeables y sellos puede evitar la absorción de humedad, lo que contribuye a mitigar las pérdidas.
            \end{itemize}



 

            

    \subsection{Medición de pérdidas por radiación}

        Si la separación entre los conductores de una línea de transmisión es una fracción apreciable de una longitud de onda, los campos electrostático y electromagnético que rodean a los conductores hacen que la línea actúe como una antena, irradiando energía hacia el medio circundante. La cantidad de energía radiada depende de diversos factores, tales como el material dieléctrico, la distancia entre conductores y la longitud de la línea. Para reducir las pérdidas por radiación, es común emplear blindajes metálicos, como en el caso de los cables coaxiales, los cuales presentan menores pérdidas en comparación con las líneas de dos hilos paralelos. Además, es importante destacar que las pérdidas por radiación son directamente proporcionales a la frecuencia de la señal transmitida. Para cuantificar estas pérdidas, se emplean dispositivos y técnicas de medición específicas, diseñadas para evaluar el rendimiento de las líneas de transmisión en diferentes condiciones operativas.

        \subsubsection{Analizadores de Red Vectorial (VNA):}

        Estos dispositivos son esenciales para medir parámetros S (scattering parameters), que ayudan a determinar las pérdidas de inserción y retorno en las líneas de transmisión. Los VNAs son muy precisos y se utilizan ampliamente en la caracterización de componentes de RF y microondas.
        
        \subsubsection*{Funcionamiento de los VNAs:}
            
            \begin{enumerate}
            
                \item \textbf{Generación de Señal:} El VNA genera una señal de RF o microondas que se envía a través de la línea de transmisión o el dispositivo bajo prueba (Device Under Test (DUT)).
            
                \item \textbf{Medición de Parámetros S:} Los VNAs miden los parámetros de dispersión (parámetros S), que describen cómo las señales se reflejan y transmiten a través del DUT. Los parámetros S más comunes son:
            
                \begin{itemize}
                    \item $S_{11}$: Coeficiente de reflexión de entrada, que indica la cantidad de señal reflejada de vuelta hacia la fuente.
                    \item $S_{21}$: Coeficiente de transmisión, que indica la cantidad de señal que pasa a través del DUT.
                \end{itemize}
            
                \item \textbf{Análisis de Reflexión y Transmisión:} Al medir $S_{11}$ y $S_{21}$, el VNA puede determinar las pérdidas de inserción (pérdida de señal al pasar a través del DUT) y las pérdidas de retorno (pérdida de señal reflejada de vuelta hacia la fuente).
            
                \item \textbf{Calibración:} Para obtener mediciones precisas, los VNAs deben ser calibrados. Esto implica usar estándares conocidos para corregir cualquier error sistemático en las mediciones.
            
                \item \textbf{Visualización y Análisis:} Los resultados se visualizan en una pantalla, mostrando gráficos como el diagrama de Smith, que ayuda a interpretar las características de impedancia y las pérdidas en la línea de transmisión.
            \end{enumerate}

        \subsubsection*{Componentes de los VNAs:}

            \begin{itemize}
                \item Generador de Señales
                \item Detectores de Señales
                \item Sistema de Conmutación
                \item Sistema de Control y Procesamiento de Datos
            \end{itemize}


        \subsubsection{Medidores de ROE (Relación de Onda Estacionaria)}

            Los medidores de ROE (Relación de Onda Estacionaria), o SWR (Standing Wave Ratio) en inglés, son instrumentos fundamentales en la evaluación de la eficiencia de sistemas de radiofrecuencia (RF). Estos dispositivos permiten cuantificar la adaptación de impedancias entre un transmisor y una línea de transmisión. Una ROE elevada indica una mala adaptación, lo que conlleva a una significativa reflexión de energía hacia el transmisor. Esta condición no solo disminuye la potencia efectiva radiada, sino que también puede sobrecargar componentes del equipo, acortando su vida útil. Mediante la medición y ajuste de la ROE, es posible optimizar el rendimiento de los sistemas RF, minimizando pérdidas y maximizando la eficiencia de la transmisión.

        \subsubsection*{Funcionamiento de los medidores de ROE:}

            \begin{enumerate}
                \item \textbf{Medición de Ondas Incidentales y Reflejadas:} El medidor de ROE mide la amplitud de las ondas incidentales (que viajan hacia la carga) y las ondas reflejadas (que regresan hacia la fuente) utilizando el acoplador direccional y los diodos detectores.
                \item \textbf{Cálculo de la ROE:} La relación de onda estacionaria se calcula como la relación entre la amplitud de la onda incidente y la onda reflejada. Una ROE de 1:1 indica una adaptación perfecta, mientras que valores más altos indican una mayor cantidad de energía reflejada.
                \item \textbf{Visualización de Resultados:} Los resultados se muestran en el indicador de ROE, permitiendo al usuario ajustar la antena o la línea de transmisión para minimizar las pérdidas.
            \end{enumerate}

        \subsubsection*{Componentes de medidores de ROE:}

            \begin{itemize}
                \item Acoplador Direccional
                \item Diodos Detectores
                \item Indicador de ROE
                \item Fuente de Alimentación
            \end{itemize}

        \subsubsection{Analizadores de Espectro}

            Los analizadores de espectro permiten visualizar y cuantificar el contenido espectral de señales de radiofrecuencia (RF) y microondas. Resultan indispensables en la caracterización de líneas de transmisión, ya que facilitan la identificación y cuantificación de pérdidas por radiación. Al analizar el espectro de las señales irradiadas desde la línea, se pueden detectar fugas no deseadas, evaluar la eficiencia de los sistemas de blindaje y localizar puntos de alta atenuación. La información obtenida a través de estos instrumentos es fundamental para optimizar el diseño y el rendimiento de las líneas de transmisión, minimizando pérdidas y asegurando la integridad de la señal.

        \subsubsection*{Funcionamiento de los Analizadores de Espectro:}
            
            \begin{enumerate}
                \item \textbf{Generación de Señal:} El analizador recibe una señal de radiofrecuencia (RF) o microondas desde la línea de transmisión que se desea analizar.
                \item \textbf{Conversión a Frecuencia Intermedia (FI):} La señal de entrada se mezcla con una señal de un oscilador local (LO) en un mezclador. Este proceso produce una nueva señal, la frecuencia intermedia (FI), que es la diferencia entre la frecuencia de la señal de entrada y la del LO. La FI se selecciona para facilitar el filtrado y la amplificación.
                \item \textbf{Filtrado y Amplificación:} La señal de FI pasa a través de un filtro que selecciona la banda de frecuencia de interés y atenúa las componentes fuera de banda. A continuación, se amplifica para mejorar la relación señal-ruido.
                \item \textbf{Detección:} La señal amplificada se detecta para obtener una señal de voltaje proporcional a la amplitud de la componente de frecuencia correspondiente.
                \item \textbf{Visualización:} La señal detectada se muestra en una pantalla, típicamente como un gráfico de amplitud versus frecuencia, lo que permite al usuario visualizar el espectro de la señal de entrada.
                \item \textbf{Análisis del Espectro:} El usuario puede analizar el espectro para determinar la frecuencia de las componentes, su amplitud relativa y la presencia de ruido o interferencias.
            \end{enumerate}

        \subsubsection*{Componentes de los Analizadores de Espectro:}

            \begin{itemize}
                \item Atenuador de Entrada
                \item Filtro de Paso Bajo
                \item Oscilador Local (OL)
                \item Mezclador
                \item Amplificador de FI
                \item Detector
            \end{itemize}

        \subsubsection{Cámaras Anecoicas}

            Las cámaras anecoicas son recintos especialmente diseñados para absorber prácticamente la totalidad de las ondas electromagnéticas incidentes, minimizando así las reflexiones y creando un entorno electromagnéticamente aislado. Este tipo de cámaras se utilizan de manera extensiva en pruebas de compatibilidad electromagnética (EMC) y caracterización de antenas, ya que permiten realizar mediciones precisas de la radiación emitida por dispositivos electrónicos, como líneas de transmisión. Al eliminar las interferencias externas y las reflexiones, se obtienen datos más confiables sobre las pérdidas por radiación, lo que a su vez facilita la optimización del diseño de los equipos y sistemas de radiofrecuencia.

        \subsubsection*{Funcionamiento de las Cámaras Anecoicas:}
            
            \begin{enumerate}
                \item \textbf{Absorción de Ondas Electromagnéticas:} Las paredes, el techo y el suelo de la cámara anecoica están recubiertos con materiales absorbentes que minimizan las reflexiones de las ondas electromagnéticas. Esto permite simular un espacio libre y realizar mediciones precisas de las características de radiación de antenas y dispositivos electrónicos.
                \item \textbf{Medición Precisa:} Dentro de la cámara, se colocan antenas de medición que emiten y reciben señales de radiofrecuencia. Estas mediciones permiten evaluar con precisión las pérdidas por radiación en líneas de transmisión, componentes electrónicos y sistemas completos.
                \item \textbf{Análisis de Resultados:} Los datos obtenidos de las mediciones se analizan utilizando software especializado para identificar puntos de alta radiación, evaluar la efectividad del blindaje electromagnético y determinar la compatibilidad electromagnética (EMC) de los dispositivos bajo prueba.
            \end{enumerate}

        \subsubsection*{Componentes de las Cámaras Anecoicas:}
            
            \begin{itemize}
              \item Jaula de Faraday
              \item Materiales Absorbentes
              \item Antenas de Medición
            \end{itemize}

        \subsubsection{Sondas de Campo Electromagnético}

            Las sondas electromagnéticas son instrumentos de medición utilizados para cuantificar la intensidad de los campos eléctricos y magnéticos en las proximidades de las líneas de transmisión. Estas sondas resultan indispensables para identificar y localizar puntos de alta radiación, lo que puede indicar la presencia de pérdidas significativas en la línea o un diseño inadecuado del blindaje. Además, permiten evaluar la eficacia de las medidas de protección implementadas para reducir las emisiones electromagnéticas no deseadas. La información recopilada mediante estas sondas es fundamental para optimizar el diseño de las líneas de transmisión, garantizar la compatibilidad electromagnética y minimizar los riesgos asociados a la exposición a campos electromagnéticos.

        \subsubsection*{Funcionamiento de las sondas de campo electromagnético:}

            \begin{enumerate}
                \item \textbf{Detección de Campos Electromagnéticos:} La sonda cuenta con una antena especialmente diseñada para captar los campos eléctricos y magnéticos generados por la fuente de radiación, como una línea de transmisión.
                \item \textbf{Conversión de Señal:} La señal electromagnética capturada por la antena se convierte en una señal eléctrica, generalmente una corriente o un voltaje, que puede ser procesada electrónicamente.
                \item \textbf{Amplificación y Procesamiento:} La señal eléctrica resultante es amplificada para mejorar su relación señal-ruido y luego procesada por circuitos electrónicos. Este procesamiento puede incluir filtrado, detección y conversión analógica-digital.
                \item \textbf{Visualización de Resultados:} Los resultados del procesamiento de la señal se muestran en una pantalla, ya sea en forma de valores numéricos (intensidad del campo en V/m o A/m) o en forma gráfica. Esta visualización permite al usuario identificar las áreas con mayor intensidad de campo y evaluar la efectividad de las medidas de protección contra las radiaciones electromagnéticas.
            \end{enumerate}

        \subsubsection*{Componentes de las sondas de campo electromagnético:}

            \begin{itemize}
              \item Antena de Captura
              \item Detector
              \item Amplificador
              \item Unidad de Procesamiento
            \end{itemize}

        \subsubsection{Estrategias de mitigación}

            \begin{itemize}
                \item \textbf{Diseño adecuado de las líneas de transmisión:}  Las líneas de transmisión deben diseñarse para minimizar el desajuste de impedancia, ya que esto puede aumentar las pérdidas por radiación. Se deben evitar cambios bruscos en la geometría de los conductores y realizar una correcta adaptación de impedancias.

                \item \textbf{Minimizar la longitud de los tramos sin apantallamiento: } Para líneas de transmisión en frecuencias más altas (como las de microondas), la instalación de blindaje o apantallamiento alrededor de los conductores puede reducir las pérdidas por radiación.

                \item \textbf{Uso de cables coaxiales o guías de onda:}   En sistemas de transmisión de alta frecuencia, el uso de cables coaxiales, guías de onda o cables apantallados es una forma efectiva de contener la radiación electromagnética y reducir estas pérdidas.
            \end{itemize}

    \subsection{Medición de pérdidas por acoplamiento}

        Las líneas de transmisión, encargadas de transportar energía eléctrica a grandes distancias, son susceptibles a interferencias electromagnéticas que pueden causar pérdidas de energía y degradar la calidad de la señal. Estas interferencias, conocidas como acoplamiento, se producen cuando campos electromagnéticos externos inducen corrientes parásitas en los conductores de la línea.
        
        Para garantizar la eficiencia y la confiabilidad de los sistemas de transmisión, es fundamental contar con dispositivos capaces de medir y monitorear las pérdidas por acoplamiento. Estos dispositivos permiten identificar las fuentes de interferencia, evaluar su impacto en la línea y tomar medidas correctivas para minimizar sus efectos.

        \subsubsection{Reflectómetro en el Dominio del Tiempo (TDR)}

            El reflectómetro en el dominio del tiempo (TDR) es un instrumento indispensable para caracterizar líneas de transmisión. Al analizar las reflexiones de pulsos electromagnéticos, el TDR permite identificar y localizar no solo pérdidas por acoplamiento, sino también discontinuidades como cortocircuitos, circuitos abiertos y cambios de impedancia, proporcionando información crucial para la evaluación de la integridad del sistema.

        \subsubsection*{Funcionamiento del TDR}

            \begin{enumerate}
                \item \textbf{Generación de Pulso}: El TDR envía un pulso de señal de alta frecuencia a lo largo de la línea de transmisión.
                \item \textbf{Reflexión de la Señal}: Si la línea tiene una impedancia uniforme y está correctamente terminada, el pulso se absorbe al final de la línea. Sin embargo, cualquier discontinuidad en la impedancia (como empalmes, conectores o defectos) reflejará parte del pulso de vuelta hacia el TDR.
                \item \textbf{Análisis de la Reflexión}: El TDR mide el tiempo que tarda la señal reflejada en regresar. Dado que la velocidad de propagación de la señal es constante para una impedancia dada, el tiempo de retorno se puede convertir en una distancia, localizando así la discontinuidad.
            \end{enumerate}

        \subsubsection*{Componentes del TDR}

            \begin{itemize}
                \item Generador de Pulso
                \item Transmisor/Receptor
                \item Osciloscopio o Display
                \item Procesador de Señal
            \end{itemize}

        \subsubsection{Puentes de medición}

            Los puentes de medición son dispositivos utilizados para medir parámetros eléctricos como resistencia, capacitancia e inductancia. Son especialmente útiles para detectar pérdidas por acoplamiento en líneas de transmisión. En el contexto de las líneas de transmisión, los puentes de medición pueden detectar pérdidas por acoplamiento al medir la impedancia en diferentes puntos de la línea. Al equilibrar el puente, se pueden identificar discontinuidades y variaciones en la impedancia que indican pérdidas de energía.

        \subsubsection*{Funcionamiento de los Puentes de medición}

            Los puentes de medición funcionan comparando una cantidad desconocida con una cantidad conocida. El principio básico es equilibrar el puente, de modo que la corriente a través del detector (generalmente un galvanómetro) sea cero. Este equilibrio indica que las proporciones de las resistencias (o impedancias) en los brazos del puente son iguales, permitiendo calcular el valor desconocido.

        \subsubsection*{Componentes de los Puentes de medición}
            
            \begin{itemize}
                \item Resistencias Conocidas
                \item Resistencia Desconocida
                \item Galvanómetro
                \item Fuente de Alimentación
                \item Ajustes Finos
            \end{itemize}

        \subsubsection{Osciloscopios de Alta Frecuencia}

            Los osciloscopios de alta frecuencia son instrumentos indispensables para el análisis detallado de señales eléctricas en líneas de transmisión. Al permitir visualizar y cuantificar las formas de onda, estos dispositivos facilitan la detección y caracterización de pérdidas por acoplamiento, distorsiones y otras no linealidades que pueden afectar la calidad de la señal. Esta información es crucial para optimizar el rendimiento de los sistemas de comunicación y garantizar la integridad de los datos transmitidos.

        \subsubsection*{Funcionamiento de los Osciloscopios de Alta Frecuencia}

            \begin{enumerate}
                \item \textbf{Captura de Señal}: Los osciloscopios de alta frecuencia capturan señales eléctricas a través de sondas conectadas a la línea de transmisión. Estas sondas pueden ser pasivas o activas, dependiendo de la frecuencia y la precisión requeridas.
                \item \textbf{Amplificación de Señal}: La señal capturada se amplifica para que pueda ser procesada y visualizada. Esto es crucial para señales de alta frecuencia que pueden ser muy débiles.
                \item \textbf{Digitalización}: La señal amplificada se convierte de analógica a digital mediante un convertidor analógico-digital (ADC). Esto permite un análisis más preciso y detallado.
                \item \textbf{Visualización}: La señal digitalizada se muestra en una pantalla, permitiendo observar la forma de onda en función del tiempo. Esto ayuda a identificar anomalías y pérdidas por acoplamiento en la línea de transmisión.
            \end{enumerate}

        \subsubsection*{Componentes de los Osciloscopios de Alta Frecuencia}

            \begin{itemize}
                \item Sondas
                \item Amplificador Vertical
                \item Convertidor Analógico-Digital (ADC)
                \item Pantalla
                \item Unidad de Procesamiento
                \item Controles de Tiempo y Voltaje
            \end{itemize}

        \subsubsection{Estrategias de mitigación}

            \begin{itemize}
                \item \textbf{Aumentar la separación entre líneas:} Incrementar la distancia entre las líneas de transmisión reduce el acoplamiento capacitivo e inductivo. Esto es particularmente importante en redes donde varias líneas están ubicadas cerca una de la otra.

                \item \textbf{Uso de apantallamiento entre conductores: } Colocar materiales conductores entre las líneas que puedan actuar como barreras físicas reduce la posibilidad de acoplamiento electromagnético. El blindaje metálico es efectivo para evitar el acoplamiento de señales entre conductores adyacentes.

                \item \textbf{Optimización del diseño de la red: }   El uso de configuraciones simétricas de las líneas y la torsión de los conductores ayuda a minimizar las pérdidas por acoplamiento. Además, la planificación cuidadosa de la red para evitar la proximidad entre líneas de alta potencia y líneas de señales sensibles reduce las pérdidas.
            \end{itemize}

    \subsection{Medición de pérdidas por efecto corona}

        Las líneas de transmisión de alta tensión son esenciales para el transporte de energía eléctrica a grandes distancias. Sin embargo, durante este proceso, pueden producirse pérdidas energéticas debido a diversos fenómenos físicos, uno de los cuales es el efecto corona. Este fenómeno se manifiesta como una descarga eléctrica parcial que ocurre en las proximidades de los conductores cuando el campo eléctrico excede un cierto valor crítico.
        
        El efecto corona tiene diversas consecuencias negativas, como la generación de ruido audible, interferencias radioeléctricas, producción de ozono y, lo más importante, pérdidas de energía. Para mitigar estos efectos y garantizar la eficiencia de las líneas de transmisión, es fundamental contar con dispositivos capaces de medir y monitorear el efecto corona.

        \subsubsection{Detectores de Fallos de Arco (AFDD)}

            Los detectores de fallas de arco (AFDD) son dispositivos de protección diseñados para detectar y desconectar circuitos afectados por arcos eléctricos. Su implementación en líneas de transmisión contribuye a la prevención de incendios, la protección de equipos y la mejora de la continuidad del servicio. Los AFDD permiten identificar proactivamente condiciones anómalas, facilitando las tareas de mantenimiento.

        \subsubsection*{Funcionamiento de los detectores de fallas de arco (AFDD)}

            \begin{enumerate}
                \item \textbf{Detección de Patrones de Onda}: Los AFDD monitorean continuamente las formas de onda de la corriente en la línea de transmisión. Detectan patrones de onda aleatorios y persistentes que indican la presencia de un arco eléctrico.
                \item \textbf{Análisis de Señal}: Utilizan algoritmos avanzados para analizar las señales eléctricas y diferenciar entre arcos peligrosos y otras perturbaciones eléctricas normales.
                \item \textbf{Desconexión del Circuito}: Cuando se detecta un patrón de onda que indica un arco peligroso, el AFDD se dispara y desconecta el circuito afectado para evitar daños mayores.
            \end{enumerate}

        \subsubsection*{Componentes de los detectores de fallas de arco (AFDD)}
            
            \begin{itemize}
                \item Sensor de Corriente
                \item Módulo de Análisis
                \item Interruptor de Corte
                \item Transformador de Secuencia Cero
                \item Indicadores de Estado
            \end{itemize}

        \subsubsection{Medidores de Descarga Parcial}

            Los medidores de descarga parcial son herramientas indispensables tanto en la fabricación como en la operación de equipos eléctricos. En la fase de fabricación, estos instrumentos garantizan que los componentes cumplan con los estándares de calidad, identificando cualquier defecto en el aislamiento. En servicio, los medidores de descarga parcial permiten monitorear continuamente el estado del aislamiento de las líneas de transmisión, detectando y cuantificando las descargas parciales para evaluar el riesgo de falla y programar el mantenimiento de manera proactiva.
            
        \subsubsection*{Funcionamiento de los Medidores de Descarga Parcial}

            \begin{enumerate}
                \item \textbf{Detección de Pulsos de Descarga}: Los medidores de descarga parcial detectan pulsos eléctricos que ocurren cuando hay una ruptura dieléctrica parcial en el aislamiento de la línea de transmisión.
                \item \textbf{Preamplificación y Digitalización}: La señal de entrada, que incluye el pulso de descarga parcial, se preamplifica y digitaliza mediante un convertidor analógico/digital.
                \item \textbf{Procesamiento de Señal}: La señal digitalizada se procesa mediante filtros digitales y algoritmos de detección para identificar y cuantificar las descargas parciales.
                \item \textbf{Análisis y Diagnóstico}: Los datos procesados se analizan para evaluar la integridad del aislamiento y determinar la ubicación y severidad de las descargas parciales.
            \end{enumerate}

        \subsubsection*{Componentes de los Medidores de Descarga Parcial}

            \begin{itemize}
            \item Sensores de Descarga Parcial (ultrasonido, HFCT, campo eléctrico)
            \item Preamplificador
            \item Convertidor Analógico/Digital (ADC)
            \item Unidad de Procesamiento
            \item Interfaz de Usuario
            \end{itemize}

        \subsubsection{Sensores de Campo Eléctrico}

            Los sensores de campo eléctrico son instrumentos de diagnóstico esenciales en el mantenimiento de líneas de transmisión. Al medir continuamente el campo eléctrico a lo largo de la línea, estos dispositivos permiten detectar de forma temprana anomalías que pueden indicar la presencia de descargas parciales, corrosión o daños en los conductores. Esta detección temprana facilita la planificación de un mantenimiento basado en la condición del equipo, optimizando la disponibilidad y confiabilidad del sistema eléctrico.

        \subsubsection*{Funcionamiento de los Sensores de Campo Eléctrico}

            \begin{enumerate}
                \item \textbf{Generación de Campo Eléctrico}: Los sensores de campo eléctrico funcionan creando un campo eléctrico entre dos electrodos. Cuando hay una variación en el campo eléctrico debido a una discontinuidad o defecto en la línea de transmisión, el sensor detecta esta variación.
                \item \textbf{Detección de Variaciones}: Los cambios en el campo eléctrico son convertidos en señales eléctricas que pueden ser analizadas. Estos cambios pueden ser causados por fallos de aislamiento, descargas parciales o pérdidas por acoplamiento.
                \item \textbf{Procesamiento de Señal}: Las señales eléctricas generadas por las variaciones en el campo eléctrico son procesadas y analizadas para determinar la naturaleza y ubicación del problema.
            \end{enumerate}

        \subsubsection*{Componentes de los Sensores de Campo Eléctrico}

            \begin{itemize}
                \item Electrodos
                \item Convertidor de Señal
                \item Unidad de Procesamiento
            \end{itemize}

        \subsubsection{Estrategias de mitigación}

            \begin{itemize}
                \item \textbf{Reducción de la resistencia y aumento de la eficiencia: } Asegurarse de que los conductores sean de alta calidad y bajo mantenimiento, lo que incluye la inspección y el reemplazo oportuno de conductores degradados.

                \item \textbf{Monitoreo en tiempo real:  } El uso de tecnologías como la termografía infrarroja, sensores y drones permite identificar áreas con alta generación de pérdidas, ya sea por sobrecalentamiento, desgaste o problemas de corona, permitiendo intervenciones preventivas.

                \item \textbf{Mantenimiento preventivo: }   Realizar un mantenimiento regular para evitar el deterioro de los conductores, los aisladores y otros componentes de la línea. El mantenimiento adecuado prolonga la vida útil del sistema y evita pérdidas adicionales.
            \end{itemize}
            


\newpage
    
    \section{Propuesta de diseño} \label{sec:propuesta}

    En este apartado el enfoque va dirigido a la simulación a realizar para poder observar, calcular y mitigar, en especial un tipo de pérdida en la línea de transmisión. Esta sera la pérdida resistiva del conductor como se estudia más adelante, de una manera detallada en las secciones \ref{sec:marco_teorico} y \ref{sec:dispositivos}. 

    La propuesta de diseño se desea plasmar a través de gráficas que demuestren que con diferentes materiales de un conductor se puede aminorar las perdidas de este, por ende, permite demostrar de una manera sencilla y simplificada, a través de los cálculos, lo que se puede detallar en la vida real simulando los datos que se tienen para la ejecución de un proyecto basado en el despliegue de una línea de transmisión.

    \subsection{Demostración}

        \begin{enumerate}
            \item \textbf{Facilidad de simulación:} las pérdidas resistivas dependen directamente de la resistencia del conductor, la corriente y la longitud de la línea, lo cual es sencillo de modelar y simular.

            \item \textbf{Disponibilidad de medidas:} se puede simular el comportamiento de una línea de transmisión aplicando las leyes básicas de Ohm y calcular las pérdidas de potencia en función de las propiedades del conductor.

            \item \textbf{Mitigación clara:} La mitigación se puede realizar al aumentar el diámetro del conductor, cambiar el material o mejorar la calidad del conductor, lo que permite incluir fácilmente en la simulación y observar los resultados.
        \end{enumerate}

    \subsection{Proceso}

        \begin{enumerate}
            \item Definir los parámetros de la línea de transmisión:

                \begin{itemize}
                    \item \textbf{Longitud de la línea:} $L$ en metros.

                    \item \textbf{Resistencia del conductor (Medida indirecta):} se utiliza la fórmula \ref{eq:rho}, $R=\rho        \dfrac{L}{A}$, donde:

                        \begin{align*}
                            \rho &= \text{es la resistividad del material (ejemplo, para el cobre,} \, \rho \approx 1.8 \times 10^{-8} \, \Omega \cdot m). \\[0.2cm]
                            A &= \text{es el área de la sección transversal del conductor.}
                        \end{align*}

                    \item \textbf{Corriente: } $I$ en amperios.
                    
                \end{itemize}

            \item \textbf{Cálculo de las pérdidas resistivas:} Se utiliza la fórmula de pérdidas resistivas \ref{eq:pot_resistiva}: $P_{perdida}=I^2R$

                \begin{itemize}
                    \item \textbf{Entrada:}  Se define un valor de corriente y calcula la resistencia de la línea.

                    
                    \item \textbf{Salida:} Calcula las pérdidas resistivas.
                \end{itemize}

            \item \textbf{Implementación de mitigación:}

                Como se observa en la formula \ref{eq:pot_resistiva}, las pérdidas resistivas o la potencia que se pierde es directamente proporcional a la resistencia, debido a esto se puede asumir que al cambiar el diámetro del conductor y recalcular las pérdidas, ya sea aumentando el área de la sección transversal $A$, disminuirá la resistencia $R$ y, por lo tanto, las pérdidas también. 

                De hecho, se puede aplicar otro escenario, que es al cambiar el material del conductor, siendo este la propiedad de la resistividad siendo proporcional a la resistencia, si se utiliza uno con menor resistividad (por ejemplo, cambiar de aluminio a cobre) y recalcular las perdidas. Como consecuencia se obtendrá una reducción en su perdida.

            \item \textbf{Simulación gráfica:}

                Se puede generar gráficos que muestren cómo varían las pérdidas en función de la corriente, el diámetro del conductor y/o la resistividad del material. Esto permitirá visualizar el efecto de la mitigación.
        \end{enumerate}

        Es importante acotar, que las simulaciones se llevará a cabo a través del lenguaje de programación Python, siendo esta la herramienta facilitadora de los cálculos y la gráfica pertinente.
    





\newpage

    
    \section{Simulaciones}
    
    A continuación, se hará uso del siguiente código para poder hacer los cálculos adecuados para el análisis de la simulación antes mencionada en la sección \ref{sec:propuesta}. 

    \subsection{Código en Python para la simulación de pérdidas resistivas}

        %\lstinputlisting[
        %  language=Python,
        %  basicstyle=\ttfamily\small,
         % caption={Código Python para la simulación de pérdidas en el conductor},
          %label={lst:sim_perdidas}
        %]{code/sim_perdidas.py}
    

    De acuerdo con los cálculos obtenidos del código \ref{lst:sim_perdidas}, los valores de cada parámetro y, la gráfica se mostraran a continuación, de esta manera poder analizar los resultados obtenidos de la simulación realizada.

        \begin{figure}[H]
            \centering
            \setcounter{figure}{0}
            \renewcommand{\figurename}{Gráfica}
            \includegraphics[width=\textwidth]{imagenes/perdidas_resistivas.png}
            \caption{Pérdidas resistivas en función del diámetro del conductor para distintos materiales y corrientes}
            \label{fig:Perdidas_resistivas}
        \end{figure}

       \begin{figure}[H]
           \centering
           \setcounter{figure}{0}
           \renewcommand{\figurename}{Tabla}
           \includegraphics[width=\textwidth]{imagenes/sim_datos_perdidas.png}
           \caption{Tabla de datos de la simulación de Pérdidas resistivas en función del diámetro del conductor para distintos materiales y corrientes}
           \label{fig:sim_datos_perdidas}
       \end{figure}

      Cómo se puede apreciar en la Gráfica \ref{fig:Perdidas_resistivas} y la  Tabla \ref{fig:sim_datos_perdidas}, existe unas diferencias notables, en especial en la Gráfica \ref{fig:Perdidas_resistivas}, evidenciando a simple vista que los diferentes tipos de materiales si son importantes para la mitigación de las pérdidas resistivas del conductor. 

      Por otro lado, los distintos cambios realizado a los parámetros como lo fue la corriente, y su diámetro, trae como consecuencia la acción de aumentar o disminuir las pérdidas; seguidamente sus valores de resistencia, calculado indirectamente debido a los cambios antes efectuado, por ende, se obtuvo una aproximación de ¿Cuál es el material que va a mitigar las perdidas resistivas?

      Como se observa en la figura \ref{fig:awg}, los estándares en calibres van de 0.3606 (mm) a 11.86 (mm), es decir, se hará un énfasis a los valores de la tabla \ref{fig:sim_datos_perdidas}, que serán del 10 (mm) hasta los 30 (mm).

      Con una corriente de $50 \, A$, el cobre obtuvo una disminución con los distintos parámetros, observándose una disminución en sus resistencia y pérdidas, cuando su diámetro aumentaba, al igual que los demás elementos, sin embargo, de estos materiales quien le sigue la menor pérdida sería el oro, por ultimo el aluminio.

      Así mismo, va ocurriendo al aumentar la corriente, demostrando que existe mayor pérdida al tener mas corriente debido al movimiento de carga en el conductor, produciendo en este caso el efecto Joule.

      






\newpage
    
    \section{Conclusiones y posibles aplicaciones}


   En este trabajo se simuló y analizó un caso hipotético en el que se evaluaron las pérdidas resistivas en diferentes conductores, utilizando Python para modelar las propiedades resistivas de cada material. Los conductores analizados fueron cobre, aluminio y oro, considerando distintos valores de resistividad, diámetro y corriente. Esta simulación permitió obtener una visión simplificada de cómo la resistividad y el diámetro del conductor afectan las pérdidas de potencia en una línea de transmisión.

    De acuerdo con los resultados obtenidos, se concluyó que el mejor material para mitigar las pérdidas resistivas en los conductores es el cobre. Si bien el oro también podría ofrecer buenos resultados debido a su baja resistividad, su alto costo y uso limitado en aplicaciones específicas de electrónica lo hacen menos viable para aplicaciones de transmisión a gran escala. Al analizar los datos de la tabla \ref{fig:sim_datos_perdidas}, se observó que las pérdidas con el cobre fueron menores en un $31.15 \%$ en comparación con el aluminio, lo que resulta en una menor resistencia total. Se evidenció que la resistencia es directamente proporcional a las pérdidas resistivas y que estas pueden reducirse aumentando el área transversal del conductor (en este caso, con un cable de longitud fija de 1000 m), ya que un mayor diámetro reduce las pérdidas resistivas.
    
    Por otra parte, tanto el aluminio como el oro pueden descartarse como materiales preferidos para una línea de transmisión, lo que coincide con las prácticas industriales donde el cobre es el material de referencia para la mayoría de los cables conductores, aunque existen excepciones en situaciones específicas.
    
    En cuanto a los métodos de medición de las pérdidas, la elección del procedimiento más adecuado dependerá de las herramientas disponibles, la complejidad del proyecto y la experiencia del personal a cargo. Sin embargo, es importante destacar que las pérdidas en una línea de transmisión pueden medirse de manera simplificada comparando la potencia de entrada con la potencia de salida. Cualquier diferencia entre ambas representaría la cantidad de energía disipada en forma de calor debido a las pérdidas resistivas.
    
    Finalmente, en la sección \ref{sec:dispositivos} se presentaron distintos dispositivos y técnicas para mitigar las pérdidas en sistemas eléctricos, proporcionando alternativas para mejorar la eficiencia. Aunque inicialmente se consideró la posibilidad de simular los circuitos mediante herramientas como Multisim, Proteus o LTSpice, se optó por un enfoque más eficiente y directo, centrado en la simulación de las pérdidas resistivas en los conductores, lo que permitió demostrar con claridad los efectos de la resistividad y el diámetro en la eficiencia del sistema.


\newpage

    
    \printbibliography

    %\nocite{*}
    
\label{Lastpage}

\end{document}

